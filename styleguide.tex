% arara: makeindex

% Template for IEEE papers
%% bare_conf.tex
%% V1.4b
%% 2015/08/26
%% by Michael Shell
%% See:
%% http://www.michaelshell.org/
%% for current contact information.
%%
%% This is a skeleton file demonstrating the use of IEEEtran.cls
%% (requires IEEEtran.cls version 1.8b or later) with an IEEE
%% conference paper.
%%
%% Support sites:
%% http://www.michaelshell.org/tex/ieeetran/
%% http://www.ctan.org/pkg/ieeetran
%% and
%% http://www.ieee.org/

%%*************************************************************************
%% Legal Notice:
%% This code is offered as-is without any warranty either expressed or
%% implied; without even the implied warranty of MERCHANTABILITY or
%% FITNESS FOR A PARTICULAR PURPOSE!
%% User assumes all risk.
%% In no event shall the IEEE or any contributor to this code be liable for
%% any damages or losses, including, but not limited to, incidental,
%% consequential, or any other damages, resulting from the use or misuse
%% of any information contained here.
%%
%% All comments are the opinions of their respective authors and are not
%% necessarily endorsed by the IEEE.
%%
%% This work is distributed under the LaTeX Project Public License (LPPL)
%% ( http://www.latex-project.org/ ) version 1.3, and may be freely used,
%% distributed and modified. A copy of the LPPL, version 1.3, is included
%% in the base LaTeX documentation of all distributions of LaTeX released
%% 2003/12/01 or later.
%% Retain all contribution notices and credits.
%% ** Modified files should be clearly indicated as such, including  **
%% ** renaming them and changing author support contact information. **
%%*************************************************************************


% *** Authors should verify (and, if needed, correct) their LaTeX system  ***
% *** with the testflow diagnostic prior to trusting their LaTeX platform ***
% *** with production work. The IEEE's font choices and paper sizes can   ***
% *** trigger bugs that do not appear when using other class files.       ***                          ***
% The testflow support page is at:
% http://www.michaelshell.org/tex/testflow/

\documentclass{book}
\usepackage[quiet]{fontspec}
\usepackage[table,xcdraw,dvipsnames]{xcolor} % Used by spritegrid and others.
\usepackage[obeyspaces,spaces]{url}
\usepackage{longtable}
\usepackage{arydshln}
\usepackage{booktabs}
\usepackage{afterpage}
\usepackage{flushend}
\usepackage{titletoc}
\usepackage[toc]{appendix}
\usepackage{parskip}
\usepackage{graphicx,wrapfig}
\usepackage{float}
\usepackage{caption}
\usepackage{pdfpages}
\usepackage{tikzpagenodes}
\usepackage{imakeidx}
\usepackage[pagestyles,raggedright]{titlesec}
\usepackage[all]{nowidow}
\usepackage[bookmarks=true,linktoc=all]{hyperref}
\hypersetup{
  colorlinks   = true, %Colours links instead of ugly boxes
  urlcolor     = blue, %Colour for external hyperlinks
  % Each main .tex file configures via \titleformat the \chapter command
  % to do {\chapmtoc\insertminitoc} and \chapmtoc, as defined below, will
  % use \hypersetup{linkcolor=white} to avoid blue-on-blue TOC links
  % Besides, each main .tex file will issue the \tableofcontents command
  % between \hypersetup{linkcolor=black} and \hypersetup{linkcolor=blue}
  % This means however that if "blue" is modified here it must be modified
  % in these files too.
  linkcolor    = blue, %Colour of internal links
  citecolor   = red %Colour of citations
}
\usepackage{aeb-minitoc}
\usepackage{fix-cm}
\usepackage{textpos}
\usepackage{enumitem}
\usepackage{tcolorbox}
\tcbuselibrary{breakable,listings,skins,xparse}
%\usepackage{wrapfig}
\usepackage{needspace}
\usepackage{verbatim}
\usepackage{ean13isbn}
\usepackage{setspace}

% Use CHAPTER-PAGE page numbering to make it easier to modify chapters
% later, without messing up page number of the rest of the book.
\usepackage[auto]{chappg}

% Allow cross-references between the various books to the big The MEGA65 Book
\usepackage{xr}
\usepackage{varioref}
\usepackage{xparse}
\externaldocument[M65Book-]{mega65-book}
% And a \ref alternative that checks if it needs to be a cross-reference to the
% MEGA65 Book instead.
\makeatletter
\newcommand{\bookref}[1]{%
    \@ifundefined{r@#1}{%
      {\em the MEGA65 Book}, \nameref{M65Book-#1} (\autoref{M65Book-#1})}{\autoref{#1}}%
}
\newcommand{\bookvref}[1]{%
    \@ifundefined{r@#1}{%
      {\em the MEGA65 Book}, \nameref{M65Book-#1} (\autoref{M65Book-#1})}{Chapter/Appendix \vref{#1}}%
}
\makeatother

% For fixed-width columns in register maps
\usepackage{array}

% Makes tables with double-ruled lines look better
\usepackage{hhline}

% Makes better use of space for reference tables in appendix
\usepackage{multicol}

% Shaded tables with alternate rows colored for better legibility
% Best used with larger tables rather than small tables
\usepackage{colortbl}
\usepackage{adjustbox}
\usepackage[strict]{changepage}

% \makecell command for forcing line breaks in table cells
\usepackage{makecell}

\newcolumntype{L}[1]{>{\raggedright\let\newline\\\arraybackslash\hspace{0pt}}m{#1}}
\newcolumntype{C}[1]{>{\centering\let\newline\\\arraybackslash\hspace{0pt}}m{#1}}
\newcolumntype{R}[1]{>{\raggedleft\let\newline\\\arraybackslash\hspace{0pt}}m{#1}}

% clear to left page for making two page tables starting on the odd page
\newcommand{\cleartoleftpage}{%
  \clearpage
  \ifodd\value{page}\hbox{}\newpage\fi
}

% For displaying Letter keys and the MEGA key
% This is a `keys' element for displaying a Mega65 keyboard key
% using a black filled label with rounded edges.
% In order to display a key as a title, use:
%
%     \megakey[title]{Run/Stop}
%
% For displaying a key as a part of the normal document flow, simply use:
%
%    \specialkey{SHIFT}
%
%
% If you get warnings on special characters, mathematical characters etc, use $, eg:
%
%    \megakey{$\leftarrow$}
%
% Other sizes are supported, as part of tcolorbox:
% http://mirror.aarnet.edu.au/pub/CTAN/macros/latex/contrib/tcolorbox/tcolorbox.pdf#subsubsection.4.7.5 however, only `title' and the default: `small' are proposed for use in this manual.
%
% The second macro available here is the megasymbolkey.
% This will display the MEGA symbol as white on a black key box. Simply use:
%
%		 \megasymbolkey
%
% Some MEGA65 keys contain two lines of text like "RUN/STOP"
% You can use the specialkey macro for this:
%
%    \specialkey{SHIFT LOCK}%

\usepackage{tcolorbox}

\newtcbox{\megakeyinner}[1][small]{colback=black, coltext=white, size=#1, fontupper=\bfseries, nobeforeafter,box align=bottom,baseline=3pt,text height=7pt,valign=center}
\newcommand{\megakey}[2][small]{\megakeyinner[#1]{\uppercase{#2}}}

\newtcbox{\megakeyinnerwhite}[1][small]{colback=white, coltext=black, size=#1, fontupper=\bfseries, nobeforeafter,box align=bottom,baseline=3pt,text height=7pt,valign=center}
\newcommand{\megakeywhite}[2][small]{\megakeyinnerwhite[#1]{\uppercase{#2}}}

% Previous version of megasymbolkey
%\newtcbox{\megasymbolkeyinner}{colback=black, coltext=white, clip title=false. fontupper=\symbolfont, box align=bottom,baseline=3pt,text height=7pt}
%\newcommand{\megasymbolkey}{\megakeyinner{\megasymbol[white]}\ }

\newtcolorbox{megasymbolkeyinner}
{colback=black,coltext=white,size=small,fontupper=\small\bfseries,
width=0.65cm, height=0.55cm, box align=base,
nobeforeafter, halign=flush left, left=0mm,top=0.3mm,bottom=0mm,right=0mm
,boxsep=0.5mm,baseline=4pt, enlarge right by = 1mm
}
\newcommand{\megasymbolkey}{
\begin{megasymbolkeyinner}%
\megasymbol[white]%
\end{megasymbolkeyinner}%
}

\newtcolorbox{specialkeyinner}
{colback=black,coltext=white,size=small,fontupper=\tiny\bfseries,
width=0.80cm, height=0.55cm, box align=base,
nobeforeafter, halign=flush left, left=0mm,top=0.3mm,bottom=0mm,right=0mm
,boxsep=0.5mm,baseline=4pt
}
\newcommand{\specialkey}[1]{
\begin{specialkeyinner}%
#1%
\end{specialkeyinner}%
}

\newtcolorbox{widekeyinner}
{colback=black,coltext=white,size=small,fontupper=\tiny\bfseries,
width=0.9cm, height=0.55cm, box align=base,
nobeforeafter, halign=flush left, left=0mm,top=0.3mm,bottom=0mm,right=0mm
,boxsep=0.5mm,baseline=4pt
}
\newcommand{\widekey}[1]{
\begin{widekeyinner}%
#1%
\end{widekeyinner}%
}




% For displaying print versions petscii character symbols
\input{elements/graphicsymbol}

% For Mega65 display of code, listings and screen activity
% This is a collection of elements for displaying output from the Mega65 screen.
% They can display program code or fragments to show activity on the screen.
% Example of use:
%
%    \begin{screencode}
%    10 OPEN 1,8,0,"$0:*,P,R
%    20 : IF DS THEN PRINT DS$: GOTO 100
%    30 GET#1,X$,X$
%    40 DO
%    50 : GET#1,X$,X$: IF ST THEN EXIT
%    60 : GET#1,BL$,BH$
%    70 : LINE INPUT#1, F$
%    80 : PRINT LEFT$(F$,18)
%    90 LOOP
%    100 CLOSE 1
%
%    RUN
%    \end{screencode}
%
% for inline display of code, use:
%
%    \screentext{?SYNTAX ERROR}
%

\usepackage{listings,color}

\lstnewenvironment{screenoutputlined}
   {
     \lstset{
               basicstyle=\codefont\color{white}\linespread{1.0}\normalsize,
               backgroundcolor=\color{black},fillcolor=\color{black},
               rulecolor=\color{black},
               frame=lines,
               framexleftmargin=2mm,
               framexrightmargin=2mm,
               framextopmargin=2mm,
               framexbottommargin=2mm,
               tabsize=4,
               xleftmargin=2mm,
               xrightmargin=2mm,
               basewidth={0.4em},
               literate={\*}{*}1{\-}{-}1{\/}{/}1{{\ }}{{ }}1
            }
   }
   {  }

\lstdefinestyle{megalisting}{basicstyle=\codefont\normalsize,breaklines=false,fontadjust=true,basewidth=1.5mm}
\makeatletter
\newtcblisting{screencode}{%
listing only,
colback=black,
coltext=white,
boxsep=0mm,
left=2mm,
right=0mm,
top=-1mm,
bottom=-1mm,
listing options={style=megalisting},
% Gets ignored by listings package
%fontupper=,
%enlarge left by =\csname @totalleftmargin\endcsname
}
\makeatother

% Stop - signs in listings getting turned into minus characters
\makeatletter
\lst@CCPutMacro
    \lst@ProcessOther {"2D}{\lst@ttfamily{-{}}{-}}
    \@empty\z@\@empty
% Also stop * being pushed down or faultily verically centred
\lst@CCPutMacro
    \lst@ProcessOther {"2A}{%
      \lst@ttfamily
         {{*}}% used with ttfamily
         {*}}% used with other fonts
    \@empty\z@\@empty
\makeatother


% For in-line screen text
\newcommand{\screentext}[1]{{\codefont\color{black}\normalsize{#1}}}
\newcommand{\screentextwide}[1]{{\codefontwide\color{black}\small{#1}}}
% Just to save typing
\newcommand{\stw}[1]{{\codefontwide\color{black}\small{#1}}}

% 45GS02 assembler mneomics and Acme directives
\lstdefinelanguage[45gs02]{Assembler}%
  {morekeywords={%
    adc,and,asl,asr,asw,aug,bbr0,bbr1,bbr2,%
    bbr3,bbr4,bbr5,bbr6,bbr7,bbs0,bbs1,bbs2,%
    bbs3,bbs4,bbs5,bbs6,bbs7,bcc,bcs,beq,%
    bit,bmi,bne,bpl,bra,brk,bsr,bvc,%
    bvs,clc,cld,cle,cli,clv,cmp,cpx,%
    cpy,cpz,dec,dew,dex,dey,dez,eom,%
    eor,inc,inw,inx,iny,inz,jmp,jsr,%
    lda,ldx,ldy,ldz,lsr,map,neg,nop,%
    ora,pha,php,phw,phx,phy,phz,pla,%
    plp,plx,ply,plz,rmb0,rmb1,rmb2,rmb3,%
    rmb4,rmb5,rmb6,rmb7,rol,ror,row,rti,%
    rts,sbc,sec,sed,see,sei,smb0,smb1,%
    smb2,smb3,smb4,smb5,smb6,smb7,sta,stx,%
    sty,stz,tab,tax,tay,taz,tba,trb,%
    tsb,tsx,tsy,txa,txs,tya,tys,tza,%
    adcq,andq,aslq,asrq,bitq,cmpq,deq,eorq,%
    inq,ldq,lsrq,orq,rolq,rorq,sbcq,stq},%
  morekeywords=[2]{%
    !8,!08,!by,!byte,!16,!wo,!word,!le16,%
    !be16,!24,!le24,!be24,!32,!le32,!be32,!hex,%
    !h,!fill,!fi,!skip,!align,!convtab,!ct,!text,%
    !tx,!pet,!raw,!scr,!scrxor,!to,!source,!src,%
    !binary,!bin,!zone,!zn,!symbollist,!sl,!if,!ifdef,%
    !ifndef,!for,!set,!do,!while,!endoffile,!eof,!warn,%
    !error,!serious,!macro,!initmem,!xor,!pseudopc,!cpu,!al,!as,!rl,!rs,!address,!addr,
  },%
  alsoletter=.,%
  alsodigit=?,%
  sensitive=f,%
  morestring=[b]",%
  morestring=[b]',%
  morecomment=[l]{;}%
  }[keywords,comments,strings]

\lstdefinelanguage[MEGA65]{Basic}%
  {morekeywords={%
    end,for,next,data,input\#,input,dim,read,%
    let,goto,run,if,restore,gosub,return,rem,%
    stop,on,wait,load,save,verify,def,poke,%
    print\#,print,cont,list,clr,cmd,sys,open,%
    close,get,new,tab,to,fn,spc,then,%
    not,step,+,-,*,/,\^,and,%
    or,>,=,<,sgn,int,abs,usr,%
    fre,pos,sqr,rnd,log,exp,cos,sin,%
    tan,atn,peek,len,str\$,val,asc,chr\$,%
    left\$,right\$,mid\$,go,rgraphic,rcolor,joy,rpen,%
    dec,hex\$,err\$,instr,else,resume,trap,tron,%
    troff,sound,vol,auto,import,graphic,paint,char,%
    box,circle,paste,cut,line,merge,color,scnclr,%
    xor,help,do,loop,exit,dir,dsave,dload,%
    header,scratch,collect,copy,rename,backup,delete,renumber,%
    key,monitor,using,until,while,bank,filter,play,%
    tempo,movspr,sprite,sprcolor,rreg,envelope,sleep,catalog,%
    dopen,append,dclose,bsave,bload,record,concat,dverify,%
    dclear,sprsav,collision,begin,bend,window,boot,fread\#,%
    wpoke,fwrite\#,dma,edma,mem,off,fast,speed,%
    type,bverify,ectory,erase,find,change,set,screen,%
    polygon,ellipse,viewport,gcopy,pen,palette,dmode,dpat,%
    format,turbo,foreground,background,border,highlight,mouse,rmouse,%
    disk,cursor,rcursor,loadiff,saveiff,edit,font,fgoto,%
    fgosub,mount,freezer,chdir,dot,info,bit,unlock,%
    lock,mkdir,<<,>>,vsync,pot,bump,%
    lpen,rsppos,rsprite,rspcolor,log10,rwindow,pointer,mod,%
    pixel,rpalette,rspeed,rplay,wpeek,decbin,strbin\$%
  },%
  sensitive=f,%
  morestring=[b]",%
  morecomment=[l]{rem }%
  }[keywords,comments,strings]

\lstnewenvironment{asmcode}{
  \lstset{
    language=[45gs02]Assembler,
    basicstyle=\ttfamily\normalsize,
    xleftmargin=4mm}
}{}
\newcommand\asminput[2][]{%
  \lstinputlisting[
    language=[45gs02]Assembler,
    basicstyle=\ttfamily\normalsize,
    xleftmargin=4mm,
    #1]{#2}
}

\lstnewenvironment{basiccode}{
  \lstset{
    language=[MEGA65]Basic,
    basicstyle=\ttfamily\normalsize,
    xleftmargin=4mm}
}{}
\newcommand\basicinput[2][]{%
  \lstinputlisting[
    language=[MEGA65]Basic,
    basicstyle=\ttfamily\normalsize,
    xleftmargin=4mm,
    #1]{#2}
}


% For MEGA65 screen shots with text flow
\input{elements/screenshots}

% For displaying sprite data in a grid
\input{elements/spritegrid}

% Don't number sections
\setcounter{secnumdepth}{0}

\renewcommand{\indexname}{INDEX}
\renewcommand{\appendixtocname}{APPENDICES}
\renewcommand{\appendixpagename}{APPENDICES}
\renewcommand{\appendixpage}{%
  \clearpage\thispagestyle{empty}
    \pagecolor{blue}
     \begin{center}
       {
         \large
         % Put a nice amount of vertical space before the title
         \vspace*{2cm}
               {\large\Huge\textcolor{white}{\bf{APPENDICES}}}\\
             \vspace{\fill}
       }
     \end{center}
     \newpage\pagecolor{white}\clearpage
}

\makeatletter\chardef\pdf@shellescape=\@ne\makeatother

\setcounter{tocdepth}{5}

% 1.0 cm is the distance from left of page to bullet point.
% 2.8 cm is a fudge-factor to make multi-line section names be correctly lined up.
% \@B{〈length〉} is the amount to indent prior to〈sec-num >
% \@F{〈fmt〉} is the formatting for the title heading
% \@P{〈fmt〉} is the formatting for the page number (〈pg-num〉).

\TOCLevels{chapter}{section}
\begin{minitocfmt}{\chapmtoc}
\declaretocfmt{section}{\@F{\color{white}\hypersetup{linkcolor=white}\hspace{1.0cm}\textbullet\hspace{0.25cm}\Large\bfseries}\@B{2.8cm}\@P{\mtocgobble}}
\declaretocfmt{section*}{\@F{\color{white}\hypersetup{linkcolor=white}\hspace{1.0cm}\textbullet\hspace{0.25cm}\Large\bfseries}\@B{2.8cm}\@P{\mtocgobble}}
\end{minitocfmt}

\usepackage{fontspec}
\usepackage{courier}

\setmainfont[Path=fonts/, BoldFont=MegaGlacial-Bold.otf, ItalicFont=MegaGlacial-Italic.otf]{MegaGlacial-Regular.otf}
\setmonofont[Path=fonts/, BoldFont=Inconsolata-Bold.ttf]{Inconsolata-Regular.ttf}
\newfontfamily\serifed[Path=fonts/, BoldFont=xits-bold.otf, ItalicFont=xits-italic.otf]{xits-regular.otf}
\newfontface\codefont[Path=fonts/, ItalicFont=mega80-Reverse.ttf]{mega80-Regular.ttf}
\newfontface\codefontwide[Path=fonts/]{mega40-Regular.ttf}
\newfontface\symbolfont[Path=fonts/]{MEGA65GraphicSymbols.otf}


% Set margins for inner and outer pages in A5 book format
\ifdefined\printmanual
\usepackage[a5paper,nomarginpar,includemp,bottom=2cm,top=1cm,inner=1.8cm,outer=0.8cm, footskip = 1cm]{geometry}
\else
\usepackage[a5paper,nomarginpar,includemp,bottom=2cm,top=1cm,inner=1.0cm,outer=1.0cm, footskip = 1cm]{geometry}
\fi

% Some Computer Society conferences also require the compsoc mode option,
% but others use the standard conference format.
%
% If IEEEtran.cls has not been installed into the LaTeX system files,
% manually specify the path to it like:
% \documentclass[conference]{../sty/IEEEtran}

%% \input{setup}

% correct bad hyphenation here
\hyphenation{op-tical net-works semi-conduc-tor}

\makeindex[intoc]

\pagestyle{empty}

\begin{document}
\raggedbottom

% relax word wrapping with sloppy
\sloppy
% reduce overfull \hbox warnings
\hfuzz=5pt

% macro for changing the verbatim font
\makeatletter
\newcommand{\verbatimfont}[1]{\def\verbatim@font{#1}}%
\makeatother



\megabookstart{MEGA65 DOCUMENTATION STYLE GUIDE}{WORK IN PROGRESS}

%%%

\chapter{Style Overview}

Thank you for your interest in contributing to the MEGA65 documentation! This style guide describes the language and typesetting choices we have made for this project. Adhering to a style guide makes the documentation easier to read, and easier to write consistently. Some style choices are based on linguistic best practices, some are arbitrary, and some are somewhere in between. These are ours.

The MEGA65 documentation uses British English spelling. For more information, see \sg{british-english}.

Our base style guides are the \href{https://www.bbc.co.uk/newsstyleguide/}{BBC News Style Guide} (BBC) and the \href{https://docs.microsoft.com/en-us/style-guide/welcome/}{Microsoft Style Guide} (MSG), with \href{https://support.apple.com/en-my/guide/applestyleguide/welcome/web}{Apple Style Guide} (ASG) for additional terms not in MSG. The \href{https://www.chicagomanualofstyle.org/home.html}{Chicago Manual of Style, 17th ed.} (CMoS) is also a supplementary reference. Editorial decisions in this style guide override the base guides.

The target audience for the MEGA65 documentation consists of hobbyist personal computer users and programmers, of varying interests and levels of technical expertise. Be mindful of the level of expertise expected for each category of documentation.

The voice of the documentation is casual, warm, polite, and encouraging. \emph{MEGA65 enthusiasts are international.} Avoid colloquialisms or examples that depend on cultural context. Be mindful of English words with many uses. Use concise and consistent language. Avoid jargon: define special terms, and avoid unnecessary synonyms. Keep sentences and paragraphs short. Use tables, lists, and diagrams to organise concepts visually.

If you have any questions, please ask in the {\tt \#documentation} channel on the MEGA65 Discord, or file an issue in the {\tt mega65-user-guide} Github repo.

\begin{itemize}
\item \url{https://mega65.org/chat}
\item \url{https://github.com/MEGA65/mega65-user-guide/issues}
\end{itemize}

%%%

\chapter{LaTeX Typesetting}

The MEGA65 documentation is implemented using the LaTeX typesetting system. This chapter provides guidance on how LaTeX is used in this project.

For an introduction to LaTeX in general, see \href{https://www.overleaf.com/learn/latex/Learn_LaTeX_in_30_minutes}{Overleaf.com: Learn LaTeX in 30 minutes}. Google is your friend, especially the \href{https://tex.stackexchange.com/}{TeX StackExchange}.

\section{Editing LaTeX}

\textbf{Use Visual Studio Code} with the \href{https://marketplace.visualstudio.com/items?itemName=James-Yu.latex-workshop}{LaTeX Workshop extension}.

\textbf{Write each paragraph on a single line,} and enable ``word wrap'' in your editor. This makes it easier to search for multi-word phrases using common text search tools. This requires that authors use a programming text editor capable of ``soft wrapping.'' In Visual Studio Code, open the \textbf{View} menu, then select \textbf{Word Wrap}.

\textbf{Perform test builds frequently.} Be aware that small changes can introduce build errors, especially unescaped special characters. Use the \texttt{sandbox.tex} file and the \texttt{make sandbox.pdf} command to troubleshoot sections of code. Be sure \texttt{make all} succeeds before creating a Github pull request.

\textbf{Look for the first error.} If the build output ends with lines such as:

\begin{verbatim}
Latexmk: Errors, so I did not complete making targets
...
make: *** [styleguide.pdf] Error 12
\end{verbatim}

There was a build error. Examine the build output, or the \texttt{.log} file for the build target, for lines that begin with an exclamation point (\texttt{!}). The word ``error'' may or may not appear near the error message. Look for the first error: subsequent error messages may be caused by the first.

Many ``warning'' messages can be ignored, especially ``undefined references'' and ``Overfull {\textbackslash}hbox'' or ``Underfull {\textbackslash}hbox.'' These messages may or may not indicate problems.

\begin{samepage}
\section{Special Characters}

The following characters must be escaped with a backslash to be rendered as literal characters, \emph{except} in source code listing environments that know to treat them literally:

\begin{center}
\begin{tabular}{|c|l|}
\hline
\textbf{Symbol} & \textbf{LaTeX} \\
\hline
\& & \verb|\&| \\
\% & \verb|\%| \\
\$ & \verb|\$| \\
\# & \verb|\#| \\
\_ & \verb|\_| \\
\{ & \verb|\{| \\
\} & \verb|\}| \\
\texttt{\textasciitilde} & \verb|\textasciitilde| \\
\textasciicircum & \verb|\textasciicircum| \\
\textbackslash & \verb|\textbackslash| \\
\hline
\end{tabular}
\end{center}
\end{samepage}

\section{Common Directives}

This is a summary of commonly used and custom LaTeX directives. See the related topics for more information on how to use them.

\begin{longtable}{|L{3cm}|p{5cm}|L{4cm}|}
\hline
\textbf{Element} & \textbf{LaTeX} & \textbf{Example} \\
\hline
\endfirsthead
\multicolumn{3}{l@{}}{\ldots continued}\\
\hline
\textbf{Element} & \textbf{LaTeX} & \textbf{Example} \\
\hline
\endhead
\multicolumn{3}{l@{}}{continued \ldots}\\
\endfoot
\hline
\endlastfoot

\makecell[l]{
    Sections \\
    \textbullet~\sg{headings}
} &
\makecell[l]{
    \texttt{{\textbackslash}chapter\{Chapter Title\}} \\
    \texttt{{\textbackslash}section\{Section Title\}} \\
    \texttt{{\textbackslash}subsection\{Subsection\}}
} &
\\
\hline

\makecell[l]{
    Boldface \\
    \textbullet~\sg{basic} \\
    \textbullet~\sg{user-interfaces}
} &
\texttt{The {\textbackslash}textbf\{View\} menu} &
The \textbf{View} menu \\
\hline

\makecell[l]{
    Emphasis \\
    \textbullet~\sg{emphasis} \\
    \textbullet~\sg{introducing-terms}
} &
\texttt{The {\textbackslash}emph\{job list\} is a binary format...} &
The \emph{job list} is a binary format... \\
\hline

\makecell[l]{
    Underline \\
    \textbullet~\sg{callouts}
} &
\texttt{{\textbackslash}underline\{NOTE\}: Be good.} &
\underline{NOTE}: Be good. \\
\hline

Bulleted list &
\makecell[l]{
\texttt{{\textbackslash}begin\{itemize\}} \\
\texttt{{\textbackslash}item One} \\
\texttt{{\textbackslash}item Two} \\
\texttt{{\textbackslash}end\{itemize\}}} &
\begin{itemize}
\item One
\item Two
\end{itemize} \\
\hline

Numbered list &
\makecell[l]{
\texttt{{\textbackslash}begin\{enumerate\}} \\
\texttt{{\textbackslash}item One} \\
\texttt{{\textbackslash}item Two} \\
\texttt{{\textbackslash}end\{enumerate\}}} &
\begin{enumerate}
\item One
\item Two
\end{enumerate} \\
\hline

\makecell[l]{
    Cross reference \\
    \textbullet~\sg{cross-references}
} &
\makecell[l]{
    \texttt{{\textbackslash}label\{...\}} \\
    \\
    \texttt{{\textbackslash}bookref\{...\}} \\
    \texttt{{\textbackslash}vref\{...\}} \\
    \texttt{{\textbackslash}pageref\{...\}}
} &
\makecell[l]{
    \bookref{cha:terms-and-topics} \\
    \vref{cha:terms-and-topics} \\
    \pageref{cha:terms-and-topics} \\
} \\
\hline

\makecell[l]{
    Web link \\
    \textbullet~\sg{links}
} &
\texttt{{\textbackslash}url\{https://mega65.org/\}} &
\url{https://mega65.org/} \\
\hline

\makecell[l]{
    Footnote \\
    \textbullet~\sg{footnotes}
} &
\texttt{Example{\textbackslash}footnote\{This is an example footnote.\}} &
Example\footnote{This is an example footnote.} \\
\hline

\makecell[l]{
    Index term \\
    \textbullet~\sg{index-terms}
} &
\makecell[l]{
    \texttt{{\textbackslash}index\{Term\}} \\
    \texttt{{\textbackslash}index\{Term!Child term\}}
} &
\\
\hline

\makecell[l]{
    Table \\
    \textbullet~\sg{tables}
} &
\multicolumn{2}{l@{}|}{
\makecell[l]{
    \texttt{{\textbackslash}begin\{center\}} \\
    \texttt{{\textbackslash}begin\{tabular\}\{|l|c|r|\}} \\
    \texttt{{\textbackslash}hline} \\
    \texttt{{\textbackslash}textbf\{...\} \& ... \& ... {\textbackslash}{\textbackslash}} \\
    \texttt{{\textbackslash}hline} \\
    \texttt{... \& ... \& ... {\textbackslash}{\textbackslash}} \\
    \texttt{{\textbackslash}hline} \\
    \texttt{... \& ... \& ... {\textbackslash}{\textbackslash}} \\
    \texttt{{\textbackslash}hline} \\
    \texttt{{\textbackslash}end\{tabular\}} \\
    \texttt{{\textbackslash}end\{center\}}
}}
\\
\hline

\makecell[l]{
    Images \\
    \textbullet~\sg{figures} \\
    \textbullet~\sg{screenshots}
} &
\multicolumn{2}{l@{}|}{
\makecell[l]{
    \texttt{{\textbackslash}begin\{center\}} \\
    \texttt{{\textbackslash}includegraphics[width=0.9{\textbackslash}linewidth]\{images/...\}} \\
    \texttt{{\textbackslash}end\{center\}}
}}
\\
\hline

\makecell[l]{
    Code blocks \\
    \textbullet~\sg{source-code}
} &
\multicolumn{2}{l@{}|}{
\makecell[l]{
    \texttt{{\textbackslash}begin\{basiccode\}} \\
    \texttt{10 PRINT "MEGA65 RULES!"} \\
    \texttt{20 GOTO 10} \\
    \texttt{{\textbackslash}end\{basiccode\}} \\
    \\
    \texttt{{\textbackslash}begin\{asmcode\}} \\
    \texttt{LDA \#\$52} \\
    \texttt{LDX \#\$24} \\
    \texttt{LDY \#\$00} \\
    \texttt{LDZ \#\$B3} \\
    \texttt{MAP} \\
    \texttt{EOM} \\
    \texttt{{\textbackslash}end\{asmcode\}} \\
    \\
    \texttt{{\textbackslash}begin\{lstlisting\}} \\
    \texttt{...} \\
    \texttt{{\textbackslash}end\{lstlisting\}}
}}
\\
\hline

\end{longtable}


\section{Keyboard keys}

The MEGA65 documentation includes special typesetting support for keyboard keys.

When referring to a keyboard key, use one of the following commands.

\begin{center}
\begin{tabular}{|l l|}
\multicolumn{2}{l@{}}{\textbf{Character, function, cursor keys}} \\
\hline
\texttt{{\textbackslash}megakey\{A\}} & \megakey{A} \\
\texttt{{\textbackslash}megakey\{F1\}} & \megakey{F1} \\
\texttt{{\textbackslash}megakey\{\${\textbackslash}leftarrow\$\}} & \megakey{$\leftarrow$} \\
\texttt{{\textbackslash}megakey\{\${\textbackslash}rightarrow\$\}} & \megakey{$\rightarrow$} \\
\texttt{{\textbackslash}megakey\{\${\textbackslash}uparrow\$\}} & \megakey{$\uparrow$} \\
\texttt{{\textbackslash}megakey\{\${\textbackslash}downarrow\$\}} & \megakey{$\downarrow$} \\
\hline
\end{tabular}
\end{center}

\begin{center}
\begin{tabular}{|l l|}
\multicolumn{2}{l@{}}{\textbf{Arrow character keys}} \\
\hline
\texttt{{\textbackslash}megakeywhite\{\${\textbackslash}leftarrow\$\}} & \megakeywhite{$\leftarrow$} \\
\texttt{{\textbackslash}megakeywhite\{\${\textbackslash}uparrow\$\}} & \megakeywhite{$\uparrow$} \\
\hline
\end{tabular}
\end{center}

\begin{center}
\begin{tabular}{|l l|}
\multicolumn{2}{l@{}}{\textbf{Special keys}} \\
\hline
\texttt{{\textbackslash}specialkey\{RUN{\textbackslash}{\textbackslash}STOP\}} & \specialkey{RUN\\STOP} \\
\texttt{{\textbackslash}specialkey\{ESC\}} & \specialkey{ESC} \\
\texttt{{\textbackslash}specialkey\{ALT\}} & \specialkey{ALT} \\
\texttt{{\textbackslash}specialkey\{CAPS{\textbackslash}{\textbackslash}LOCK\}} & \specialkey{CAPS\\LOCK} \\
\texttt{{\textbackslash}specialkey\{NO{\textbackslash}{\textbackslash}SCROLL\}} & \specialkey{NO\\SCROLL} \\
\texttt{{\textbackslash}specialkey\{HELP\}} & \specialkey{HELP} \\
\texttt{{\textbackslash}specialkey\{CLR{\textbackslash}{\textbackslash}HOME\}} & \specialkey{CLR\\HOME} \\
\texttt{{\textbackslash}specialkey\{INST{\textbackslash}{\textbackslash}DEL\}} & \specialkey{INST\\DEL} \\
\texttt{{\textbackslash}specialkey\{TAB\}} & \specialkey{TAB} \\
\texttt{{\textbackslash}specialkey\{CTRL\}} & \specialkey{CTRL} \\
\texttt{{\textbackslash}specialkey\{SHIFT{\textbackslash}{\textbackslash}LOCK\}} & \specialkey{SHIFT\\LOCK} \\
\texttt{{\textbackslash}specialkey\{RETURN\}} & \specialkey{RETURN} \\
\texttt{{\textbackslash}specialkey\{SHIFT\}} & \specialkey{SHIFT} \\
\hline
\end{tabular}
\end{center}

\begin{center}
\begin{tabular}{|l l|}
\multicolumn{2}{l@{}}{\textbf{Other keys}} \\
\hline
\texttt{{\textbackslash}widekey\{RESTORE\}} & \widekey{RESTORE} \\
\texttt{{\textbackslash}megasymbolkey} & \megasymbolkey \\
\hline
\end{tabular}
\end{center}

\section{PETSCII Glyphs}

The MEGA65 documentation uses a special font for rendering PETSCII graphics characters as vector art. This is used exclusively in the PETSCII and screen code reference tables. For a complete list, see the tables themselves: \texttt{appendix-petsciicodes.tex} and \texttt{appendix-screencodes.tex}.

\begin{quote}

\texttt{{\textbackslash}graphicsymbol\{a\}}
\texttt{{\textbackslash}graphicsymbol\{A\}}
\texttt{{\textbackslash}graphicsymbol\{j\}}
\texttt{{\textbackslash}graphicsymbol\{J\}}

\hrulefill

\graphicsymbol{a}
\graphicsymbol{A}
\graphicsymbol{j}
\graphicsymbol{J}

\end{quote}

\section{Screen Text}

You can typeset text in a PETSCII font to represent text as it would appear on the MEGA65 screen. 80-column and 40-column variants are available.

\begin{quote}
    \texttt{Type your command at the {\textbackslash}screentext\{READY.\} prompt.}

    \texttt{The message {\textbackslash}screentextwide\{38911 BASIC BYTES FREE\} indicates the available memory for your program.}

    \hrulefill

    Type your command at the \screentext{READY.} prompt.

    The message \screentextwide{38911 BASIC BYTES FREE} indicates the available memory for your program.

\end{quote}

The \texttt{screencode} environment uses this font to render a block of verbatim text, to represent on-screen messages. See \sg{source-code}.

\underline{NOTE}: There is a method to access PETSCII graphics characters in this environment using upper Unicode code points, but it is currently broken. (See \texttt{wrong.tex}.)


%%%%%%%%%%%%%%%%%%%%%%%%%%%%%%%%%%%%%%%%%%%%%%%%%%%%%%%%%%%

\chapter{Terms and Topics}
\label{cha:terms-and-topics}
% Please keep this organized alphabetically.

\begin{sgentry}{8-bit}{8-bit, 16-bit, 32-bit}
    A bit width, as an adjective. Always use digits and a hyphen: ``an 8-bit register''

    When referring to a bit width as a noun, describe it as an amount: ``The value is 8 bits.''
\end{sgentry}

\begin{sgentry}{abbreviations}{Abbreviations}
    Spell out the first use of an abbreviation or initialism, followed by the abbreviation in brackets. Use emphasis for the spelled-out phrase (see \sg{introducing-terms}), but not the abbreviation.

    \begin{quote}
        The VIC-IV can draw sprite-like graphical objects using the \emph{Raster Rewrite Buffer} (RRB). To use the RRB...
    \end{quote}

    Do not use dot punctuation within the abbreviation: ``RRB,'' not ``R.R.B.''
\end{sgentry}

\begin{sgentry}{adapter}{adapter}
    A component that adapts one interface to another. Not ``adaptor.''
\end{sgentry}

\begin{sgentry}{addresses}{Addresses}
    Prefer hexadecimal numbers for memory addresses. Hexadecimal numbers reflect the structure of a memory map, and can be used as literals in BASIC 65 and assembly language programs.

    \begin{quote}
        \$D021

        FFD.2FFF
    \end{quote}

    In the context of Commodore 64 BASIC, it is useful to represent a 16-bit memory address as a decimal value, because C64 BASIC lacks hexadecimal literals. In this case, omit the grouping comma: ``53281''

    See also \sg{numbers}, \sg{hexadecimal-numbers}, \sg{registers}.
\end{sgentry}

\begin{sgentry}{assembly-language}{assembly language}
    A programming language that represents machine code using mnemonics for human authoring.

    Assembly language is \emph{assembled} using a tool called an \emph{assembler} to produce a machine code program. An assembly language program listing consists of \emph{instructions} that represent machine code, and \emph{directives} that tell the assembler how to assemble the program. An instruction consists of a \emph{mnemonic} and an optional \emph{operand}, with syntax that indicates the \emph{addressing mode}.

    When referring to a machine code instruction in the abstract, use the assembly language mnemonic as spelled in the 45GS02 instruction set reference, with uppercase letters: ``the STQ instruction.''

    In assembly language code samples, use uppercase letters for assembly language mnemonics, and for hexadecimal number literals. Use mixed case for comments. See \sg{source-code}.

    Do not shorten the term ``assembly language'' as ``assembly.''

    See also \sg{machine-code}.
\end{sgentry}

\begin{sgentry}{attic-ram}{Attic RAM}
    Eight megabytes of RAM starting at address 800.0000, available in production MEGA65 and DevKit models (mainboard R3 and later). Attic RAM is not available on Nexys boards.

    Some documentation refers to ``Cellar RAM'' that could be available in a future expansion. This expansion is not yet available.

    Always capitalize ``Attic RAM.'' See also \sg{chip-ram}.
\end{sgentry}

\begin{sgentry}{basic}{BASIC}
    The BASIC programming language. When referring specifically to the dialect of BASIC in MEGA65 mode, use ``BASIC 65.''

    When referring to a BASIC keyword in the abstract, use uppercase letters and boldface text.

    \begin{quote}
        \texttt{the {\textbackslash}textbf\{PRINT\} command}

        \hrulefill

        the \textbf{PRINT} command
    \end{quote}

    When referring to an argument named in the BASIC reference prototype for the statement or function, use lowercase letters and bold text:

    \begin{quote}
        \texttt{the {\textbackslash}textbf\{speed\} argument}

        \hrulefill

        the \textbf{speed} argument
    \end{quote}

    When it is clear from context that a keyword is a command, function, or statement, use the keyword as a proper noun: ``Use \textbf{GOSUB} to call the subroutine.''
\end{sgentry}

\begin{sgentry}{basic-65}{BASIC 65}
    The dialect of the BASIC programming language available in MEGA65 mode. Not ``BASIC65.''
\end{sgentry}

\begin{sgentry}{basic-errors}{BASIC errors}
    When referring to a BASIC error type, spell its message using initial capital letters, followed by the word ``error.''

    \begin{quote}
        This program terminates with a Division By Zero error.
    \end{quote}

    Do not use all-caps or the leading question mark as it appears on screen (not ``?DIVISION BY ZERO ERROR''). If presenting the error message as it appears on screen is necessary for orientation, use a \texttt{screencode} block. See \sg{source-code}.
\end{sgentry}

\begin{sgentry}{binary}{binary}
    The base-2 numbering system. Avoid ``bin'' except as an abbreviation in a table heading.
\end{sgentry}

\begin{sgentry}{binary-numbers}{Binary numbers}
    To represent a binary number, use 0's and 1's, and a leading percent symbol (\%). Always use the percent symbol to avoid confusion with decimal or hexadecimal numbers.

    If needed for clarity, use a period symbol (.) to separate groups of four digits, counting from the right: \%0010.1101 Use leading zeroes to indicate the value width, if appropriate. Avoid representing bitfields larger than eight bits in body text.

    \begin{quote}
        \%00101101

        \%0010.1101
    \end{quote}
\end{sgentry}

\begin{sgentry}{bitstream}{bitstream}
    Data that describes the intended behavior of an FPGA. The MEGA65 loads a bitstream into the FPGA when it is first switched on. A developer can upload a new bitstream directly to the FPGA from a PC using a JTAG adapter.

    A bitstream is not the same as a ``core.'' See \sg{core}.

    See also \sg{jtag-adapter}.
\end{sgentry}

\begin{sgentry}{brackets}{brackets}
    The preferred term for parentheses punctuation. The three common bracket types are ``brackets'' (\texttt{(} and \texttt{)}), ``square brackets'' (\texttt{[} and \texttt{]}), and ``curly brackets'' (\texttt{\{} and \texttt{\}}).
\end{sgentry}

\begin{sgentry}{british-english}{British English}
    The MEGA65 documentation uses \textbf{British English}. Some notable differences between British English and other dialects include:

    \begin{itemize}
        \item \textbf{-our} suffixes: \textbf{colour}, \textbf{humour}, \textbf{flavour}
        \item \textbf{-ise} suffixes: \textbf{initialise}, \textbf{summarise}, \textbf{organise}, \textbf{specialise}
        \item \textbf{-yse} suffixes: \textbf{analyse}
        \item \textbf{-ogue} suffixes: \textbf{dialogue}
        \item \textbf{Doubled el} when conjugating verbs: \textbf{cancelled}, \textbf{labelling}
        \item \textbf{different to}, not ``different from'' or ``different than''
        \item Colour names: \textbf{grey}
    \end{itemize}

    Commodore used American English spellings in API names (e.g. \textbf{COLOR}). Use British spellings when referring to concepts, and use API typesetting and spellings when referring to specific API names.

    Be aware that some LaTeX commands use American English spelling.
\end{sgentry}

\begin{sgentry}{callouts}{Callouts}
    A callout is a short paragraph, typeset to attract the reader's attention, typically with a tag, such as ``NOTE.''

    To typeset a callout, use a paragraph. Underline the callout tag and spell it with all capital letters, then follow it with a colon and the callout text. Today, the only callout tag used in the MEGA65 documentation is ``NOTE.''

    \begin{quote}
        \texttt{{\textbackslash}underline\{NOTE\}: If the denominator is zero, the computer may crash.}

        \hrulefill

        \underline{NOTE}: If the denominator is zero, the computer may crash.
    \end{quote}
\end{sgentry}

\begin{sgentry}{chip-ram}{Chip RAM}
    384KB of RAM starting at address 0.0000, available in all MEGA65 models. Future models may expand this memory region up to 1MB.

    Always capitalize ``Chip RAM.'' See also \sg{attic-ram}.
\end{sgentry}

\begin{sgentry}{commas}{Commas}
    Use a comma before the conjunction in a list of three or more items, aka the Oxford comma or the serial comma. (\href{https://learn.microsoft.com/en-us/style-guide/punctuation/commas}{MSG: Commas}.)

    \begin{quote}
        The MEGA65's CPU supports 6502, 65C02, 65CE02, and 45GS02 instructions.
    \end{quote}
\end{sgentry}

\begin{sgentry}{commodore}{Commodore, Commodore 64, C64}
    ``Commodore'' refers to the company \href{https://en.wikipedia.org/wiki/Commodore_International}{Commodore International} (aka Commodore International, Ltd.). It may also refer to its subsidiary Commodore Business Machines (CBM).

    Various computer made by Commodore include the Commodore VIC-20, the Commodore 64, and the Commodore 128. These names are often abbreviated by omitting the word ``Commodore,'' and adding a leading ``C'' in the case of the C64, C128, and C65.

    It is safe to use the abbreviated name in a section after using the full name the first time, without explaining the abbreviation.
\end{sgentry}

\begin{sgentry}{computer}{computer}
    Refer to the MEGA65 as a ``computer.'' Not ``machine'' or ``device.''
\end{sgentry}

\begin{sgentry}{configuration-utility}{configuration utility}
    A built-in utility program that allows the user to change settings for the computer. To access the configuration utility, the user opens the utility menu by holding the Alt key while switching on the computer, then selects it from the menu.
\end{sgentry}

\begin{sgentry}{core}{core}
    A data set containing an FPGA bitstream, along with metadata used by the core selection menu.

    A user updates the firmware of the MEGA65 by downloading the latest MEGA65 core file, transferring it to the SD card, then opening the core selection menu and following prompts. A user can also install alternate cores that describe other computers. The core selection menu manages multiple \emph{core slots}. When the computer is switched on, the user can allow it to load the default core, or open the core selection menu to select an alternate core.

    The technical term for the data that gets uploaded to the FPGA is a ``bitstream.'' Prefer the term ``core'' when describing user actions involving the firmware and the core selection menu. Limit the use of ``bitstream'' to describing the technical details of the boot process, or describing how to test new FPGA firmware with a JTAG adapter.

    See \sg{bitstream}, \sg{jtag-adapter}.
\end{sgentry}

\begin{sgentry}{core-selection-menu}{core selection menu}
    A built-in utility program that allows the user to install or upgrade cores, or select a core for booting. To access the utility, the user holds the No Scroll key while switching on the computer.

    See \sg{core}.
\end{sgentry}

\begin{sgentry}{cross-references}{Cross references}
    Text can refer to a chapter, appendix, or section. LaTeX will generate text and a link suitable for printing, with the correct chapter and page number.

    The section to reference must have a label. By convention, a label for a chapter, appendix, or section begins with a prefix indicating its type.     The rest of the label should use lowercase letters and hyphens, and approximate the section heading.

    \begin{itemize}
        \item Chapter: \texttt{cha:...}
        \item Appendix: \texttt{appendix:...}
        \item Section: \texttt{sec:...}
    \end{itemize}

    The \texttt{{\textbackslash}label} directive sets the label. This does not add any text to the document. Place the label immediately beneath the section marker.

    \begin{quote}
        \begin{verbatim}
\chapter{Terms and Topics}
\label{cha:terms-and-topics}
        \end{verbatim}
    \end{quote}

    To create a cross reference to a labelled chapter, use one of these:

    \begin{center}
    \begin{tabular}{|l|l|p{6cm}|}
        \hline
        \textbf{LaTeX} & \textbf{Example} & \textbf{Description} \\
        \hline

        \texttt{{\textbackslash}bookref\{...\}} &
        \bookref{cha:terms-and-topics} &
        The chapter number. This has special support for the MEGA65 Compendium: if the chapter is in the current book, this refers to the chapter by number. If the chapter is not in the current book, it is assumed to be a chapter of the Compendium. The Compendium must be built before the current document to generate a chapter index (\texttt{make mega65-book.pdf}). \\
        \hline

        \texttt{{\textbackslash}vref\{...\}} &
        \vref{cha:terms-and-topics} &
        The chapter or appendix number, and the page number. Provide the word ``chapter'' or ``appendix,'' as appropriate. \\
        \hline

        \texttt{{\textbackslash}pageref\{...\}} &
        \pageref{cha:terms-and-topics} &
        The page number. Use this when referring to a section or other labelled element that isn't a chapter or appendix. \\
        \hline
    \end{tabular}
    \end{center}

    When \texttt{{\textbackslash}bookref} refers to a chapter outside of the current book, it looks like this: \bookref{cha:cpu}

    See also \sg{links}.
\end{sgentry}

\begin{sgentry}{decimal}{decimal}
    The base-10 numbering system. Avoid ``dec'' except as an abbreviation in a table heading.
\end{sgentry}

\begin{sgentry}{diagrams}{Diagrams}
    See \sg{figures}.
\end{sgentry}

\begin{sgentry}{direct-mode}{direct mode}
    The mode of BASIC commands entered directly at the \texttt{READY.} prompt, not run as part of a program. A command entered this way is run when the user presses the Return key.

    This is sometimes called ``immediate mode'' in vintage texts. Prefer ``direct mode'' to avoid confusion with the immediate addressing mode of the CPU.

    Historical note: \emph{The Commodore 64 User's Guide} uses ``immediate mode'' as the primary term, offering ``calculator mode'' as a synonym. Later books, including \emph{The Commodore 64 Programmer's Reference} and \emph{Programming the Commodore 64: The Definitive Guide} by Raeto West, use ``direct mode'' throughout, with ``immediate mode'' offered as a synonym. Cf. \href{https://www.c64-wiki.com/wiki/Direct_Mode}{C64 Wiki: Direct Mode}.
\end{sgentry}

\begin{sgentry}{disk}{disk}
    A platter of digital magnetic media, typically with its housing. ``The MEGA65 includes a built-in 3-1/2" disk drive. It accepts disks of type DD or HD.''

    Not ``disc,'' except when referring to ``compact disc'' optical media.
\end{sgentry}

\begin{sgentry}{eg}{e.g.}
    Introduces a brief example phrase at the end of a sentence. (An abbreviation for \emph{exempli gratia}, Latin for ``for the sake of example.'')

    Prefer to introduce examples formally, for clarify: ``For example...'' Always use English to introduce an example in a new sentence.

    Use of ``e.g.'' is allowed for informality and de-emphasis. Take care not to overuse it in a section.

    When using ``e.g.,'' follow it with a comma, then the example phrase. (\href{https://www.chicagomanualofstyle.org/qanda/data/faq/topics/Abbreviations/faq0047.html}{CMoS}.)

    \begin{quote}
        On return, it restores the previous KERNAL-managed memory map, e.g., the SYS map.

        Address numbering starts at 0, so regions of these common sizes tend to start with ``0'' digits on the right, and stop just before the start of the next region. For example, \$2000 -- \$2FFF represents a 4KB region.
    \end{quote}
\end{sgentry}

\begin{sgentry}{emphasis}{Emphasis}
    It is sometimes useful to emphasize a phrase or sentence. You can typeset this with the \texttt{{\textbackslash}emph\{...\}} directive.

    Emphasis is not especially visible in the MEGA65 typesetting. For important information, reinforce this with other structures. See \sg{callouts}, \sg{footnotes}.

    \begin{quote}
        \texttt{{\textbackslash}underline\{NOTE\}: {\textbackslash}emph\{Leave the computer switched on until this process is complete.\}}

        \hrulefill

        \underline{NOTE}: \emph{Leave the computer switched on until this process is complete.}
    \end{quote}
\end{sgentry}

\begin{sgentry}{ethernet}{Ethernet}
    A type of physical network data connection. Spell this with an initial capital letter, even when used as an adjective: ``Connect the Ethernet cable to the MEGA65's Ethernet port.'' (\href{https://support.apple.com/en-my/guide/applestyleguide/apsg076a7313/web}{ASG: Ethernet}.)
\end{sgentry}

\begin{sgentry}{figures}{Figures}
    In MEGA65 documentation, figures are centered, and do not have numbers or captions.

    Authors are encouraged to use LaTeX diagramming markup to produce figures. This describes the diagram as revision-controlled text, and doesn't depend on external tools. See \href{https://www.overleaf.com/learn/latex/Picture_environment}{Overleaf.com: Picture environment}. Admittedly, this is not everyone's idea of a good time.

    Another option is to use an open source vector art tool. Draw.io (\url{https://app.diagrams.net/}) is a free diagramming tool, available for download or as a browser-based web app, and saves an open format. Inkscape (\url{https://inkscape.org/}) can edit SVG files directly.

    When using an external tool, try to match the style of the rest of the document. Use black-on-white lines with no shadows. For label text, use the font \href{https://en.wikipedia.org/wiki/Liberation_fonts}{Liberation Sans}. For code-like text, use \href{https://fonts.google.com/specimen/Inconsolata}{Inconsolata}. Export as PDF, and commit both the PDF and the original source files (SVG or other) to the source repo.

    \begin{quote}
\begin{verbatim}
\begin{center}
\includegraphics[width=0.9\linewidth]{images/figure.pdf}
\end{center}
\end{verbatim}
    \end{quote}

    Avoid bitmap image formats (PNG, GIF, JPEG) for line art. Prefer line art to photographs when line art is an option.

    See also \sg{screenshots}.
\end{sgentry}

\begin{sgentry}{footnotes}{Footnotes}
    To add a footnote, use the \texttt{{\textbackslash}footnote\{...\}} directive at the location of the footnote marker. A typical footnote appears at the end of a sentence, after the sentence-ending punctuation. A footnote may also appear at the end of a word, after any word-ending punctuation (such as a comma).

    \begin{quote}
        \texttt{This is an example.{\textbackslash}footnote\{This is an example footnote.\}}

        \hrulefill

        This is an example.\footnote{This is an example footnote.}
    \end{quote}

    The footnote is rendered at the bottom of the page where the footnote marker appears. Be judicious about the use of space on the page.
\end{sgentry}

\begin{sgentry}{for-example}{For example}
    See \sg{eg}.
\end{sgentry}

\begin{sgentry}{fpga}{Field Programmable Gate Array (FPGA)}
    A general purpose programmable computational device. The MEGA65 hardware uses several FPGA devices, including the Xilinx Artix7 100T for its main chipset.

    The ``FPGA'' initialism is sufficient for most uses, without introducing the full term.
\end{sgentry}

\begin{sgentry}{freezer-menu}{Freezer menu}
    A menu that can be accessed while the MEGA65 is running. To access the Freezer menu, the user holds the Restore key for one second, then releases it. The user opens the Freezer menu to access multiple built-in features, such as mounting disk images in the virtual drive or adjusting audio volume settings.

    It is called the Freezer menu because the running state of the computer is frozen and stored. The Freezer menu can store multiple freeze states and allow the user to switch between them.

    Can be abbreviated as ``the Freezer,'' with an initial capital.

    See also \sg{monitor}, \sg{sprited}.
\end{sgentry}

\begin{sgentry}{go64-mode}{GO64 mode}
    The Commodore 64 running mode of the MEGA65 core and ROM, activated by a program or by the direct mode command \textbf{GO64}. Not ``C64 mode.''
\end{sgentry}

\begin{sgentry}{headings}{Headings}
    Use title-style capitalisation in titles and headings. This borrows from vintage texts, and overrides modern style guides that recommend sentence casing for headings. (\href{https://learn.microsoft.com/en-us/style-guide/capitalization}{MSG: Capitalization}.)

    Our LaTeX styles for chapter and section headings convert text to all capital letters automatically. Subsection headings will render as typed.

    \begin{quote}
        \texttt{{\textbackslash}section\{Using 28-bit Addresses in Machine Code\}}

        \hrulefill

        {\large USING 28-BIT ADDRESSES IN MACHINE CODE}
    \end{quote}
\end{sgentry}

\begin{sgentry}{hexadecimal}{hexadecimal}
    The base-16 numbering system. Avoid ``hex'' except as an abbreviation in a table heading.
\end{sgentry}

\begin{sgentry}{hexadecimal-numbers}{Hexadecimal numbers}
    To represent a hexadecimal number in the range \$0000 -- \$FFFF, use uppercase letters and a leading dollar symbol (\$). Use leading zeroes to indicate the value width, if appropriate: \$C for a 4-bit value, \$0C for an 8-bit value, \$000C for a 16-bit value. Omit the leading \$ if it is clear from context that the value is in hexadecimal format, such as in a table of register addresses.

    To represent a hexadecimal number larger than \$FFFF, use a period symbol (.) to separate groups of four digits, counting from the right: 1.3000, FFD.2FFF. The dot implies hexadecimal, so the dollar symbol can be omitted in most cases.

    \begin{quote}
        \$0000 -- \$FFFF; \$D021; \$0C; FFD.2FFF
    \end{quote}

    See also \sg{addresses}, \sg{registers}.
\end{sgentry}

\begin{sgentry}{high-byte}{high byte}
    The most significant byte of a 16-bit value. Spelled with a space (not a hyphen). See also \sg{low-byte}.

    Some MEGA65 registers and record fields have values larger than 16 bits, typically to store addresses. In these cases, ``high byte'' may refer to the second least significant byte, followed by ``bank byte'' for the third and ``megabyte byte'' for the fourth.
\end{sgentry}

\begin{sgentry}{ie}{i.e.}
    Introduces a brief paraphrase at the end of a sentence. (An abbreviation of \emph{id est}, Latin for ``that is.'')

    Prefer rewriting a point to make it clear without restating it, if possible. Always use English to introduce a paraphrase in a new sentence.

    Use of ``i.e.'' is allowed for informality and de-emphasis. Take care not to overuse it in a section.

    When using ``i.e.,'' follow it with a comma, then the example phrase. (\href{https://www.chicagomanualofstyle.org/qanda/data/faq/topics/Abbreviations/faq0047.html}{CMoS}.)

    \begin{quote}
        If the result of the operation is negative, i.e., it has its most significant bit set, the N flag will also be set.
    \end{quote}
\end{sgentry}

\begin{sgentry}{index-terms}{Index terms}
    To add a term reference to the book index, use the \texttt{{\textbackslash}index\{...\}} directive at the location of the reference.

    \begin{quote}
        \texttt{{\textbackslash}index\{Term\}}

        \hrulefill

        \texttt{{\textbackslash}index\{Term!Child term\}}
    \end{quote}

    This does not add text at the reference location. It adds an entry to the book's index.

    Be sure to use child terms consistently. Refer to the existing index to understand the indexing conventions already in use.
\end{sgentry}

\begin{sgentry}{input-output}{input/output}
    A bidirectional data port, or a subsystem of a computer for peripheral access. Abbreviated ``I/O'' (with the slash). Not ``IO.''
\end{sgentry}

\begin{sgentry}{internet}{Internet}
    The global data network. Spell this with an initial capital letter, even when used as an adjective: ``You can connect the MEGA65 to the Internet.'' ``Download the file using an Internet connection.'' (\href{https://support.apple.com/en-my/guide/applestyleguide/apsg346ef241/web}{ASG: Internet}.)

    Prefer ``network'' or ``local area network (LAN)'' when referring to a network other than the Internet.
\end{sgentry}

\begin{sgentry}{introducing-terms}{Introducing terms}
    When introducing a new special term, emphasize its first use and define it in the same sentence.

    \begin{quote}
        \texttt{The {\textbackslash}emph\{job list\} is a binary format with values packed into bytes.}

        \hrulefill

        The \emph{job list} is a binary format with values packed into bytes.
    \end{quote}

    See also \sg{abbreviations}.
\end{sgentry}

\begin{sgentry}{jtag-adapter}{JTAG adapter}
    A device that connects to pins on the mainboard, and has a mini-USB connector for running a USB cable to a PC. This connection can be used to upload a bitstream from a PC to the FPGA, and also provides a USB serial connection to the Hypervisor debugging console.

    It is also possible to connect a UART-USB serial adapter to these pins to use the serial connection without the FPGA access. Only a JTAG adapter can upload a bitstream.

    See also \sg{bitstream}, \sg{mainboard}, \sg{matrix-mode}.
\end{sgentry}

\begin{sgentry}{kernal}{KERNAL}
    The Commodore operating system kernel. ``KERNAL'' is a proper name and is always spelled in all capital letters. Not ``Kernal'' or ``kernal.''

    The word ``kernel'' refers to the concept in computer architecture. Use ``kernel'' to refer to this concept generically, and ``KERNAL'' to refer to the MEGA65 (or other Commodore) kernel specifically.
\end{sgentry}

\begin{sgentry}{kilobyte}{kilobyte}
    1,024 bytes. The unit abbreviation is KB: 64KB. See also \sg{units}.
\end{sgentry}

\begin{sgentry}{like}{like}
    Most uses of the word ``like'' are too casual for technical documentation, and its multiple definitions make it unclear. Prefer more formal synonyms.

    When referring to the reader's preferences, use ``wish'' or ``prefer:'' ``You may prefer to disable the CRT emulation setting if the display is too dark.'' Avoid the verb ``want'' when referring to the reader.

    When introducing a comparison or an example, use ``similar to'' or ``such as.'' ``The MEGA65, similar to the Commodore 128, has multiple operating modes.''

    The word ``like'' can sometimes be useful in a hyphenated phrase: ``a Commodore-like interface.'' It is sometimes useful following a verb: ``The directory listing is structured like a BASIC program in memory.'' These are also considered casual uses, but they are sometimes justified for their utility.
\end{sgentry}

\begin{sgentry}{links}{Links}
    To create a link to a web page, with the URL as the link text:

    \begin{quote}
        \texttt{{\textbackslash}url\{https://mega65.org/\}}

        \hrulefill

        \url{https://mega65.org/}
    \end{quote}

    If the URL is long or difficult to type, consider requesting a \texttt{mega65.org} shortened URL, e.g. \texttt{https://mega65.org/docs}.

    While it is possible to create a link with alternate link text, this will appear without the link in printed form. Only use such a link if the document will not be printed, or if a reader of the printed text does not need the address. Consider putting the URL in brackets after the title, or in a footnote.

    \begin{quote}
        \texttt{{\textbackslash}href\{https://files.mega65.org/\}\{Filehost\}}

        \hrulefill

        \href{https://files.mega65.org/}{Filehost}
    \end{quote}

    See also \sg{cross-references}, \sg{footnotes}.
\end{sgentry}

\begin{sgentry}{little-endian}{little-endian}
    A conventional encoding for large values as a sequence of bytes ordered from least significant byte to most significant byte. The alternative is big-endian. The MEGA65 is a little-endian architecture.

    The term is an adjective, and is spelled lowercase with a hyphen. (Cf. \href{https://en.wikipedia.org/wiki/Endianness}{Wikipedia: Endianness}.)
\end{sgentry}

\begin{sgentry}{low-byte}{low byte}
    The least significant byte of a multi-byte value. Spelled with a space (not a hyphen). See also \sg{high-byte}.
\end{sgentry}

\begin{sgentry}{machine}{machine}
    Avoid referring to the MEGA65 as a ``machine.'' Prefer \sg{computer}. (This is not inconsistent with \sg{machine-code}, a formal technical term.)
\end{sgentry}

\begin{sgentry}{machine-code}{machine code}
    Instructions for a CPU, encoded as a sequence of bytes in a computer's memory.

    An assembly language program assembles to machine code. A program in a compiled language, such as C, typically compiles to machine code. Each instruction has an \emph{opcode} that represents the instruction and addressing mode. Depending on the instruction, it may have an \emph{operand}, a value argument.

    The term ``machine language'' and its abbreviation ``ML'' are historical synonyms for machine code, as in ``ML program.'' Prefer ``machine code'' for clarity.

    See also \sg{assembly-language}, \sg{monitor}.
\end{sgentry}

\begin{sgentry}{mainboard}{mainboard}
    The circuit board inside the MEGA65 on which the FPGA resides.

    There are several revisions of the MEGA65 mainboard hardware. As of the year 2024, the only revisions relevant to the production models of MEGA65 are ``R3,'' which was distributed in the year 2022, and ``R6,'' which has been the latest production model since 2024. Technically, the MEGA65 DevKit models used the R3 board and the 2022 production models used a board known as ``R3A,'' but this distinction is irrelevant for all purposes, so the documentation refers to both revisions as ``R3.''

    Other revisions of the mainboard exist (R1, R2, R4, R5), but were never distributed.

    Not ``main board,'' not ``motherboard.''
\end{sgentry}

\begin{sgentry}{map-register}{MAP register}
    The 4510 memory mapping register. ``MAP'' is always all uppercase.

    The register has two components: ``MAPH'' and ``MAPL.''

    ``MAP'' is also the assembly language instruction that sets the MAP register.
\end{sgentry}

\begin{sgentry}{matrix-mode}{Matrix Mode}
    A debugging mode of the Hypervisor that displays a translucent console over the main display. This is toggled by pressing Mega + Tab.

    Matrix Mode provides on-device access to the Hypervisor debugging console that can also be accessed with a serial connection to pins on the mainboard, such as via a JTAG adapter. This is sometimes referred to as the ``Matrix Mode debugger'' even when not accessed via Matrix Mode.
\end{sgentry}

\begin{sgentry}{mega65}{MEGA65}
    The greatest home computer ever made. Not ``MEGA 65,'' not ``Mega65.''
\end{sgentry}

\begin{sgentry}{megabyte}{megabyte}
    1,048,576 bytes, or 1,024 kilobytes. The unit abbreviation is MB: 8MB. See also \sg{units}.
\end{sgentry}

\begin{sgentry}{monitor}{monitor}
    A program for inspecting memory, practicing machine code, and troubleshooting programs. The MEGA65 has three built-in tools that could be described as a ``monitor:''

    \begin{enumerate}
        \item A monitor provided by the KERNAL, invoked by the \textbf{MONITOR} command or by the KERNAL's \textbf{BRK} handler
        \item A monitor utility in the Freezer
        \item A monitor in the debugging console, accessible via a serial connection or ``Matrix Mode''
    \end{enumerate}

    The KERNAL's monitor is the most featureful and most likely to be referenced in documentation. Consider this ``the monitor,'' and establish context when referring to the others.

    The term ``monitor'' is also sometimes used to refer to the display peripheral. Prefer the noun ``display'' for this case, which is more common in modern style. When referring to a vintage cathode ray tube display, consider ``CRT display.''

    Historically, the term ``machine language monitor'' or ``ML monitor'' fully describes this feature, and disambiguates other uses of the term ``monitor.'' If necessary, introduce the concept as the ``machine language monitor,'' then use ``the monitor'' thereafter.
\end{sgentry}

\begin{sgentry}{nibble}{nibble}
    A four-bit value, or the upper or lower four bits of a byte.

    Not ``nybble,'' ``nybl,'' ``half-byte.'' Do not use the term ``nibble'' to refer to four discontinuous bits, or four continuous bits that are not the top or bottom four bits of a byte. While a 4-bit register contains a 4-bit value, it would be confusing to refer to it as a nibble.
\end{sgentry}

\begin{sgentry}{numbers}{Numbers}
    When a number represents a count, use the general rule to spell out numbers less than ten: ``Hold the key for one second.'' ``The computer will beep three times.'' (\href{https://learn.microsoft.com/en-us/style-guide/numbers}{MSG: Numbers}.)

    When a number represents an amount, use the decimal value followed by units. See \sg{units}.

    For decimal values, use commas (\texttt{,}) to separate decimal digits to the left of the decimal point in groups of three, counting from the right: ``1,024''

    Prefer hexadecimal numbers for memory addresses. Use hexadecimal numbers for memory values if appropriate to the application. See \sg{hexadecimal-numbers}.

    It is occasionally useful to present binary numbers in text. See \sg{binary-numbers}.
\end{sgentry}

\begin{sgentry}{parentheses}{parentheses}
    See \sg{brackets}.
\end{sgentry}

\begin{sgentry}{program}{program}
    As a noun: a computer program. As a verb: to write a computer program.

    Not ``programme,'' despite \sg{british-english}.
\end{sgentry}

\begin{sgentry}{programmer}{programmer}
    A person who programs a computer.

    The term ``developer'' or ``software developer'' is acceptable, but implies a professional programmer (as a synonym for ``software engineer''). Avoid ``coder,'' as too casual. Most people who program the MEGA65 would enjoy being called a ``programmer.''

    The term ``programmer'' may also refer to an FPGA firmware flashing device.
\end{sgentry}

\begin{sgentry}{quotes}{Quotes}
    Use English rules for quote-related punctuation.

    When a quoted phrase is followed by sentence-ending or phrase-ending punctuation, put the punctuation inside closing quote marks, ``like this.'' Do not use quote marks for typeable string values; use other typesetting tools, such as typewriter text or code blocks, to avoid ambiguity: \texttt{like this}

    To render double-quotes in paragraph text, use two backticks (\verb|``|) for the opening quote and two apostrophes (\verb|''|) for the closing quote.

    \begin{quote}
        \begin{verbatim}
``The time has come,'' the walrus said,
``to talk of many things.''
        \end{verbatim}

        \hrulefill

        ``The time has come,'' the walrus said, ``to talk of many things.''
    \end{quote}

    To render double-quotes in source code listings, use the ASCII double-quote (\verb|"|) character.
\end{sgentry}

\begin{sgentry}{r3}{R3, R6}
    See \sg{mainboard}.
\end{sgentry}

\begin{sgentry}{ready}{READY prompt}
    The prompt displayed by BASIC to indicate that it is awaiting a command in direct mode.

    The screen editor uses the \screentext{READY.} prompt to indicate it is in BASIC mode. When it is in ``Edit mode,'' it uses the \screentext{OK.} prompt.

    As a flourish, style it as screen text and include the period, like so:

    \begin{quote}
        \begin{verbatim}
the \screentext{READY.} prompt
        \end{verbatim}

        \hrulefill

        the \screentext{READY.} prompt
    \end{quote}

    See also \sg{direct-mode}.
\end{sgentry}

\begin{sgentry}{registers}{Registers}
    A MEGA65 hardware register has a label, a byte address, and an optional bit range. When referring to a register for the first time in a section, provide all three, omitting the bit range for 8-bit registers. On subsequent uses, you can use just the label; include the rest if it would help the reader.

    \begin{quote}
        To adjust the border color, set BORDERCOL \$D020 with the number of the system palette entry. You can read BORDERCOL to determine the current border color.
    \end{quote}

    Indicate the bit range by following the address with a dot then the range. Bits are numbered from 0 to 7, from least significant to most significant. If the register is 1-bit, only provide the bit address. If the register is multi-bit (but not 8-bit), specify the range from most significant to least significant, inclusive, with an N-dash (\texttt{--}).

    \begin{quote}
        The VFAST \$D054.6 register is set when the CPU is in 40 MHz mode.

        To change the character set address, set CB \$D018.3--1 to the address divided by 1,024 (1KB).
    \end{quote}

    See also \sg{addresses}.
\end{sgentry}

\begin{sgentry}{screenshots}{Screenshots}
    MEGA65 documentation can include images of the MEGA65 screen, or of PC software, to orient the user or illustrate a visual effect.

    When depicting on-screen text, such as a command and its printed output, prefer the \texttt{screencode} environment. See \sg{source-code}.

    When depicting a full MEGA65 text screen that does not involve bitmap graphics or sprites, use the \texttt{m65} tool and a JTAG adapter to capture the image. This tool reproduces the text display in a way that produces a crisp image, without interference from video capture hardware.

    \begin{quote}
        \texttt{m65 -S}
    \end{quote}

    When depicting a MEGA65 screen not supported by the \texttt{m65} tool, you can use the Xemu emulator and take a PC screenshot, or you can use video capture hardware with the MEGA65. When using Xemu, be sure to crop the image to remove emulator UI, such as the window title bar. When using real hardware, be sure to disable CRT emulation to get the brightest, sharpest image.

    You can also include screenshots of PC software. Take care to crop the screenshot to just the relevant portion of the image. PC screenshots tend to appear small in print.

    Store the screenshot using the PNG file format.

    \begin{quote}
\begin{verbatim}
\begin{center}
\includegraphics[width=0.9\linewidth]{images/screen1.png}
\end{center}
\end{verbatim}
    \end{quote}

    See also \sg{figures}.
\end{sgentry}

\begin{sgentry}{sd-card}{SD card / microSD card}
    A non-volatile memory storage card, of the types standardized by the SD Association. (Cf. \href{https://en.wikipedia.org/wiki/SD_card}{Wikipedia}.)

    With the MEGA65, ``SD card'' may refer to a full-size SD card that fits in the ``internal'' card slot, or the microSD card that fits in the ``external'' card slot.
\end{sgentry}

\begin{sgentry}{sd-card-utility}{SD card utility}
    A built-in utility program that allows the user to prepare a new SD card for use. To access the SD card utility, the user opens the utility menu by holding the Alt key while switching on the computer, then selects it from the menu.
\end{sgentry}

\begin{sgentry}{should}{should}
    Avoid most uses of ``should.''

    When referring to an action that the reader will take, use the imperative voice (``Switch on the computer''), or suggest the option (``You can switch on the computer''). Make clear the reasons why the reader would want to perform the action.

    When referring to a behavior of the computer, use a definite present tense statement: ``The power light turns green.'' When referring to a conditional behavior, use hypothetical future tense: ``When you switch on the computer, the power light will turn green.''
\end{sgentry}

\begin{sgentry}{sid-chip}{SID chip}
    The Commodore sound synthesizer device, of which the MEGA65 has four. ``SID'' is always all uppercase.

    Use ``SID chip'' or ``SID'' to refer to the sound device and its registers.

    While ``SID'' is an initialism for ``Sound Interface Device,'' it is not necessary to spell this out.
\end{sgentry}

\begin{sgentry}{source-code}{Source code}
    Computer source code is typeset as a block element, using tools specific to the programming language.

    LaTeX supports typesetting source code listings, either listed directly in the \texttt{.tex} file, or included from another file. Do not escape LaTeX special characters inside listing environments: LaTeX knows to do that automatically.

    To typeset a block of BASIC code, use the \texttt{basiccode} environment. This mode typesets BASIC 65 keywords. Use uppercase letters, and avoid upper PETSCII characters. (We currently do not use a petcat-style convention for PETSCII control character literals.)

    \begin{quote}
        \texttt{{\textbackslash}begin\{basiccode\} \\
        10 PRINT "MEGA65 RULES!" \\
        20 GOTO 10 \\
        {\textbackslash}end\{basiccode\}}

        \hrulefill

\begin{basiccode}
10 PRINT "MEGA65 RULES!"
20 GOTO 10
\end{basiccode}
    \end{quote}

    When demonstrating BASIC code that contains PETSCII control code literals, or when including both BASIC commands and responses together, use the \texttt{screencode} environment. \emph{In the BASIC reference only,} use \texttt{screencode} consistently throughout the reference.

    \begin{quote}
        \texttt{{\textbackslash}begin\{screencode\} \\
        READY. \\
        PRINT "HELLO" \\
        HELLO \\
        \\
        READY. \\
        {\textbackslash}end\{screencode\}}

        \hrulefill

\begin{screencode}
READY.
PRINT "HELLO"
HELLO

READY.
\end{screencode}
    \end{quote}

    To typeset a block of assembly language code, use \texttt{asmcode}. This mode typesets 45GS02 instructions and Acme assembler directives. Use uppercase letters for opcodes, and mixed case ASCII as it would appear in a source file otherwise.

    \begin{quote}
        \texttt{{\textbackslash}begin\{asmcode\} \\
        LDA \#\$52   ; MAPLO = select \$2 offset \$45200 \\
        LDX \#\$24 \\
        LDY \#\$00   ; MAPHI = select \$B offset \$30000 \\
        LDZ \#\$B3 \\
        MAP \\
        EOM \\
        {\textbackslash}end\{asmcode\}}

        \hrulefill

\begin{asmcode}
LDA #$52   ; MAPLO = select $2 offset $45200
LDX #$24
LDY #$00   ; MAPHI = select $B offset $30000
LDZ #$B3
MAP
EOM
\end{asmcode}
    \end{quote}

    To include a source code listing from another file, use the \texttt{{\textbackslash}basicinput} directive for BASIC, \texttt{{\textbackslash}asminput} for assembly language, or \texttt{{\textbackslash}lstinputlisting} for all others. For more information, see \href{https://mirror.ox.ac.uk/sites/ctan.org/macros/latex/contrib/listings/listings.pdf}{the Listings package documentation}.

    \begin{quote}
        \texttt{{\textbackslash}basicinput\{examples/ex1.bas\}}

        \texttt{{\textbackslash}asminput\{examples/mapdemo.s\}}

        \texttt{{\textbackslash}lstinputlisting[language=C]\{examples/bounce.c\}}
    \end{quote}

    Prefer code blocks to code inline with paragraph text. You can usually use code name typesetting inline, then follow the paragraph with a code sample. See \sg{assembly-language}, \sg{basic}, \sg{user-interfaces}.

    When inline code text is the best choice, typeset it with typewriter text:

    \begin{quote}
        \texttt{To set the border colour register: {\textbackslash}texttt\{POKE {\textbackslash}\$D020,6\} This has the same effect as the {\textbackslash}textbf\{BORDER\} command.}

        \hrulefill

        To set the border colour register: \texttt{POKE \$D020,6} This has the same effect as the \textbf{BORDER} command.
    \end{quote}

    There are ways to render inline text in a PETSCII-style font. Prefer typewriter text for legibility. If a situation calls for it:

    \begin{quote}
        \texttt{Type the command at the {\textbackslash}screentext\{READY.\} prompt.}

        \hrulefill

        Type the command at the \screentext{READY.} prompt.
    \end{quote}
\end{sgentry}

\begin{sgentry}{sprited}{SpritEd}
    A utility for editing sprite images, accessible from the Freezer menu. ``SpritEd'' is a proper noun, and always spelled with capital letters as shown.

    See also \sg{freezer-menu}.
\end{sgentry}

\begin{sgentry}{switch-on}{switch on / switch off}
    Enable or disable the computer by operating its power switch.

    \begin{quote}
        It is now safe to switch off the computer.
    \end{quote}

    Not ``turn on'' or ``power on.'' See also \sg{computer}.
\end{sgentry}

\begin{sgentry}{tables}{Tables}
    Use a table to represent an ordered collection of values with more than one property.

    In general, tables in MEGA65 documentation are typeset centered, with border lines surrounding and between every cell, including around the entire table and heading cells. Heading cells are boldface. Tables do not have numbers or captions.

    A table intended to appear unbroken on a single page can use the \texttt{tabular} environment.

    \begin{quote}
\begin{verbatim}
\begin{center}
\begin{tabular}{|l|l|}
\hline
{\bf Command} & {\bf Description} \\
\hline
{\tt put {\it filename}} & Send a file to the MEGA65. \\
\hline
{\tt get {\it filename}} & Retrieve a file from the MEGA65. \\
\hline
{\tt dir} & Display a directory listing of the SD card. \\
\hline
\end{tabular}
\end{center}
\end{verbatim}

\hrulefill

        \begin{center}
        \begin{tabular}{|l|l|}
        \hline
        {\bf Command} & {\bf Description} \\
        \hline
        {\tt put {\it filename}} & Send a file from the PC to the MEGA65. \\
        \hline
        {\tt get {\it filename}} & Retrieve a file from the MEGA65 to the PC. \\
        \hline
        {\tt dir} & Display a directory listing of the MEGA65 SD card. \\
        \hline
        \end{tabular}
        \end{center}
    \end{quote}

    If the table may span multiple pages, use \texttt{longtable}.

\begin{quote}
\begin{verbatim}
\begin{longtable}{|L{2cm}|p{4.5cm}|L{3.5cm}|}
\hline
\textbf{Element} & \textbf{LaTeX} & \textbf{Example} \\
\hline
\endfirsthead
\multicolumn{3}{l@{}}{\ldots continued}\\
\hline
\textbf{Element} & \textbf{LaTeX} & \textbf{Example} \\
\hline
\endhead
\multicolumn{3}{l@{}}{continued \ldots}\\
\endfoot
\hline
\endlastfoot

... & ... & ... & ... \\
\hline
... & ... & ... & ... \\
\hline
... & ... & ... & ... \\
\hline

\end{longtable}
\end{verbatim}
\end{quote}

    Use your discretion to modify this style and use other LaTeX table features in the best interests of the information being conveyed. Table cell alignment, column sizing, \texttt{{\textbackslash}multicolumn} cells, and \texttt{{\textbackslash}multicell} cells are just a few useful features.

    See \href{https://www.overleaf.com/learn/latex/Tables}{Overleaf.com: Tables}, \href{https://en.wikibooks.org/wiki/LaTeX/Tables}{WikiBooks: LaTeX/Tables}, \href{https://ctan.org/pkg/longtable?lang=en}{CTAN.org: longtable}.
\end{sgentry}

\begin{sgentry}{titles}{Titles}
    See \sg{headings}.
\end{sgentry}

\begin{sgentry}{trademarks}{Trademarks}
    In general, do not use a trademark symbol when referring to a company or product name in technical material. (\href{https://www.chicagomanualofstyle.org/qanda/data/faq/topics/RegisteredTrademarks.html}{CMoS: Registered Trademarks}.)

    As a courtesy, we acknowledge the trademarks for ``Commodore'' and ``Commodore 64'' in the front matter of our documentation, and they do not need to be acknowledged elsewhere. We also acknowledge trademarks in packaging and sales material.

    To typeset a trademark, registered trademark, or copyright symbol, always use LaTeX commands. Never use ``(c)''.

    \begin{quote}
        \texttt{Commodore\{{\textbackslash}texttrademark\}}

        \hrulefill

        Commodore{\texttrademark}
    \end{quote}
\end{sgentry}

\begin{sgentry}{units}{Units}
    When a number refers to a measure of something, follow the number with a space, then the unit of measure: 3.5 mm.

    When referring to a unit in the abstract, spell out the word: ``Sound data can be many megabytes in size.''

    Data sizes use vintage binary prefixes (\emph{not} modern IEC 60027-2 standard prefixes):

    \begin{center}
    \begin{tabular}{|c|c|l|}
        \hline
        \textbf{Unit} & \textbf{Abbreviation} & \textbf{Amount in bytes} \\
        \hline
        byte & B & 1 \\
        kilobyte & KB & 1,024 \\
        megabyte & MB & 1,048,576 \\
        gigabyte & GB & 1,073,741,824 \\
        \hline
    \end{tabular}
    \end{center}

    Typically, there is no need to provide additional clarification when data sizes with decimal units are used. This is only likely to matter when referring to the capacity of SD cards. The reader is expected to know the difference.

    In the specific case of data sizes, do not use a space between the number and the unit: ``64KB'' ``8MB''

    Frequencies use the SI unit abbreviations, with decimal multiples:

    \begin{center}
    \begin{tabular}{|c|c|l|}
        \hline
        \textbf{Unit} & \textbf{Abbreviation} & \textbf{Amount in hertz} \\
        \hline
        hertz & Hz & 1 \\
        kilohertz & kHz & 1,000 \\
        megahertz & MHz & 1,000,000 \\
        \hline
    \end{tabular}
    \end{center}

    See also \sg{numbers}, \sg{8-bit}.
\end{sgentry}

\begin{sgentry}{user-interfaces}{User interfaces}
    There are three major categories of user interface described by MEGA65 documentation: visual user interfaces (PC or MEGA65), a PC command line interface (Linux shell, Windows command prompt), and the BASIC 65 ``READY.'' prompt.

    When referring to a text label of a visual user interface, use bold text, and describe it as it appears in the interface. Refer to menus and buttons with the noun ``menu'' or ``button,'' and use only the label in other cases.

    \begin{quote}
        \texttt{the {\textbackslash}textbf\{File\} menu}

        \hrulefill

        the \textbf{File} menu
    \end{quote}

    When referring to a PC command line command or flag name, use typewriter text.

    \begin{quote}
        \texttt{the {\textbackslash}texttt\{mega65{\textbackslash}\_ftp\} command}

        \hrulefill

        the \texttt{mega65\_ftp} command
    \end{quote}

    Prefer verbatim code blocks to inline text when referring to full commands.

    \begin{quote}
        \texttt{{\textbackslash}begin\{verbatim\} \\
        mega65\_ftp -e -c 'put file.d81' -c 'exit' \\
        {\textbackslash}end\{verbatim\}}

        \hrulefill

\begin{verbatim}
mega65_ftp -e -c 'put file.d81' -c 'exit'
\end{verbatim}
    \end{quote}

    For information on typesetting BASIC keywords, see \sg{basic}. Prefer the \texttt{basiccode} environment for typesetting BASIC commands, and \texttt{screencode} for typesetting BASIC commands with expected replies. See \sg{source-code}.
\end{sgentry}

\begin{sgentry}{utility-menu}{utility menu}
    A menu that provides access to the configuration utility and the SD card utility. To access the menu, the user holds the Alt key while switching on the computer.

    See also \sg{configuration-utility} and \sg{sd-card-utility}.
\end{sgentry}

\begin{sgentry}{verb-tense}{Verb tense}
    Use standard English grammar rules for verb tense (past, present, future, etc.). Make sure the verb tense agrees with the subject: ``Computers are cool.'' ``The MEGA65 is cool.''

    When referring to the capabilities of vintage computers, \emph{use present tense.} Vintage computers are still in use by hobbyists today, and retain their technical properties.

    \begin{quote}
        The Commodore 64 has 64 kilobytes of memory.
    \end{quote}

    Not:

    \begin{quote}
        The Commodore 64 had 64 kilobytes of memory.
    \end{quote}

    Use past tense when referring to past events:

    \begin{quote}
        The Commodore 64 was released by Commodore International in 1982.
    \end{quote}
\end{sgentry}

\begin{sgentry}{vic-chip}{VIC chip}
    The MEGA65 video device, or the Commodore equivalent. ``VIC'' is always all uppercase.

    Use ``VIC chip'' or ``VIC'' to refer to the video device and its registers in general. When referring to a feature specific to a VIC personality, refer to the specific personality: ``VIC-II,'' ``VIC-III,'' ``VIC-IV.''

    While ``VIC'' is an initialism for ``Video Interface Chip,'' it is not necessary to spell this out.
\end{sgentry}

\begin{sgentry}{want}{want}
    As a matter of formality, avoid the verb ``want'' when referring to the reader. See \sg{like}.
\end{sgentry}

\begin{sgentry}{you}{you}
    Refer to the reader in the 2nd person singular: ``You can download the latest version from Filehost.''

    Only use the 1st person plural ``we'' to refer to the MEGA65 team, and do so sparingly: ``We recommend upgrading to the latest firmware.''

    To refer to an optional activity for the reader, use ``can,'' not ``could.''
\end{sgentry}

%%%

\input{common-footer}
