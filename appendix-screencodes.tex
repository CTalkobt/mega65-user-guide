\chapter{Screen Codes}

\section{Screen Codes}

\label{appendix:screencodes}

A text character is represented in screen memory by a screen code. There are
256 possible screen codes, each referring to an image in the current character
set.

A complete character set contains two groups of 256 images, one for the
uppercase mode and one for the lowercase mode, for a total of 512 images. Only
one mode can be displayed at a time. The built-in character sets use the first
128 characters of each group for normal characters and the next 128 for
reversed versions of the same characters.

In BASIC, the {\bf T@\&()} special array provides access to the characters on
the screen using column and row indexes. The values in this special array are
screen codes. The {\bf FONT} command changes between the built-in character
sets. The {\bf CHARDEF} command changes the image associated with a screen
code.

\underline{Note}: Screen codes are different to PETSCII codes. PETSCII codes
are used to store, transmit, and receive textual data, and control the way
strings are printed to the screen. When a PETSCII character is printed to the
screen, the corresponding screen code is written to screen memory. For a list
of PETSCII codes, see appendix \vref{appendix:asciicodes}.

The following table lists the screen codes. When a code produces a different
character based on the mode, the character is listed as ``uppercase /
lowercase.''

\index{Keyboard!Screen Codes}
\begin{adjustwidth}{}{-2cm}
\begin{multicols}{4}
\begin{description}[align=left,labelwidth=0.2cm]
    \item [0]   @
    \item [1]   A / a
    \item [2]   B / b
    \item [3]   C / c
    \item [4]   D / d
    \item [5]   E / e
    \item [6]   F / f
    \item [7]   G / g
    \item [8]   H / h
    \item [9]   I / i
    \item [10]  J / j
    \item [11]  K / k
    \item [12]  L / l
    \item [13]  M / m
    \item [14]  N / n
    \item [15]  O / o
    \item [16]  P / p
    \item [17]  Q / q
    \item [18]  R / r
    \item [19]  S / s
    \item [20]  T / t
    \item [21]  U / u
    \item [22]  V / v
    \item [23]  W / w
    \item [24]  X / x
    \item [25]  Y / y
    \item [26]  Z / z
    \item [27]  [
    \item [28]  \pounds
    \item [29]  ]
    \item [30]  $\uparrow$
    \item [31]  $\leftarrow$
    \item [32]  space
    \item [33]  !
    \item [34]  "
    \item [35]  \#
    \item [36]  \$
    \item [37]  \%
    \item [38]  \&
    \item [39]  '
    \item [40]  (
    \item [41]  )
    \item [42]  *
    \item [43]  +
    \item [44]  ,
    \item [45]  -
    \item [46]  .
    \item [47]  /
    \item [48]  0
    \item [49]  1
    \item [50]  2
    \item [51]  3
    \item [52]  4
    \item [53]  5
    \item [54]  6
    \item [55]  7
    \item [56]  8
    \item [57]  9
    \item [58]  :
    \item [59]  ;
    \item [60]  <
    \item [61]  =
    \item [62]  >
    \item [63]  ?
    \item [64]  \graphicsymbol{C}
    \item [65]  \graphicsymbol{A} / A
    \item [66]  \graphicsymbol{B} / B
    \item [67]  \graphicsymbol{C} / C
    \item [68]  \graphicsymbol{D} / D
    \item [69]  \graphicsymbol{E} / E
    \item [70]  \graphicsymbol{F} / F
    \item [71]  \graphicsymbol{G} / G
    \item [72]  \graphicsymbol{H} / H
    \item [73]  \graphicsymbol{I} / I
    \item [74]  \graphicsymbol{J} / J
    \item [75]  \graphicsymbol{K} / K
    \item [76]  \graphicsymbol{L} / L
    \item [77]  \graphicsymbol{M} / M
    \item [78]  \graphicsymbol{N} / N
    \item [79]  \graphicsymbol{O} / O
    \item [80]  \graphicsymbol{P} / P
    \item [81]  \graphicsymbol{Q} / Q
    \item [82]  \graphicsymbol{R} / R
    \item [83]  \graphicsymbol{S} / S
    \item [84]  \graphicsymbol{T} / T
    \item [85]  \graphicsymbol{U} / U
    \item [86]  \graphicsymbol{V} / V
    \item [87]  \graphicsymbol{W} / W
    \item [88]  \graphicsymbol{X} / X
    \item [89]  \graphicsymbol{Y} / Y
    \item [90]  \graphicsymbol{Z} / Z
    \item [91]  \graphicsymbol{+}
    \item [92]  \graphicsymbol{-}
    \item [93]  \graphicsymbol{B}
    \item [94]  \graphicsymbol{\textbackslash} %/ TODO: inverted checker
    \item [95]  \graphicsymbol{]} %/ TODO: back slashes
    \item [96]  space
    \item [97]  \graphicsymbol{j}
    \item [98]  \graphicsymbol{i}
    \item [99]  \graphicsymbol{t}
    \item [100] \graphicsymbol{[}
    \item [101] \graphicsymbol{g}
    \item [102] \graphicsymbol{=}
    \item [103] \graphicsymbol{m}
    \item [104] \graphicsymbol{/}
    \item [105] \graphicsymbol{?} %/ TODO: forward slashes
    \item [106] \graphicsymbol{n}
    \item [107] \graphicsymbol{q}
    \item [108] \graphicsymbol{d}
    \item [109] \graphicsymbol{z}
    \item [110] \graphicsymbol{s}
    \item [111] \graphicsymbol{p}
    \item [112] \graphicsymbol{a}
    \item [113] \graphicsymbol{e}
    \item [114] \graphicsymbol{r}
    \item [115] \graphicsymbol{w}
    \item [116] \graphicsymbol{h}
    \item [117] \graphicsymbol{j}
    \item [118] \graphicsymbol{l}
    \item [119] \graphicsymbol{y}
    \item [120] \graphicsymbol{u}
    \item [121] \graphicsymbol{p}
    \item [122] \graphicsymbol{\{} %/ TODO: checkmark
    \item [123] \graphicsymbol{f}
    \item [124] \graphicsymbol{c}
    \item [125] \graphicsymbol{x}
    \item [126] \graphicsymbol{v}
    \item [127] \graphicsymbol{b}
\end{description}
\end{multicols}
\end{adjustwidth}

\underline{Note}: In the built-in character sets, codes 128-255 are reversed versions of 0-127.

% TODO: Remove this once we have real images representing this.
\underline{Note}: In the lowercase group, 94 is an inverted version of \graphicsymbol{=}. 95 is a diagonal line pattern. 105 is the diagonal line pattern in the other direction. 122 is a checkmark.
