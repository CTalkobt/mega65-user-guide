\chapter{Assemblers}

The table below shows an overview of assemblers known to work with MEGA65.
For general use we recommend {\bf ACME} as it has good support
for the 45GS02 instruction set; is open source; and finally written in C. The latter
means that it may be ported to run natively on the MEGA65 in the future.

\begin{longtable}{ | l | l | l | l |}\hline
Name     & 45GS02 & Source & Reference \\\hline
ACME     &  yes   & C      & \url{https://sourceforge.net/projects/acme-crossass}\\
KickAss  &  yes   & Java   & \href{https://gitlab.com/jespergravgaard/kickassembler65ce02}{gitlab.com/jespergravgaard/kickassembler65ce02}\\
LLVM-MOS &  yes   & C++    & \href{https://llvm-mos.org/wiki/Welcome}\\
Ophis    &  yes   & Python & \url{https://github.com/michaelcmartin/Ophis}\\
BSA      &  yes   & C      & \url{https://github.com/Edilbert/BSA}\\
CA65     &  no\footnote{Our fork of CA65 (part of CC65) correctly detects the MEGA65's CPU, but has no explicit support for the processor's features} & C & \url{https://github.com/mega65/cc65}\\\hline
\end{longtable}

The {\bf BSA} assembler is currently used to build the {\bf MEGA65.ROM}.
Most of this source code is written in the syntax
of the ancient {\bf BSO} assembler (Boston Systems Office), which was used in the
years 1989 - 1991 by software developers, working on the C65.
The {\bf BSA} Assembler has a compatibility mode, which makes it
possible to assemble these old source codes with minor or none modifications.
The {\bf BSA} Assembler has currently only a description of commands
embedded in the C-source of the assembler.

The {\bf LLVM-MOS} project comes with a GNU compatible macro assembler and disassembler that supports all
mnemonics of the 45GS02. It can be used as a stand-alone tool with the command {\bf llvm-mc}.

%Therefore a chapter, describing the usage and the features of {\bf BSA}
%is started after this chapter and will be completed during the next weeks.

