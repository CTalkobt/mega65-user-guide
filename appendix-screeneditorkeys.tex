\chapter{Screen Editor Keys}

\section{Screen Editor Keys}

The following key combinations perform actions in the MEGA65 screen editor.

In some cases, a program can print the equivalent PETSCII codes to perform the same actions. For example, \specialkey{CTRL} + \megakey{G}, which plays a bell sound, can be printed by a program as \screentext{CHR\$(7)}. To print an \specialkey{ESC} sequence, use \screentext{CHR\$(27)} to represent the \specialkey{ESC} key, followed by the next key in the sequence.

\section{Control codes}
\label{appendix:controlcodes}
\index{Keyboard!CTRL}
\begin{center}
\setlength{\def\arraystretch{1.5}\tabcolsep}{6pt}
\begin{longtable}{c|L{5.5cm}}
	\textbf{Keyboard Control} & \textbf{Function}\\
   \hhline{==}
	\endhead

  \multicolumn{2}{l}{\textbf{Colours}} \\
  \hhline{==}
\specialkey{CTRL} + \megakey{1} to \megakey{8} &
Choose from the first range of colours. More information on the colours
    available is under the BASIC {\bf BACKGROUND} command on page \pageref{colourtable}.\\
\hline
\megasymbolkey + \megakey{1} to \megakey{8} &
Choose from the second range of colours.  \\
\hline
\specialkey{CTRL} + \megakey{E} &
Restores the colour of the cursor back to the default (white).\\
\hline
\specialkey{CTRL} + \megakey{D} &
Switches the VIC-IV to colour range 0-15 (default colours). These colours can be accessed with \specialkey{CTRL} and keys \megakey{1} to \megakey{8} (for the first 8 colours), or \megasymbolkey and keys \megakey{1} to \megakey{8} (for the remaining 8 colours).\\
\hline
\specialkey{CTRL} + \megakey{A} &
Switches the VIC-IV to colour range 16-31 (alternate/rainbow colours). These colours can be accessed with \specialkey{CTRL} and keys \megakey{1} to \megakey{8} (for the first 8 colours), or \megasymbolkey and keys \megakey{1} to \megakey{8} (for the remaining 8 colours).\\
  \hhline{==}
  \multicolumn{2}{l}{\textbf{Tabs}} \\
  \hhline{==}
\specialkey{CTRL} + \megakey{Z} &
Tabs the cursor to the left. If there are no tab positions remaining, the cursor will remain at the start of the line.\\
\hline
\specialkey{CTRL} + \megakey{I} &
Tabs the cursor to the right. If there are no tab positions remaining, the cursor will remain at the end of the line.\\
\hline
\specialkey{CTRL} + \megakey{X} &
Sets or clears the current screen column as a tab position.
 Use \specialkey{CTRL} +\megakey{Z} and \megakey{I}  to jump back and forth to all positions set with \megakey{X}.\\
  \hhline{==}
  \multicolumn{2}{l}{\textbf{Movement}} \\
  \hhline{==}
\specialkey{CTRL} + \megakey{Q} &
Moves the cursor down one line at a time. Equivalent to \megakey{$\downarrow$}.\\
\hline
\specialkey{CTRL} + \megakey{J} &
Moves the cursor down a position. If you are on a long line of BASIC code that has extended to two lines, then the cursor will move down two rows to be on the next line.\\
\hline
\specialkey{CTRL} + \megakey{]} &
Equivalent to \megakey{$\rightarrow$}.\\
\hline
\specialkey{CTRL} + \megakey{T} &
Backspace the character immediately to the left and to shift all rightmost characters one position to the left. This is equivalent to \specialkey{INST DEL}.\\
\hline
\specialkey{CTRL} + \megakey{M} &
Performs a carriage return, equivalent to \specialkey{RETURN}.\\

  \hhline{==}
  \multicolumn{2}{l}{\textbf{Word movement}} \\
  \hhline{==}

\specialkey{CTRL} + \megakey{U} &
Moves the cursor backward to the start of the previous word. If there is no previous word on the current line, it moves to the first column of the current line, then to the previous line, until a line with a word is encountered.\\
\hline
\specialkey{CTRL} + \megakey{W} &
Advances the cursor forward to the start of the next word. If there is no next word on the current line, it moves to the first column of the next line, until a line with a word is encountered.\\

  \hhline{==}
  \multicolumn{2}{l}{\textbf{Scrolling}} \\
  \hhline{==}

\specialkey{CTRL} + \megakey{P} &
Scroll BASIC listing down one line. Equivalent to \megakey{F9}.\\
\hline
\specialkey{CTRL} + \megakey{V} &
Scroll BASIC listing up one line. Equivalent to \megakey{F11}.\\
\hline
\specialkey{CTRL} + \megakey{S} &
Equivalent to \specialkey{NO\\SCROLL}.\\

  \hhline{==}
  \multicolumn{2}{l}{\textbf{Formatting}} \\
  \hhline{==}

\specialkey{CTRL} + \megakey{B} &
Enables underline text mode. You can disable underline mode by pressing \specialkey{ESC}, then \megakey{O}.\\
\hline
\specialkey{CTRL} + \megakey{O} &
Enables flashing text mode. You can disable flashing mode by pressing \specialkey{ESC}, then \megakey{O}.\\
\hline

  \hhline{==}
  \multicolumn{2}{l}{\textbf{Casing}} \\
  \hhline{==}

\specialkey{CTRL} + \megakey{N} &
Changes the text case mode from uppercase to lowercase.\\
\hline
\specialkey{CTRL} + \megakey{K} &
Locks the uppercase/lowercase mode switch usually performed with \megasymbolkey + \specialkey{SHIFT}.\\
\hline
\specialkey{CTRL} + \megakey{L} &
Enables the uppercase/lowercase mode switch that is performed with the \megasymbolkey + \specialkey{SHIFT}.\\

  \hhline{==}
  \multicolumn{2}{l}{\textbf{Miscellaneous}} \\
  \hhline{==}

\specialkey{CTRL} + \megakey{G} &
Produces a bell tone.\\
\hline
\specialkey{CTRL} + \megakey{[} &
Equivalent to pressing \specialkey{ESC}.\\
\hline
\specialkey{CTRL} + \megakey{*} &
Enters the Matrix Mode Debugger.\\
\hline

\end{longtable}
\end{center}


\section{Shifted codes}
\label{appendix:shiftedcodes}
\index{Keyboard!Shift Keys}
\begin{center}
\setlength{\def\arraystretch{1.5}\tabcolsep}{6pt}
\begin{longtable}{c|L{5.5cm}}
	\textbf{Keyboard Control} & \textbf{Function}\\
  \hhline{==}
	\endhead

\specialkey{SHIFT} + \specialkey{INST DEL} &
Insert a character at the current cursor position and move all characters to the right by one position.\\
\hline
\specialkey{SHIFT} + \specialkey{HOME} &
Clear home, clear the entire screen, and move the cursor to the home position.\\
\hline

\end{longtable}
\end{center}



\section{Escape Sequences}
\label{escape-sequences}
\index{Keyboard!Escape Sequences}
To perform an Escape Sequence, briefly press and release \specialkey{ESC},
then press one of the following keys to perform the sequence.

\begin{center}
\setlength{\def\arraystretch{1.5}\tabcolsep}{6pt}
\begin{longtable}{c|L{5.5cm}}
	\textbf{Key} & \textbf{Sequence}\\
  \hhline{==}
	\endhead

  \multicolumn{2}{l}{\textbf{Editor behaviour}} \\
  \hhline{==}

\specialkey{ESC} \megakey{X} &
Clears the screen and toggles between 40 $\times$ 25 and 80 $\times$ 25 text modes.\\
\hline

\specialkey{ESC} \megakey{4} &
Clears the screen and switches to 40 $\times$ 25 text mode.\\
\hline

\specialkey{ESC} \megakey{8} &
Clears the screen and switches to 80 $\times$ 25 text mode.\\
\hline

\specialkey{ESC} \megakey{5} &
Switches to 80 $\times$ 50 text mode.\\
\hline
\multicolumn{2}{L{7cm}}{
  Note that some programs expect to be started in 80 $\times$ 25 mode,
  and may not behave correctly when started in 80 $\times$ 50 mode.} \\
\hline

\specialkey{ESC} \megakey{@} &
%TODO the megakey with @ shows the label far smaller than any other key.
% Perhaps it needs more styling?
Clears a region of the screen, starting from the current cursor position, to the end of the screen.\\
\hline

\specialkey{ESC} \megakey{O} &
Cancels the quote, reverse, underline, and flash modes.\\

  \hhline{==}
  \multicolumn{2}{l}{\textbf{Scrolling}} \\
  \hhline{==}
% TODO for some reason this doesn't leave enough space (the keys overlap with the hhline)
\specialkey{ESC} \megakey{V} &
Scrolls the entire screen up one line.\\
\hline
\specialkey{ESC} \megakey{W} &
Scrolls the entire screen down one line.\\
\hline
\specialkey{ESC} \megakey{L} &
Enables scrolling when \megakey{$\downarrow$} is pressed at the bottom of the screen.\\
\hline
\specialkey{ESC} \megakey{M} &
Disables scrolling. When pressing \megakey{$\downarrow$} at the bottom of the screen, the cursor will move to the top of the
screen. However, when pressing \megakey{$\uparrow$} at the top of the screen, the cursor will remain on the first line.\\
  \hhline{==}
  \multicolumn{2}{l}{\textbf{Insertion and deletion}} \\
  \hhline{==}
\specialkey{ESC} \megakey{I} &
Inserts an empty line at the current cursor position and moves all subsequent lines down one position.\\
\hline
\specialkey{ESC} \megakey{D} &
Deletes the current line and moves lines below the cursor up one position.\\
\hline
\specialkey{ESC} \megakey{P} &
Erases all characters from the cursor to the start of the current line.\\
\hline
\specialkey{ESC} \megakey{Q} &
Erases all characters from the cursor to the end of the current line.\\
  \hhline{==}
  \multicolumn{2}{l}{\textbf{Movement}} \\
  \hhline{==}
\specialkey{ESC} \megakey{J} &
Moves the cursor to the start of the current line.\\
\hline
\specialkey{ESC} \megakey{K} &
Moves the cursor to the last non-whitespace character on the current line.\\
\hline
\specialkey{ESC} \megakeywhite{$\uparrow$} &
\index{Keyboard!Cursor Keys}
Saves the current cursor position. Use \specialkey{ESC} \megakeywhite{$\leftarrow$} (next to \megakey{1}) to move it
    back
to the saved position. Note that the \megakeywhite{$\uparrow$} used here is next to \widekey{RESTORE}.\\
\hline
\specialkey{ESC} \megakeywhite{$\leftarrow$} &
Restores the cursor position to the position stored via a prior a press of the \specialkey{ESC}
    \megakeywhite{$\uparrow$}
(next to \widekey{RESTORE}) key sequence. Note that the \megakeywhite{$\leftarrow$} used here is next to \megakey{1}.\\
\hline
\specialkey{ESC} \specialkey{HOME} &
Restores the cursor position to the position stored via a prior a press of \specialkey{HOME}.\\

  \hhline{==}
  \multicolumn{2}{l}{\textbf{Windowing}} \\
  \hhline{==}
\specialkey{ESC} \megakey{T} &
Sets the top-left corner of the windowed area.
All typed characters and screen activity will be restricted to the area. Also see \specialkey{ESC} \megakey{B}.
Windowed mode can be disabled by pressing \specialkey{CLR HOME} twice. \\
\hline
\specialkey{ESC} \megakey{B} &
Sets the bottom right corner of the windowed area.
All typed characters and screen activity will be restricted to the area. Also see \specialkey{ESC} \megakey{T}.
Windowed mode can be disabled by pressing \specialkey{CLR HOME} twice.\\
  \hhline{==}
  \multicolumn{2}{l}{\textbf{Cursor behaviour}} \\
  \hhline{==}
\specialkey{ESC} \megakey{A} &
Enables auto-insert mode. Any keys pressed will be inserted at the current cursor position, shifting all characters
on the current line after the cursor to the right by one position.\\
\hline
\specialkey{ESC} \megakey{C} &
Disables auto-insert mode, reverting back to overwrite mode.\\
\hline
\specialkey{ESC} \megakey{E} &
Sets the cursor to non-flashing mode.\\
\hline
\specialkey{ESC} \megakey{F} &
Sets the cursor to regular flashing mode.\\
  \hhline{==}
  \multicolumn{2}{l}{\textbf{Bell behaviour}} \\
  \hhline{==}
\specialkey{ESC} \megakey{G} &
Enables the bell which can be sounded using \specialkey{CTRL} and \megakey{G}.\\
\hline
\specialkey{ESC} \megakey{H} &
Disable the bell so that pressing \specialkey{CTRL} and \megakey{G} will have no effect.\\
  \hhline{==}
  \multicolumn{2}{l}{\textbf{Colours}} \\
  \hhline{==}
\specialkey{ESC} \megakey{U} &
\label{appendix:escape-colours}
Switches the VIC-IV to colour range 0-15 (default colours). These colours can be accessed with \specialkey{CTRL} and keys \megakey{1} to \megakey{8} (for the first 8 colours), or \megasymbolkey and keys \megakey{1} to \megakey{8} (for the remaining colours).\\
\hline
\specialkey{ESC} \megakey{S} &
Switches the VIC-IV to colour range 16-31 (alternate/rainbow colours). These colours can be accessed with \specialkey{CTRL} and keys \megakey{1} to \megakey{8} (for the first 8 colours), or \megasymbolkey and keys \megakey{1} to \megakey{8} (for the remaining colours).\\
\hline
  \hhline{==}
  \multicolumn{2}{l}{\textbf{Tabs}} \\
  \hhline{==}
\specialkey{ESC} \megakey{Y} &
Set the default tab stops (every 8 spaces) for the entire screen.\\
\hline
\specialkey{ESC} \megakey{Z} &
Clears all tab stops. Any tabbing with \specialkey{CTRL} and \megakey{I} will move the cursor to the end of the line.\\
\hline
\end{longtable}
\end{center}
