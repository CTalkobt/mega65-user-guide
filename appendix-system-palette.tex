\chapter{System Palette}

\section{System Palette}
\label{appendix:colourtable}

The following table describes the system colour palette as it is defined by default. These palette entries can be changed with the {\bf PALETTE COLOR} command, and restored to their defaults with the {\bf PALETTE RESTORE} command.

Colour palette indexes are used as values in the {\bf C@\&()} special array, and as arguments for BASIC commands such as {\bf BACKGROUND}, {\bf BORDER}, {\bf COLOR}, {\bf FOREGROUND}, {\bf HIGHLIGHT}, {\bf PEN}, and {\bf SCREEN CLR}.

\definecolor{m65black}{RGB}{0, 0, 0}
\definecolor{m65white}{RGB}{255, 255, 255}
\definecolor{m65red}{RGB}{255, 0, 0}
\definecolor{m65cyan}{RGB}{0, 255, 255}
\definecolor{m65purple}{RGB}{255, 0, 255}
\definecolor{m65green}{RGB}{0, 255, 0}
\definecolor{m65blue}{RGB}{0, 0, 255}
\definecolor{m65yellow}{RGB}{255, 255, 0}
\definecolor{m65orange}{RGB}{255, 111, 0}
\definecolor{m65brown}{RGB}{175, 79, 0}
\definecolor{m65lightred}{RGB}{255, 127, 127}
\definecolor{m65darkgrey}{RGB}{95, 95, 95}
\definecolor{m65mediumgrey}{RGB}{143, 143, 143}
\definecolor{m65lightgreen}{RGB}{159, 255, 159}
\definecolor{m65lightblue}{RGB}{159, 159, 255}
\definecolor{m65lightgrey}{RGB}{191, 191, 191}
\definecolor{m65gurumeditation}{RGB}{239, 0, 0}
\definecolor{m65rambutan}{RGB}{255, 95, 0}
\definecolor{m65carrot}{RGB}{255, 191, 0}
\definecolor{m65lemontart}{RGB}{239, 239, 0}
\definecolor{m65pandan}{RGB}{127, 255, 0}
\definecolor{m65seasickgreen}{RGB}{111, 239, 111}
\definecolor{m65soylentgreen}{RGB}{0, 239, 61}
\definecolor{m65slimergreen}{RGB}{0, 255, 159}
\definecolor{m65theothercyan}{RGB}{0, 223, 223}
\definecolor{m65seasky}{RGB}{0, 159, 255}
\definecolor{m65smurfblue}{RGB}{0, 63, 255}
\definecolor{m65screenofdeath}{RGB}{0, 0, 239}
\definecolor{m65plumsauce}{RGB}{127, 0, 255}
\definecolor{m65sourgrape}{RGB}{207, 0, 255}
\definecolor{m65bubblegum}{RGB}{255, 0, 191}
\definecolor{m65hottamales}{RGB}{255, 61, 111}

\begin{center}
    {\setlength{\tabcolsep}{1mm}
    \begin{tabular}{*{4}{|R{1.2cm}}|l|}
    \hline
    {\bf Index}  & {\bf Red} & {\bf Green} & {\bf Blue} & {\bf Colour} \\
    \hline
      {\bf 0} &    0  &   0   &  0   & \tikz[scale=0.3] \draw[fill=m65black] (0,0) rectangle (2,1); Black \\
      {\bf 1} &   15  &  15   & 15   & \tikz[scale=0.3] \draw[fill=m65white] (0,0) rectangle (2,1); White \\
      {\bf 2} &   15  &   0   &  0   & \tikz[scale=0.3] \draw[fill=m65red] (0,0) rectangle (2,1); Red   \\
      {\bf 3} &    0  &  15   & 15   & \tikz[scale=0.3] \draw[fill=m65cyan] (0,0) rectangle (2,1); Cyan  \\
      {\bf 4} &   15  &   0   & 15   & \tikz[scale=0.3] \draw[fill=m65purple] (0,0) rectangle (2,1); Purple\\
      {\bf 5} &    0  &  15   &  0   & \tikz[scale=0.3] \draw[fill=m65green] (0,0) rectangle (2,1); Green \\
      {\bf 6} &    0  &   0   & 15   & \tikz[scale=0.3] \draw[fill=m65blue] (0,0) rectangle (2,1); Blue  \\
      {\bf 7} &   15  &  15   &  0   & \tikz[scale=0.3] \draw[fill=m65yellow] (0,0) rectangle (2,1); Yellow\\
      {\bf 8} &   15  &   6   &  0   & \tikz[scale=0.3] \draw[fill=m65orange] (0,0) rectangle (2,1); Orange\\
      {\bf 9} &   10  &   4   &  0   & \tikz[scale=0.3] \draw[fill=m65brown] (0,0) rectangle (2,1); Brown \\
      {\bf 10} &   15  &   7   &  7   & \tikz[scale=0.3] \draw[fill=m65lightred] (0,0) rectangle (2,1); Light Red (Pink)  \\
      {\bf 11} &    5  &   5   &  5   & \tikz[scale=0.3] \draw[fill=m65darkgrey] (0,0) rectangle (2,1); Dark Grey\\
      {\bf 12} &    8  &   8   &  8   & \tikz[scale=0.3] \draw[fill=m65mediumgrey] (0,0) rectangle (2,1); Medium Grey\\
      {\bf 13} &    9  &  15   &  9   & \tikz[scale=0.3] \draw[fill=m65lightgreen] (0,0) rectangle (2,1); Light Green \\
      {\bf 14} &    9  &   9   & 15   & \tikz[scale=0.3] \draw[fill=m65lightblue] (0,0) rectangle (2,1); Light Blue\\
      {\bf 15} &   11  &  11   & 11   & \tikz[scale=0.3] \draw[fill=m65lightgrey] (0,0) rectangle (2,1); Light Grey\\
    \hline
      {\bf 16} &   14  &   0   &  0   & \tikz[scale=0.3] \draw[fill=m65gurumeditation] (0,0) rectangle (2,1); Guru Meditation\\
      {\bf 17} &   15  &   5   &  0   & \tikz[scale=0.3] \draw[fill=m65rambutan] (0,0) rectangle (2,1); Rambutan\\
      {\bf 18} &   15  &  11   &  0   & \tikz[scale=0.3] \draw[fill=m65carrot] (0,0) rectangle (2,1); Carrot\\
      {\bf 19} &   14  &  14   &  0   & \tikz[scale=0.3] \draw[fill=m65lemontart] (0,0) rectangle (2,1); Lemon Tart\\
      {\bf 20} &    7  &  15   &  0   & \tikz[scale=0.3] \draw[fill=m65pandan] (0,0) rectangle (2,1); Pandan\\
      {\bf 21} &    6  &  14   &  6   & \tikz[scale=0.3] \draw[fill=m65seasickgreen] (0,0) rectangle (2,1); Seasick Green\\
      {\bf 22} &    0  &  14   &  3   & \tikz[scale=0.3] \draw[fill=m65soylentgreen] (0,0) rectangle (2,1); Soylent Green\\
      {\bf 23} &    0  &  15   &  9   & \tikz[scale=0.3] \draw[fill=m65slimergreen] (0,0) rectangle (2,1); Slimer Green\\
      {\bf 24} &    0  &  13   &  13  & \tikz[scale=0.3] \draw[fill=m65theothercyan] (0,0) rectangle (2,1); The Other Cyan\\
      {\bf 25} &    0  &   9   &  15  & \tikz[scale=0.3] \draw[fill=m65seasky] (0,0) rectangle (2,1); Sea Sky\\
      {\bf 26} &    0  &   3   &  15  & \tikz[scale=0.3] \draw[fill=m65smurfblue] (0,0) rectangle (2,1); Smurf Blue\\
      {\bf 27} &    0  &   0   &  14  & \tikz[scale=0.3] \draw[fill=m65screenofdeath] (0,0) rectangle (2,1); Screen of Death\\
      {\bf 28} &    7  &   0   &  15  & \tikz[scale=0.3] \draw[fill=m65plumsauce] (0,0) rectangle (2,1); Plum Sauce\\
      {\bf 29} &   12  &   0   &  15  & \tikz[scale=0.3] \draw[fill=m65sourgrape] (0,0) rectangle (2,1); Sour Grape\\
      {\bf 30} &   15  &   0   &  11  & \tikz[scale=0.3] \draw[fill=m65bubblegum] (0,0) rectangle (2,1); Bubblegum\\
      {\bf 31} &   15  &   3   &   6  & \tikz[scale=0.3] \draw[fill=m65hottamales] (0,0) rectangle (2,1); Hot Tamales\\
    \hline
    \end{tabular}
    }
    \end{center}

You can use these 32 colours with the screen terminal by typing or printing PETSCII control codes. The cursor draw state is either in the first 16 colours or the second 16 colours, and this state can be changed with control codes. Each of 16 control codes selects the colour for the cursor within that set.

To select colour set A, press \specialkey{CTRL} + \megakey{D}, or print PETSCII code 4.

To select colour set B, press \specialkey{CTRL} + \megakey{A}, or print PETSCII code 1.

\begin{center}
  {\setlength{\tabcolsep}{1mm}\renewcommand{\arraystretch}{1.5}
  \begin{tabular}{|l|R{1.3cm}|l|l|}
  \hline
  {\bf Key} & {\bf PETSCII} & {\bf Colour A} & {\bf Colour B} \\
  \hline
  \specialkey{CTRL} + \megakey{1} & 144 & \tikz[scale=0.3] \draw[fill=m65black] (0,0) rectangle (2,1); Black & \tikz[scale=0.3] \draw[fill=m65gurumeditation] (0,0) rectangle (2,1); Guru Meditation\\
  \hline
  \specialkey{CTRL} + \megakey{2} & 5 & \tikz[scale=0.3] \draw[fill=m65white] (0,0) rectangle (2,1); White & \tikz[scale=0.3] \draw[fill=m65rambutan] (0,0) rectangle (2,1); Rambutan\\
  \hline
  \specialkey{CTRL} + \megakey{3} & 28 & \tikz[scale=0.3] \draw[fill=m65red] (0,0) rectangle (2,1); Red & \tikz[scale=0.3] \draw[fill=m65carrot] (0,0) rectangle (2,1); Carrot\\
  \hline
  \specialkey{CTRL} + \megakey{4} & 159 & \tikz[scale=0.3] \draw[fill=m65cyan] (0,0) rectangle (2,1); Cyan & \tikz[scale=0.3] \draw[fill=m65lemontart] (0,0) rectangle (2,1); Lemon Tart\\
  \hline
  \specialkey{CTRL} + \megakey{5} & 156 & \tikz[scale=0.3] \draw[fill=m65purple] (0,0) rectangle (2,1); Purple & \tikz[scale=0.3] \draw[fill=m65pandan] (0,0) rectangle (2,1); Pandan\\
  \hline
  \specialkey{CTRL} + \megakey{6} & 30 & \tikz[scale=0.3] \draw[fill=m65green] (0,0) rectangle (2,1); Green & \tikz[scale=0.3] \draw[fill=m65seasickgreen] (0,0) rectangle (2,1); Seasick Green\\
  \hline
  \specialkey{CTRL} + \megakey{7} & 31 & \tikz[scale=0.3] \draw[fill=m65blue] (0,0) rectangle (2,1); Blue & \tikz[scale=0.3] \draw[fill=m65soylentgreen] (0,0) rectangle (2,1); Soylent Green\\
  \hline
  \specialkey{CTRL} + \megakey{8} & 158 & \tikz[scale=0.3] \draw[fill=m65yellow] (0,0) rectangle (2,1); Yellow & \tikz[scale=0.3] \draw[fill=m65slimergreen] (0,0) rectangle (2,1); Slimer Green\\
  \hline
  \megasymbolkey + \megakey{1} & 129 & \tikz[scale=0.3] \draw[fill=m65orange] (0,0) rectangle (2,1); Orange & \tikz[scale=0.3] \draw[fill=m65theothercyan] (0,0) rectangle (2,1); The Other Cyan\\
  \hline
  \megasymbolkey + \megakey{2} & 149 & \tikz[scale=0.3] \draw[fill=m65brown] (0,0) rectangle (2,1); Brown & \tikz[scale=0.3] \draw[fill=m65seasky] (0,0) rectangle (2,1); Sea Sky\\
  \hline
  \megasymbolkey + \megakey{3} & 150 & \tikz[scale=0.3] \draw[fill=m65lightred] (0,0) rectangle (2,1); Light Red (Pink) & \tikz[scale=0.3] \draw[fill=m65smurfblue] (0,0) rectangle (2,1); Smurf Blue\\
  \hline
  \megasymbolkey + \megakey{4} & 151 & \tikz[scale=0.3] \draw[fill=m65darkgrey] (0,0) rectangle (2,1); Dark Grey & \tikz[scale=0.3] \draw[fill=m65screenofdeath] (0,0) rectangle (2,1); Screen of Death\\
  \hline
  \megasymbolkey + \megakey{5} & 152 & \tikz[scale=0.3] \draw[fill=m65mediumgrey] (0,0) rectangle (2,1); Medium Grey & \tikz[scale=0.3] \draw[fill=m65plumsauce] (0,0) rectangle (2,1); Plum Sauce\\
  \hline
  \megasymbolkey + \megakey{6} & 153 & \tikz[scale=0.3] \draw[fill=m65lightgreen] (0,0) rectangle (2,1); Light Green & \tikz[scale=0.3] \draw[fill=m65sourgrape] (0,0) rectangle (2,1); Sour Grape\\
  \hline
  \megasymbolkey + \megakey{7} & 154 & \tikz[scale=0.3] \draw[fill=m65lightblue] (0,0) rectangle (2,1); Light Blue & \tikz[scale=0.3] \draw[fill=m65bubblegum] (0,0) rectangle (2,1); Bubblegum\\
  \hline
  \megasymbolkey + \megakey{8} & 155 & \tikz[scale=0.3] \draw[fill=m65lightgrey] (0,0) rectangle (2,1); Light Grey & \tikz[scale=0.3] \draw[fill=m65hottamales] (0,0) rectangle (2,1); Hot Tamales\\
  \hline
\end{tabular}
  }
\end{center}
