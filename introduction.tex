\chapter{Introduction}


\section{Welcome to the MEGA65!}

Congratulations on your purchase of one of the most long-awaited computers in the history of computing! The MEGA65 is community designed, and based on the never-released Commodore{\textregistered} 65\footnote{Commodore is a trademark of C= Holdings} computer; a computer designed in 1989 and intended for public release in 1990. Decades have passed, and we have endeavoured to invoke memories of an earlier time when computers were simple and friendly. They were not only simple to operate and understand, but friendly and approachable for new users.

These 1980s computers inspired many of their owners to pursue the exciting and rewarding technology careers they have today. Just imagine the exhilaration these early computing pioneers experienced, as they learned they could use their new computer to solve problems, write a letter, prepare taxes, invent new things, discover how the universe works, and perhaps even play an exciting game or two! We want to re-awaken that same level of excitement (which alas, is no longer found in modern computing), so we have created the {\bf MEGA65}.

The MEGA65 team believes that owning a computer is like owning a home. You don't just use a home; you change things, big and small, to make it your own custom living space. After a while, when you settle in, you may decide to renovate or expand your home to make it more comfortable, or provide more utility. Think of the MEGA65 as your very own ``computing home''.

This guide will teach you how to do more than just hang pictures on a wall; it will show you how to build your dream home. While you read this user's guide, you will learn how to operate the MEGA65, write programs, add additional software, and extend hardware capabilities. What won't be immediately obvious is that along the journey, you will also learn about the history of computing as you explore the many facets of BASIC version 65 and operating system commands.

Computer graphics and music make computing more fun, and we designed the MEGA65 to be fun! In this user's guide, you will learn how to write programs using the MEGA65's built-in {\bf graphics} and {\bf sound} capabilities. But you don't need to be a programmer to have fun with the MEGA65. Because the MEGA65 includes a complete Commodore{\textregistered} 64{\texttrademark}\footnote{Commodore 64 is a trademark of C= Holdings}, it can also run thousands of existing games, utilities, and business software packages, as well as new programs being written today by Commodore computer enthusiasts. Excitement for the MEGA65 will grow as we all witness the programming marvels our MEGA65 community create, as they (and you!) discover and master the powerful capabilities of this modern Commodore computer recreation. Together, we can build a new ``homebrew'' community, teeming with software and projects that push the MEGA65's capabilities far beyond what anyone thought would be possible.

We welcome you on this journey! Thank you for becoming a part of the MEGA65
community of users, programmers, and enthusiasts!

\section{Other Books in this Series}

This book is one of several within the MEGA65 documentation suite. The series includes:

\begin{itemize}
  \item {\bf The MEGA65 User's Guide} \newline
      Provides an introduction to the MEGA65, and a condensed BASIC 65 command reference
    \item {\bf The MEGA65 BASIC 65 Reference} \newline
      Comprehensive documentation of all BASIC 65 commands, functions and operators
    \item {\bf The MEGA65 Chipset Reference} \newline
      Detailed documentation about the MEGA65 and C65's custom chips
    \item {\bf The MEGA65 Developer's Guide} \newline
      Information for developers who wish to write programs for the MEGA65
    \item {\bf The MEGA65 Complete Compendium} \newline
      (Also known as {\bf The MEGA65 Book}) \newline
      All volumes in a single huge PDF for easy searching. 1200 pages and growing!
\end{itemize}

\section{Come Join Us!}
Get involved, learn more about your MEGA65, and join us online at:

\begin{itemize}
    \item \url{https://mega65.org/chat}
    \item \url{https://mega65.org/forum}
\end{itemize}
