% TODO: index terms

\chapter{Programming with Memory}
\label{cha:programming-with-memory}

When you write programs in assembly language, understanding how memory works is
crucial. Nearly every operation interacts with memory in some form. Data
processing happens in memory, and programs are stored in and executed from memory.
With a Commodore-style hardware register architecture, every aspect of the
computer's input and output involves manipulating values through the memory system.

You can write a substantial program in MEGA65 BASIC without having to
think about the computer's memory. The BASIC system and most of its commands
represent data in terms of numbers, arrays, and strings stored in named variables.
You never have to know how that data is represented as 1's and 0's, or which addresses
refer to which data. That said, understanding how memory works can take your
BASIC programs to the next level. You can invent new ways to manage your
program's data, and interface directly with MEGA65 hardware. Some of the
MEGA65's advanced capabilities are only accessible from memory registers.

Like several systems in the MEGA65, the memory system is more sophisticated than you
might expect, with multiple modes of operation. When you first turn on your
MEGA65, it starts with a memory configuration intended to support the
BASIC 65 environment. A program can reconfigure the memory to take full
advantage of the MEGA65's capabilities, or to make the memory environment
compatible with a Commodore 64 (as \stw{GO64} does) or a Commodore 65.

In this chapter, you will learn all about the MEGA65 memory system, with a
focus on writing programs for the MEGA65 in BASIC or machine language. Complete
descriptions of the memory features, including backwards compatibility features
for the C64 and C65, are available in the appendicies.

\newpage
\section{Memory Concepts}
\label{sec:programming-with-memory-concepts}

A computer performs tasks by manipulating data stored in its {\em memory}.
Computer memory consists of many thousands of cells, each containing a single
{\em byte} of data capable of representing 256 possible values. Each cell has a numeric
{\em address}, and the program using the memory decides what kind of data is stored
at each address. The program itself is stored as bytes in memory, as
machine code instructions to be executed by the CPU.

\subsection{Bits, Bytes, and Nibbles}

A {\em bit} is a single digital signal with two possible values: off or on. A
bit is often combined with other bits to represent more possible values. A byte
is 8 bits, with 256 possible values.

Bits and bytes are often used to represent numbers, and often have their values
described as positive integers even when they represent other kinds of data.
When discussing the bits of a byte, the bits are arranged in increasing place
value from right to left. Bit 0 is the rightmost bit, also known as the
``lowest'' bit.  Bit 7 is the leftmost bit, also known as the ``highest'' bit.\footnote{See
\bookvref{cha:decimal-binary-and-hexadecimal} for a complete explanation of
binary and hexadecimal notation.}

\begin{center}
\begin{tabular}{ccccccccl}
{\bf b\textsubscript{7}} &
{\bf b\textsubscript{6}} &
{\bf b\textsubscript{5}} &
{\bf b\textsubscript{4}} &
{\bf b\textsubscript{3}} &
{\bf b\textsubscript{2}} &
{\bf b\textsubscript{1}} &
{\bf b\textsubscript{0}} & \\
\huge\texttt{0} &
\huge\texttt{1} &
\huge\texttt{1} &
\huge\texttt{0} &
\huge\texttt{1} &
\huge\texttt{0} &
\huge\texttt{1} &
\huge\texttt{1} & = 107 \\
\small{}128 &
\small{}64 &
\small{}32 &
\small{}16 &
\small{}8 &
\small{}4 &
\small{}2 &
\small{}1 & \\
\end{tabular}
\end{center}

A {\em nibble} is 4 bits. In computer architecture, the term ``nibble'' refers
exclusively to one of the two halves of a byte. The {\em high nibble} is bits 7 --
4, and the {\em low nibble} is bits 3 -- 0.

\subsection{Hexadecimal Notation}

Hexadecimal (``hex'') is a numbering system using sixteen possible values per digit,
written as numerals 0 -- 9 followed by letters A -- F. It is especially
convenient when describing values in computer memory because each hex digit
represents a nibble (4 bits) of data. The 8-bit binary value \texttt{\%01101011} can be
read as the two nibbles \texttt{0110 1011}, or \$6B in hexadecimal.

Memory addresses are often presented in hexadecimal notation. Computer memory
is typically organized in amounts that can be represented as round hexadecimal
numbers, like \$C000.

\begin{center}
\begin{tabular}{|c|c|c|c|c|c|c|c|c}
\cline{1-3}\cline{5-8}
\$0000 & \$0001 & \$0002 & $\cdots$ & \$0FFE & \$0FFF & \$1000 & \$1001 &
$\cdots$ \\
\cline{1-3}\cline{5-8}
\end{tabular}
\end{center}

Round hexadecimal numbers represent common memory sizes, like so:

\begin{center}
\begin{tabular}{rl}
\$0001 & 1 byte \\
\$0010 & 16 bytes \\
\$0100 & 256 bytes \\
\$0400 & 1 kilobyte (1KB = 1024 bytes) \\
\$1000 & 4 kilobytes (4KB = 4096 bytes) \\
\$4000 & 16KB \\
\$10000 & 64KB \\
\$100000 & 1 megabyte (1MB = 1024KB) \\
\end{tabular}
\end{center}

Address numbering starts at 0, so regions of these common sizes tend to start
with ``0'' digits on the right, and stop just before the start of the next region.
For example, \$2000 -- \$2FFF represents a 4KB region.

You can enter numbers as hexadecimal in BASIC using the dollar sign prefix
(\$). To display the decimal equivalent of a hexadecimal number in BASIC:

\begin{screencode}
PRINT $C000
\end{screencode}

To display a number in hexadecimal notation in BASIC, use the
{\bf HEX\$()}\index{BASIC 65 Functions!HEX\$} function:

\begin{screencode}
PRINT HEX$(127)
\end{screencode}

When programming in assembly language or using a machine language
monitor, addresses and values are presented in hexadecimal by default.

Much like how commas are used to separate decimal digits in long numbers, this
book sometimes uses a dot (.) to separate groups of four hex digits to make
them easier to read: FFF.FFFF. You do not type the dot when entering a long
address into a program or command. To enter FFF.FFFF in BASIC, type
\stw{\$FFFFFFF}.

Each hex digit represents a nibble of a value. The value FFF.FFFF can be
represented in seven nibbles (28 bits).

\subsection{Random Access Memory}

{\em Random Access Memory}, or RAM, is memory that contains data temporarily. A
program can read from and write to this data. When the computer is turned off,
this data is deleted. A program that needs to
preserve this data must save it to a long-term storage device like a floppy
disk (or the MEGA65 SD card) before the computer is powered down.

Try this. Enter the following command at the \screentext{READY.} prompt:

\begin{screencode}
POKE $1800,127
\end{screencode}

This {\bf POKE}\index{BASIC 65 Commands!POKE} command stores the byte value
127 at memory address \$1800. Now enter this command:

\begin{screencode}
PRINT PEEK($1800)
\end{screencode}

The {\bf PEEK()}\index{BASIC 65 Commands!PEEK} function evaluates to the byte
value stored at the memory address \$1800, in this case 127. The {\bf
PRINT}\index{BASIC 65 Commands!PRINT} command outputs this value.

Try turning off your MEGA65, turning it back on, and executing
\screentext{PRINT PEEK(\$1800)} again. The RAM value is reset.

\subsection{Read-Only Memory}

{\em Read-Only Memory}, or ROM, contains data that is written permanently at
the factory. Like other vintage microcomputers, the Commodore 65 has a ROM chip
that contains the program code for the operating system: BASIC, the kernel,
routines for accessing peripherals and performing common calculations. ROM
data cannot be overwritten directly by a program, and it remains in the
computer when the computer is turned off.

The MEGA65 does not contain a traditional ROM chip. Instead, the MEGA65
Hypervisor (the operating system that manages the features of the MEGA65)
loads the ``ROM'' program data from a file on the SD card into a region of RAM.
The Hypervisor protects that RAM as ``read only'' by default, simulating the ROM
chip of a Commodore 65. You can replace the ROM file on the SD card to upgrade
to newer versions of the MEGA65 ROM.

Try this. Enter these commands, and try to predict what the second command will
display. Change 63 to another number then try again.

\begin{screencode}
POKE $2100,63
PRINT PEEK($2100)
\end{screencode}

The address \$2100 refuses to accept a new value. A complete explanation
of why this is requires several concepts described in this chapter. For now, it
is sufficient to say that the memory at address \$2100 is configured to behave
as ROM when accessed in this way.

\subsection{I/O Registers}

Some addresses refer to special memory cells used by the computer's hardware.
These are known as {\em I/O registers}. You can interact with the hardware by writing
data to and reading data from these addresses. This is how BASIC commands
display graphics, play sounds, and read joysticks: they read and set values in
the I/O registers.

Try this command. Try replacing 7 with another number between 0 and 31.

\begin{screencode}
POKE $D020,7
\end{screencode}

The VIC video chip uses address \$D020 to store the colour of the screen border.
The {\bf BORDER} command sets this register to achieve the same effect.

\begin{screencode}
BORDER 7
\end{screencode}

\subsection{Addresses}

The examples so far have used addresses consisting of four hexadecimal digits,
such as \$1800. Each hexadecimal digit represents four bits (one nibble), so a
four-digit hexadecimal number represents a 16-bit address. Such an address is
in the range \$0000 -- \$FFFF, with 65,536 possible addresses. The Commodore
64's CPU uses 16-bit addresses to access up to 64 kilobytes (64KB) of memory.
In other words, the Commodore 64 has an \em{address space} of 16 bits, in the
range \$0000 -- \$FFFF.

The MEGA65 has an address space of 28 bits, in the
range 000.0000 -- FFF.FFFF. Such addresses can be stored as a 32-bit value with
the highest nibble set to 0. In hexadecimal, this would be an
eight-digit number with the leftmost digit of 0. (This digit is typically
omitted from notation.)

The BASIC {\bf POKE} command and {\bf PEEK()} function can use 28-bit
addresses. For example, to store then retrieve a value from RAM at address \$8000000:

\begin{screencode}
POKE $8000000,63
PRINT PEEK($8000000)
\end{screencode}

This example works on the MEGA65 and the DevKit. The Nexys board lacks RAM
at that address, and will ignore attempts to write it.

\subsection{16-bit Address Translation}

The MEGA65's memory system allows programs to refer to memory using 16-bit
addresses. To determine the full 28-bit address, the memory system translates
the address according to rules that can be set by the program. This translation
is also known as {\em remapping}, or, in some contexts, {\em
banking}.

16-bit address translation provides many advantages. The 45GS02 CPU processes
16-bit addresses faster than 32-bit addresses. 16-bit addresses take up less
space in memory when stored. Most program tasks operate in a single 64KB region
of memory, so it is often convenient for a program to set the memory
configuration at the beginning of a task and use 16-bit addresses.

Address translation is a common feature of most microcomputers. The Commodore
64 uses banking to allow programs to access 64KB of RAM {\em and} additional
ROM and I/O registers using only 16-bit addresses. The Commmodore 65 was
designed such that the memory system could be configured to resemble a
Commodore 64 memory map through 16-bit address translation, and also support
C64-style banking of ROM and I/O registers.

Overall, the MEGA65 provides multiple mechanisms of address translation. Memory
system configuration is one of the more complex topics of MEGA65 programming,
and this chapter covers it in detail.

\subsection{Banks and Pages}

MEGA65 addresses can be divided into regions of 64KB known as {\em banks}. It's
easy to identify the bank given an address specified in hexadecimal: the bank
number is the leftmost digits of the address, and the rightmost four digits are
a location within the bank.

\medskip
\begin{center}
\begin{tabular}{m{0.4cm}m{0.4cm}m{0.05cm}m{0.4cm}m{0.4cm}m{0.4cm}m{0.4cm}}
\multicolumn{1}{c}{\huge\texttt{\$}} &
\multicolumn{1}{c}{\huge\texttt{3}} &
\multicolumn{1}{c}{ } &
\multicolumn{1}{c}{\huge\texttt{F}} &
\multicolumn{1}{c}{\huge\texttt{D}} &
\multicolumn{1}{c}{\huge\texttt{A}} &
\multicolumn{1}{c}{\huge\texttt{8}} \\
\cline{2-2}\cline{4-5}
& \multicolumn{1}{c}{\small Bank} & & \multicolumn{2}{c}{\small Page} &
\multicolumn{2}{c}{ } \\
\end{tabular}
\end{center}
\medskip

For example, the address 3.FDA8 (or \$003FDA8) is an address in bank \$3. The
address 0.1800 (\$0001800) is an address in bank \$0. The address 880.12AB
(\$88012AB) is in bank \$880.\footnote{In a few cases, this is better understood as
``megabyte \$88, bank \$0, offset \$12AB.''}

It is important to remember that if you see a 16-bit
address such as \$2100, the actual bank that it refers to may not be bank 0.
The full address depends on how the memory system is configured, and how the
memory is accessed.

A bank can be further divided into regions of 256 bytes known as {\em pages}.
As with banks, the page is easy to identify from a hexadecimal address: it is
the fourth and third rightmost digits. For example, page \$1C consists of
\$1C00 -- \$1CFF. The address \$1CB2 is in page \$1C.

\subsection{How Addresses are Stored in Memory}

Computers often manipulate memory addresses as data, including storing
addresses in RAM or in registers. A 16-bit address is stored as two bytes,
and a 28-bit address is stored as four bytes (with the leftmost nibble set
to zero).

To determine the bytes for an address, take the hexadecimal digits and
group them in pairs, starting from the {\em rightmost} pair. Each pair of
digits is a byte value. For example, the address \$853FDA8 can be represented
by the bytes \$A8, \$FD, \$53, and \$08.

This ordering of the bytes from right to left is known as {\em Little Endian
byte order}. The MEGA65 stores multi-byte values using Little Endian byte
order by convention. The 45GS02 CPU, I/O registers, and other routines expect
addresses to be stored in this way.\footnote{There exist computers that use the
opposite ordering convention, known as {\em Big Endian} byte order. Most
microcomputers use the Little Endian convention.}

For example, to store the 28-bit address 853.FDA8, you need four bytes of
memory. To store this at memory locations 0.1800 -- 0.1803, you would
start by storing the rightmost byte (\$A8) in location 0.1800. The rest is
as follows:

\begin{center}
\begin{tabular}{cccc}
\hline
\multicolumn{1}{|c}{\huge\texttt{A8}} &
\multicolumn{1}{|c}{\huge\texttt{FD}} &
\multicolumn{1}{|c}{\huge\texttt{53}} &
\multicolumn{1}{|c|}{\huge\texttt{08}} \\
\hline
\rotatebox{90}{0.1800 } &
\rotatebox{90}{0.1801 } &
\rotatebox{90}{0.1802 } &
\rotatebox{90}{0.1803 } \\
\end{tabular}
\end{center}

When a 28-bit address is stored as four bytes, the upper four bits are unused.
There are a few cases where a 28-bit address is stored in a way that uses those
bits for something else. You must take care when writing such addresses that
the upper bits retain appropriate values. This is typically done with Boolean
logic to ``mask'' the upper bits.

The following BASIC example sets the variable \stw{B} to \$A9, then updates it
such that the upper nibble is preserved and the lower nibble is set to \$3.
Try to guess what this will print before running it.

\begin{screencode}
10 B = $A9
20 B = B AND $F0 OR $03
30 PRINT HEX$(B)
\end{screencode}

The Commodore 65 uses a 20-bit address space, so it sometimes only uses five
nibbles (five hexadecimal digits) to encode an address. In a few cases, the
MEGA65 extends this to 28 bits by storing a separate ``megabyte byte'' in
addition to a ``bank nibble'' and two bytes for the lower part of the address.
An example of this will be explained later in this chapter.

BASIC 65 includes commands for reading and writing 16-bit values as bytes
without having to convert to Little Endian explicitly. {\bf WPOKE addr,val}
sets {\bf addr} to the low byte of {\bf val} and {\bf addr+1} to the high byte.
Similarly, {\bf WPEEK(addr)} reads the bytes at {\bf addr} and {\bf addr+1} and
evaluates to the 16-bit number.

\begin{screencode}
WPOKE $1900,$FFD2
\end{screencode}

BASIC does not have dedicated commands for reading or writing 32-bit values,
but you can use the 16-bit commands to handle two bytes at a time. Take care to
use Little Endian order for each half:

\begin{screencode}
WPOKE $1802, $0853
WPOKE $1800, $FDA8
\end{screencode}


\newpage
\section{The 28-bit Address Space}
\label{sec:programming-with-memory-address-space}

The 28-bit addresses form an {\em address space} in the range 000.0000
-- FFF.FFFF. Only some addresses are assigned to memory cells. Others are assigned to I/O
registers and other specialty devices. The remaining addresses are unassigned, reserved
for future use by expansion hardware and future versions of the computer.

A MEGA65 (board revision R3A) and a DevKit (R3) have three major regions of
assigned address space:

\begin{itemize}
\item {\bf Chip RAM}, 384KB, in range 000.0000 -- 005.FFFF
\item {\bf Attic RAM}, 8MB, in range 800.0000 -- 87F.FFFF
\item {\bf I/O registers and specialty RAM}, in range F00.0000 -- FFF.FFFF
\end{itemize}

The remaining regions of the address space are reserved for future expansion
and future models. Nexys boards do not have Attic RAM. See
\bookvref{cha:memory-map} for details on the reserved regions.

Chip RAM is the primary working space for the MEGA65. It contains program code,
the MEGA65 ROM code, and space for variables and other data. It is called ``Chip
RAM'' because it can be accessed by the CPU running at full speed. BASIC and
kernel functions use portions of Chip RAM, and your BASIC and machine code
programs will make extensive use of it as well.

Attic RAM provides expanded memory for general use, with a few limitations.
Typically, a program uses Attic RAM via the MEGA65's Direct Memory Access (DMA) capability to
copy data between Attic RAM and Chip RAM (described later). Attic RAM cannot be used directly by the VIC
chip for graphics data, nor can it be used directly for audio sample playback
(``audio DMA''). The CPU can run code stored in Attic RAM, but it runs more
slowly than code stored in Chip RAM. In general, accessing Attic RAM is about
ten times slower than accessing Chip RAM.

The upper I/O register space is the permanent home for device registers, and
memory for the Hypervisor functions. Most programs access these features
indirectly using the memory configuration features described later in this
chapter, especially the VIC colour RAM, character ROM, and VIC and SID I/O
registers. See \bookvref{cha:memory-map} for a detailed list of upper I/O regions.


\newpage
\section{The Chip RAM Memory Map}

The following table summarizes how the MEGA65 kernel and BASIC use Chip RAM.

\setlength{\tabcolsep}{3pt}
\begin{longtable}{|L{1.5cm}|L{1.5cm}|p{6cm}|}
\hline
{\bf{Start}} & {\bf{End}} & {\bf{Description}} \\
\hline
\endfirsthead
\multicolumn{3}{l@{}}{\ldots continued}\\
\hline
{\bf{Start}} & {\bf{End}} & {\bf{Description}} \\
\endhead
\multicolumn{3}{l@{}}{continued \ldots}\\
\endfoot
\hline
\endlastfoot
\hline
\small 0.0000 & \small 0.0000 & \multicolumn{1}{p{6cm}|}{CPU I/O Port Data Direction}\\
\hline
\small 0.0001 & \small 0.0001 & \multicolumn{1}{p{6cm}|}{CPU I/O Port Data}\\
\hline
\small 0.0002 & \small 0.15FF & \multicolumn{1}{p{6cm}|}{Kernel variables and data}\\
\hline
\small 0.1600 & \small 0.1EFF & \multicolumn{1}{p{6cm}|}{Free for program use}\\
\hline
\small 0.1F00 & \small 0.1FFF & \multicolumn{1}{p{6cm}|}{BASIC bitmap graphics
base page}\\
\hline
\small 0.2000 & \small 0.F6FF & \multicolumn{1}{p{6cm}|}{BASIC: program text}\\
\hline
\small 0.F700 & \small 0.FEFF & \multicolumn{1}{p{6cm}|}{BASIC: scalar variables}\\
\hline
\small 0.FF00 & \small 0.FFFF & \multicolumn{1}{p{6cm}|}{Reserved}\\
\hline
\hline
\small 1.0000 & \small 1.1FFF & \multicolumn{1}{p{6cm}|}{DOS buffers and variables}\\
\hline
\small 1.2000 & \small 1.F6FF & \multicolumn{1}{p{6cm}|}{BASIC: arrays and strings}\\
\hline
\small 1.F700 & \small 1.F7FF & \multicolumn{1}{p{6cm}|}{Reserved}\\
\hline
\small 1.F800 & \small 1.FFFF & \multicolumn{1}{p{6cm}|}{Colour memory window}\\
\hline
\hline
\small 2.0000 & \small 2.FFFF & \multicolumn{1}{p{6cm}|}{ROM}\\
\hline
\small 3.0000 & \small 3.FFFF & \multicolumn{1}{p{6cm}|}{ROM}\\
\hline
\hline
\small 4.0000 & \small 4.FFFF & \multicolumn{1}{p{6cm}|}{BASIC bitmap graphics, utilities}\\
\hline
\small 5.0000 & \small 5.FFFF & \multicolumn{1}{p{6cm}|}{BASIC bitmap graphics, utilities}\\
\hline
\end{longtable}

The ROM banks are arranged in regions. Most programs don't need to access these
addresses directly.

\setlength{\tabcolsep}{3pt}
\begin{longtable}{|L{1.5cm}|L{1.5cm}|p{6cm}|}
\hline
{\bf{Start}} & {\bf{End}} & {\bf{Description}} \\
\hline
\endfirsthead
\multicolumn{3}{l@{}}{\ldots continued}\\
\hline
{\bf{Start}} & {\bf{End}} & {\bf{Description}} \\
\endhead
\multicolumn{3}{l@{}}{continued \ldots}\\
\endfoot
\hline
\endlastfoot
\hline
\small 2.0000 & \small 2.3FFF & \multicolumn{1}{p{6cm}|}{DOS}\\
\hline
\small 2.4000 & \small 2.8FFF & \multicolumn{1}{p{6cm}|}{Reserved}\\
\hline
\small 2.9000 & \small 2.9FFF & \multicolumn{1}{p{6cm}|}{Character set A}\\
\hline
\small 2.A000 & \small 2.BFFF & \multicolumn{1}{p{6cm}|}{C64 BASIC}\\
\hline
\small 2.C000 & \small 2.CFFF & \multicolumn{1}{p{6cm}|}{Interface}\\
\hline
\small 2.D000 & \small 2.DFFF & \multicolumn{1}{p{6cm}|}{C64 character set
(C)}\\
\hline
\small 2.E000 & \small 2.FFFF & \multicolumn{1}{p{6cm}|}{C64 BASIC and kernel}\\
\hline
\hline
\small 3.0000 & \small 3.1FFF & \multicolumn{1}{p{6cm}|}{Monitor}\\
\hline
\small 3.2000 & \small 3.7FFF & \multicolumn{1}{p{6cm}|}{C65 BASIC}\\
\hline
\small 3.8000 & \small 3.BFFF & \multicolumn{1}{p{6cm}|}{C65 BASIC graphics}\\
\hline
\small 3.C000 & \small 3.CFFF & \multicolumn{1}{p{6cm}|}{Reserved}\\
\hline
\small 3.D000 & \small 3.DFFF & \multicolumn{1}{p{6cm}|}{Character set B}\\
\hline
\small 3.E000 & \small 3.FFFF & \multicolumn{1}{p{6cm}|}{C65 kernel}\\
\hline
\end{longtable}

\subsection{How Programs Use Chip RAM}

There are two common kinds of MEGA65 programs: programs written entirely in BASIC,
and programs written entirely in machine code (via assembly language, or a
compiled language such as C or Rust).

A typical BASIC 65 program never accesses memory directly.\footnote{Commodore
64 BASIC programmers are accustomed to using {\bf POKE} and {\bf PEEK} to
access memory and I/O registers for most functions like graphics, sound, and
input devices. BASIC 65 provides dedicated commands for these functions.} It
uses BASIC commands and language features to manipulate variables, arrays,
strings, sprites, and bitmap graphics. The BASIC system uses all of Chip RAM to
support these features, and manages memory configuration to access registers on
behalf of the program. This requires that most of the ROM loaded and protected
by the Hypervisor remain in place.

A BASIC program that does not use the bitmap graphics system can repurpose the
related memory regions for other purposes. This includes the region 0.1F00 --
0.1FFF, and all of banks 4 and 5. If you intend to use bitmap graphics but
need some spare Chip RAM for another purpose, you can use the {\bf
MEM}\index{BASIC Commands!MEM} command to reserve blocks in banks 4 and 5. The
graphics system knows to avoid regions reserved by {\bf MEM}, at the expense
of needing to use fewer screens, lower resolution, or lower colour bit depth.
For a description of the {\bf MEM} command, see \pageref{BASIC 65 Commands!MEM}.

Some immediate mode BASIC tools, such as {\bf RENUMBER}, make temporary use of
memory in banks 4 and 5. This does not affect a program that initialises its
memory when it starts, but it may interfere with interactive tasks
performed at the \screentext{READY.} prompt that need that memory preserved.

A common technique for using memory from a BASIC program is to take advantage
of the unused ``program text'' memory after the end of the BASIC program data
in 0.2000 -- 0.F6FF. If your program needs to do this, note that some BASIC 65
graphics commands use some of the memory immediately following the end of the
program as temporary storage. Depending on the length of your program, it is
usually safe to choose a high address range, such as 0.C000-0.F6FF, for
program use.

\begin{center}
\begin{tabular}{m{0.14cm}m{0.06cm}m{1.45cm}m{0.21cm}m{1.4cm}m{0.1cm}m{0.1cm}m{3.3cm}m{3.3cm}l}
\cline{1-1}\cline{3-9}
\multicolumn{1}{|l|}{\rotatebox{90}{Kernel}} & \multicolumn{1}{l}{\ldots} &
\multicolumn{1}{|l}{\rotatebox{90}{BASIC}} & \multicolumn{1}{|l}{\rotatebox{90}{DOS}} &
\multicolumn{1}{|l}{\rotatebox{90}{BASIC}} & \multicolumn{1}{|l}{\rotatebox{90}{Res.}} &
\multicolumn{1}{|l}{\rotatebox{90}{Colour}} & \multicolumn{1}{|l}{\rotatebox{90}{ROM}} &
\multicolumn{1}{|l|}{\rotatebox{90}{BASIC Gfx }} & \\
\cline{1-1}\cline{3-9}
\rotatebox{90}{\small 0.0000} & \rotatebox{90}{\small 0.1600} &
\rotatebox{90}{\small 0.2000} & \rotatebox{90}{\small 1.0000} &
\rotatebox{90}{\small 1.2000} & \rotatebox{90}{\small 1.F700} &
\rotatebox{90}{\small 1.F800} & \rotatebox{90}{\small 2.0000} &
\rotatebox{90}{\small 4.0000} & \rotatebox{90}{\small 5.FFFF} \\
\end{tabular}
\end{center}

A typical program written in machine code never uses the BASIC system at all,
and instead manages its own memory. It might use a short BASIC program to
start the program, but never returns control to the system after the program
is invoked. Such a program is free to use any region of Chip RAM that would
otherwise be used by BASIC.

If the program calls the kernel, such as for disk functions, the program must
avoid the areas of Chip RAM reserved for kernel use. It can do whatever it likes
with the rest.

\begin{center}
\begin{tabular}{m{0.14cm}m{0.06cm}m{1.45cm}m{0.21cm}m{1.4cm}m{0.1cm}m{0.1cm}m{3.3cm}m{3.3cm}l}
\cline{1-1}\cline{4-4}\cline{6-8}
\multicolumn{1}{|l|}{\rotatebox{90}{Kernel}} & \multicolumn{1}{l}{\ldots} &
\multicolumn{1}{l}{\ldots} & \multicolumn{1}{|l|}{\rotatebox{90}{DOS}} &
\multicolumn{1}{l}{\ldots} & \multicolumn{1}{|l}{\rotatebox{90}{Res.}} &
\multicolumn{1}{|l}{\rotatebox{90}{Colour }} & \multicolumn{1}{|l|}{\rotatebox{90}{ROM}} &
\multicolumn{1}{l}{\ldots} & \\
\cline{1-1}\cline{4-4}\cline{6-8}
\rotatebox{90}{\small 0.0000} & \rotatebox{90}{\small 0.1600} &
\rotatebox{90}{\small 0.2000} & \rotatebox{90}{\small 1.0000} &
\rotatebox{90}{\small 1.2000} & \rotatebox{90}{\small 1.F700} &
\rotatebox{90}{\small 1.F800} & \rotatebox{90}{\small 2.0000} &
\rotatebox{90}{\small 4.0000} & \rotatebox{90}{\small 5.FFFF} \\
\end{tabular}
\end{center}

If the program never needs to call the kernel, it can unlock and remove the
ROM entirely, and claim every byte of the 384KB Chip RAM for its own purposes.
Using a technique discussed later, if you don't need all 32KB of the MEGA65's
colour RAM, you can repurpose the ``colour RAM window'' at 1.F800 as regular
chip RAM.

\subsection{The Memory Map Contract}

The MEGA65 project adopted the ROM code from Commodore's unfinished draft for
the C65, and invested time and effort into finishing incomplete features and
fixing bugs. This project is on-going, and the MEGA65 team expects to release
new versions of the ROM in the future.

This makes writing programs with the MEGA65 ROM characteristically different
to writing programs with the Commodore 64 ROM, which is etched in silicon
and is no longer under development. It is important to write code that
conforms to documented behaviors described in these manuals---and not rely on
undocumented behaviors that might change in a future version.

It is a goal of the project to keep the Chip RAM memory map consistent with
the table above across future versions of the MEGA65 ROM. To maintain future
compatibility, programs should avoid relying on ``reserved'' regions as if
they were free space. If a new mode or feature is introduced that changes the
memory map, it will be something a program must request, such that older
programs can expect the original memory map.

Within each block of the memory map, only some addresses are guaranteed to
remain constant between ROM versions. For example, a machine code program can
invoke kernel routines by calling addresses in a ``jump table,'' a set of
fixed addresses in ROM guaranteed to remain the same in future
versions.\footnote{The jump table is not yet described in this book. The C65
used the same jump table as the C128, and so far the MEGA65 ROM has not
extended it. If a ROM revision expands this table, existing entries will
remain constant. See the MEGA65 Wiki for a list:
\url{https://mega65.atlassian.net/}}

Programs are not expected to read or modify kernel data directly, so these
locations are not documented. Some memory locations, such as the location of
screen memory, can be determined from (and adjusted by) registers. (See
\bookvref{cha:viciv}.)

The region of Chip RAM in 0.1600 -- 0.1EFF is guaranteed to be unused by the
MEGA65 kernel and BASIC. 0.1F00 -- 0.1FFF is also available if you do not use the
BASIC bitmap graphics system. This region is exposed as RAM in the default memory
configuration, so it is useful for storing machine code and data that can be
accessed from a BASIC program.


\newpage
\section{Using Memory from BASIC}

BASIC 65 includes several commands for accessing and manipulating memory, and
for calling machine code subroutines:

\begin{itemize}
\item {\bf POKE address,value} : stores a value at an address
\item {\bf WPOKE address,value} : stores a 16-bit value starting at an address
\item {\bf PEEK(address)} : reads a value at an address
\item {\bf WPEEK(address)} : reads a 16-bit value starting at an address
\item {\bf BLOAD filename [,args...]} : loads data into memory from a file
\item {\bf BSAVE filename, P start TO P end [,args...]} : saves a region of memory to a file
\item {\bf SYS address [,registers...]} : calls a machine code subroutine
stored at an address
\item {\bf EDMA ...} : transforms or copies large regions of memory quickly
using hardware accelleration
\end{itemize}

The {\bf BOOT} and {\bf WAIT} commands also operate on memory in some fashion.
See the BASIC 65 Command Reference for more information about these commands.

\subsection{BASIC Address Remapping}

You may have noticed that the addresses from the BASIC {\bf POKE} commands
described at the beginning of this chapter all refer to areas of Chip
RAM described in the memory map as either ``Free for program use,'' or ``BASIC:
program text.'' However, they seem to behave differently:

\begin{itemize}
\item \stw{POKE \$1800,127} behaves like RAM: it sets the memory at the address,
and the memory can be read with \stw{PRINT PEEK(\$1800)}.
\item \stw{POKE \$2010,63} behaves like ROM: attempting to set the memory does nothing,
and reading the memory with \stw{PRINT PEEK(\$2010)} returns the original ROM value.
\item \stw{POKE \$D020,7} behaves like an I/O register: it changes the colour of
the border immediately. While this value can be read with \stw{PRINT
PEEK(\$D020)}, it is not useful as memory because the border colour changes with the value.
\end{itemize}

\newpage

BASIC 65 commands and functions treat addresses in the range \$0000 -- \$FFFF
according to special remapping rules, also known as {\em banking}. In the
default mode, BASIC 65 remaps these addresses as follows:

\begin{center}
\begin{tabular}{|c|c|c|}
\hline
{\bf 16-bit address block} & {\bf Mapped address block} & {\bf Type} \\
\hline
\$0000 -- \$1FFF & 0.0000 -- 0.1FFF & Chip RAM \\
\hline
\$2000 -- \$3FFF & 3.2000 -- 3.3FFF & ROM \\
\hline
\$4000 -- \$5FFF & 3.4000 -- 3.5FFF & ROM \\
\hline
\$6000 -- \$7FFF & 3.6000 -- 3.7FFF & ROM \\
\hline
\$8000 -- \$9FFF & 3.8000 -- 3.9FFF & ROM \\
\hline
\$A000 -- \$BFFF & 3.A000 -- 3.BFFF & ROM \\
\hline
\$C000 -- \$CFFF & 2.C000 -- 2.CFFF & ROM \\
\hline
\$D000 -- \$DFFF & FFD.2000 -- FFD.2FFF & MEGA65 I/O\\
\hline
\$E000 -- \$FFFF & 3.E000 -- 3.FFFF & ROM \\
\hline
\end{tabular}
\end{center}

For example, accessing I/O registers via \$D000 -- \$DFFF actually accesses
the I/O registers in the upper region of the address space. These two BASIC
commands are equivalent:

\begin{screencode}
POKE $D020,7

POKE $FFD2020,7
\end{screencode}

The BASIC system itself uses Chip RAM directly in accordance with the Chip RAM
memory map. For example, BASIC stores program text starting at 0.2000 in Chip
RAM, even though the BASIC function \stw{PEEK(\$2000)} reads from 3.2000 using
the default mapping mode.

BASIC 65's remapping of addresses only applies to addresses in the range \$0000
-- \$FFFF. When you use an address \$10000 or higher with {\bf POKE} or
{\bf PEEK()}, it accesses that Chip RAM address directly. {\bf SYS} also
supports long addresses, with limitations (discussed later).

You can control BASIC address remapping using the {\bf BANK}\index{BASIC 65
Commands!BANK} command. This command accepts a bank number as an argument, in the range 0
-- 5. Subsequent uses of addresses in the range \$0000 -- \$FFFF are
interpreted as being offsets in that bank. For example, to write the value 63
to address 0.2010 in Chip RAM:

\begin{screencode}
BANK 0
POKE $2010,63
\end{screencode}

The {\bf BANK} mechanism can only be used with banks 0 -- 5. It cannot be used
to reach Attic RAM or upper I/O registers.

To re-enable BASIC address remapping, execute {\bf BANK} with the argument 128.
This does not refer to a bank in memory. Instead, it tells the {\bf BANK}
command to re-enable mapped addresses.

\begin{screencode}
BANK 128
POKE $D020,9
\end{screencode}

{\bf BANK} affects every BASIC command and function that accepts an address as
a parameter. This includes {\bf PEEK}\index{BASIC 65 Functions!PEEK},
{\bf POKE}\index{BASIC 65 Commands!POKE},
{\bf WAIT}\index{BASIC 65 Commands!WAIT},
{\bf BOOT}\index{BASIC 65 Commands!BOOT},
{\bf BSAVE}\index{BASIC 65 Commands!BSAVE}, and {\bf BLOAD}\index{BASIC 65
Commands!BLOAD}. {\bf SYS} also has special behavior with
regards to {\bf BANK}.

\subsection{BANK and SYS}

The {\bf SYS}\index{BASIC 65 Commands!SYS} command executes a machine code
subroutine at a given address. It handles the address differently from other
BASIC commands, and comes with some limitations. Because {\bf SYS} transfers
control to a custom routine, it needs to take extra precautions to preserve
the system's access to memory used to support BASIC.

Specifically, {\bf SYS} only the {\bf BANK} setting for addresses in the range
\$2000 -- \$7FFF. Other addressees always go to either bank 0 RAM, KERNAL ROM,
or I/O registers, as follows:

\begin{tabular}{|l|l|}
\hline
{\bf Address range} & {\bf SYS destination} \\
\hline
\$0000 -- \$1FFF & RAM in bank 0 \\
\hline
\$2000 -- \$7FFF & High word of address or {\bf BANK} setting \\
\hline
\$8000 -- \$BFFF & RAM in bank 0 \\
\hline
\$C000 -- \$CFFF & Character ROM, or bank 0 (ROMC bit of \$D030) \\
\hline
\$D000 -- \$DFFF & I/O registers, or bank 0 (C64-style banking register \$01) \\
\hline
\$E000 -- \$FFFF & KERNAL ROM in bank 3 \\
\hline
\end{tabular}

{\bf SYS} ignores bit 7 of the {\bf BANK} setting. \stw{BANK 128} is
equivalent to \stw{BANK 0} for {\bf SYS}. This differs from {\bf PEEK},
{\bf POKE}, et al., which see ROM code from \$2000 -- \$BFFF with
\stw{BANK 128}.

\begin{screencode}
BANK 128
SYS $1800        : rem Calls subroutine at 0.1800
SYS $7000        : rem Calls subroutine at 0.7000
SYS $FFD2,65     : rem Calls kernel ROM routine (with an argument)

BANK 1
SYS $1800        : rem Calls subroutine in bank 0 at 0.1800
SYS $7000        : rem Calls subroutine in bank 1 at 1.7000
SYS $FFD2,65     : rem Calls kernel ROM routine at 3.FFD2 (with an argument)
\end{screencode}

{\bf SYS} can accept an address larger than \$FFFF that refers to a location
in banks 1 -- 5. However, these addresses have the same limitations as when
using {\bf BANK}. Only offsets \$2000 -- \$7FFF within the bank actually refer
to the memory of that bank, such as address \$47000 in bank 4. For other
offsets, the {\bf SYS} command behaves according to the table (RAM in bank 0,
ROM, or I/O registers).

This restriction produces the counterintuitive behavior that, for {\bf SYS}, a
long address doesn't necessarily refer to the memory at that address, and
multiple addresses may refer to the same location in memory. \stw{SYS \$11800}
and \stw{SYS \$41800} both behave like \stw{SYS \$1800}, which calls a
subroutine at address 0.1800 in bank 0.

\begin{screencode}
SYS $11800       : rem Calls subroutine in bank 0 at 0.1800
SYS $41800       : rem Calls subroutine in bank 0 at 0.1800
SYS $51800       : rem Calls subroutine in bank 0 at 0.1800

SYS $17000       : rem Calls subroutine in bank 1 at 1.7000
SYS $47000       : rem Calls subroutine in bank 4 at 4.7000
SYS $57000       : rem Calls subroutine in bank 5 at 5.7000

SYS $1FFD2,65    : rem Calls kernel ROM routine at 3.FFD2 (with an argument)
\end{screencode}

The {\bf SYS} command cannot access every address. Address offsets \$C000 --
\$FFFF are hidden by ROM and I/O registers in every bank, and address offsets
\$0000 -- \$1FFF and \$8000 -- \$BFFF can only access bank 0 and are hidden in
banks 1 -- 5.

{\bf SYS} is limited to banks 0 -- 5. It cannot access Attic RAM or upper
memory addresses, even when using long addresses.

Despite these limitations, {\bf SYS} is quite powerful. It is common for a
machine code program to start with a short BASIC program, so the user can type
{\bf RUN} after loading it. This program calls {\bf SYS} to start the machine
code stored in memory after the end of the BASIC text.

\begin{screencode}
10 SYS $200F
\end{screencode}

In early versions of the MEGA65 ROM, it was necessary to set {\bf BANK} to 0
before calling {\bf SYS} in this way. This is a best practice anyway, so that
the program can start even if the user changed the {\bf BANK} setting:

\begin{screencode}
10 BANK 0:SYS $2014
\end{screencode}

You can also use {\bf SYS} to augment a BASIC program with machine code
subroutines. A convenient location for these subroutines is in the \$1600 --
\$1FFF range of bank 0. {\bf SYS} can access this location regardless of the
current {\bf BANK} setting.


\newpage
\section{Using Memory from Machine Code}

Every 45GS02 machine code instruction that operates on memory has one or more
variants that can locate memory via {\em addressing modes}. One of these
addressing modes operates on full 28-bit addresses, reading the full address
from four consecutive bytes of memory in Little Endian order.

For speed, space efficiency, and backwards compatibility with processors
earlier in the MOS 6502 lineage, most 45GS02 addressing modes operate on 16-bit
addresses. The 45GS02 calculates the complete 28-bit address from a special
purpose register known as {\em the MAP register}. Setting this register
involves executing a sequence of special-purpose CPU instructions.

With the MEGA65, three other mechanisms affect 16-bit memory address
translation. The Commodore 64 ROM and I/O banking mechanism controlled by the
CPU registers at addresses 0.0000 and 0.0001 is supported for backwards
compatibility and access to I/O registers. The Commodore 65 has another ROM
banking mechanism using a register at address \$D030. Finally, if a cartridge
is inserted in the expansion port, additional rules expose cartridge data
lines at specific 16-bit address banks.

Programs written for the MEGA65 can rely entirely on the MAP register for
accessing memory. The C64, C65, and cartridge memory mapping mechanisms are
included in the MEGA65 primarily for compatibility purposes.

This section describes accessing memory with 45GS02 addressing modes and the
MAP register. The other mapping mechanisms are discussed briefly later in the
chapter. For complete documentation on addressing modes, see
\bookvref{sec:addressing-modes}. For a detailed explantion of the banking
mechanisms, see \bookvref{cha:45gs02}.

\subsection{The MAP Register}

The MAP register remaps 8KB (\$2000) blocks of the 16-bit address space using an {\em
offset}, an amount that's added to the 16-bit address to determine the final
address. The offset must be a multiple of 256 bytes (\$100).

For example, if the block \$2000 -- \$3FFF is offset by \$45200, then accessing
the 16-bit address \$335F will access physical address \$4855F (in bank 4).

\begin{center}
\begin{tabular}{ccccc}
\$335F & + & \$45200 & = & \$4855F \\
\vtop{\hbox{16-bit}\hbox{address}} & & offset & &
\vtop{\hbox{actual}\hbox{address}} \\
\end{tabular}
\end{center}

The 4510 CPU from the Commodore 65, on which the MEGA65's 45GS02 is based,
supports offsets up to \$FFF00. This allows any 8KB block to be remapped anywhere in
Chip RAM on a 256-byte boundary, including possible future RAM expansions up to
1MB. (The 45GS02 also has a way to set larger offsets, discussed later.)

The MAP register can store two offsets at a time, one for blocks in the range
\$0000 -- \$7FFF (``MAPLO''), and one for blocks in the range \$8000 -- \$FFFF
(``MAPHI''). It also stores a selection flag for each 8KB block that says
whether the offset should be applied to that block.

Each 8KB address block is either ``unmapped'' or uses the MAP offset for its
half (MAPLO or MAPHI). All mapped address blocks in the same half must share
the same offset.

\begin{center}
\begin{tabular}{c|c|c|c|c|l}
MAPLO: & \$6000 -- \$7FFF & \$4000 -- \$5FFF & \$2000 -- \$3FFF & \$0000 --
\$1FFF & \\
& 0 & 0 & {\bf 1} & 0 & = \$2 \\
\hline
MAPHI: & \$E000 -- \$FFFF & \$C000 -- \$DFFF & \$A000 -- \$BFFF & \$8000 --
\$9FFF & \\
& {\bf 1} & 0 & {\bf 1} & {\bf 1} & = \$B \\
\end{tabular}
\end{center}

The selection flags and the offset for MAPLO and MAPHI are encoded altogether
as four bytes, two for MAPLO and two for MAPHI. The first nibble contains the
selection flags for the four blocks of the region. Notice the order of the
flag bits: the higher bits correspond to the higher address blocks.

For example, to select just the \$2000 -- \$3FFF block, set the MAPLO selection
flag to binary ``0 0 1 0,'' which is the hexadecimal digit \$2.

The remaining three nibbles are the upper three hex digits of the
offset. To use the offset \$45200 for the selected MAPLO blocks, set the
remaining bits of MAPLO to \$452. For this example, the complete MAPLO value is
\$2452.

\begin{center}
\begin{tabular}{cc|cc}
\multicolumn{2}{c}{MAPLO} & \multicolumn{2}{c}{MAPHI} \\
X & A & Z & Y \\
\hline
\$s\textsubscript{lo} o\textsubscript{1} &
\$o\textsubscript{2} o\textsubscript{3} &
\$s\textsubscript{hi} o\textsubscript{1} &
\$o\textsubscript{2} o\textsubscript{3} \\
\hline
\$24 & \$52 & \$B3 & \$00 \\
\end{tabular}
\end{center}

To set the MAP register, load the encoded bytes for MAPLO and MAPHI into the four CPU
registers X, A, Z, and Y, then call the {\bf MAP} and {\bf EOM} (``end of MAP'')
instructions. Notice the order of assignment of the registers. To set MAPLO, load X
with the selection flags and the upper nibble of the offset, and load A with the lower
nibbles of the offset. To set MAPHI, load Z and Y accordingly.

\begin{screencode}
LDA #$52   ; MAPLO = select $2 offset $45200
LDX #$24
LDY #$00   ; MAPHI = select $B offset $30000
LDZ #$B3
MAP
EOM
\end{screencode}

Some assemblers, including the one built into the MEGA65 monitor, do not know
the EOM instruction. You can substitute the NOP instruction for EOM, which
uses the same encoding. Assemblers that support 65CE02 instructions use the
AUG instruction for MAP.

The MAP setting stays active until it is changed. Be aware that kernel
routines change MAP for their own purposes. If your program depends on a MAP
setting, it must restore that setting after calling a kernel routine.

You can examine the value of the MAP register using the Matrix Mode debugger, a
serial console provided by the MEGA65 Hypervisor. To access the Matrix Mode
debugger on the MEGA65, hold down \megasymbolkey and press \specialkey{TAB}.
(Press \megasymbolkey + \specialkey{TAB} again to exit.) You can also access
this console over a UART serial or JTAG connection with the appropriate
hardware. While in Matrix Mode, use the {\bf R} command, or just press
\specialkey{RETURN}, to view all registers, including MAPHI and MAPLO.\footnote{For more
information about the Matrix Mode debugger, see \bookvref{sec:matrix-mode}.}

To see this in action, try the following:

\begin{enumerate}
\item Start the MEGA65 monitor with the {\bf MONITOR} command.
\item Assemble the following machine language program at address \$1800 by
entering the first line as: \stw{A1800 LDA \#\$52} The monitor will calculate
the subsequent addresses for each line. Press \specialkey{RETURN} without an
instruction after the last line.

\begin{screencode}
LDA #$52
LDX #$24
LDY #$00
LDZ #$B3
MAP
NOP
JMP $180A
\end{screencode}

\item Call this routine with the monitor's ``jump'' command: \stw{J1800} The
MEGA65 will enter an infinite loop and not return from the call, appearing to hang.
\item Open the Matrix Mode debugger: hold \megasymbolkey, press \specialkey{TAB}.
\item Press \specialkey{RETURN} to print a fresh set of registers. Notice the
MAPH and MAPL values.
\end{enumerate}

Be sure to set memory configuration correctly to avoid interrupt vectors jumping to
invalid code. In this example, MAPHI is set to \$B300 to keep kernel ROM mapped
into the higher banks. This is required to keep the default interrupt vectors
pointing at valid kernel code. In general, use the SEI and CLI instructions to
disable and re-enable interrupts while adjusting interrupt vectors and related
memory configuration.

\subsection{Using 28-bit Addresses in Machine Code}

The 45GS02 CPU includes an addressing mode for accessing 28-bit addresses
directly, without 16-bit address translation. While technically this mode
requires more CPU cycles to execute, the convenience often outweighs this cost
when accessing single upper-memory addresses, or when accessing memory via
pointers, compared to using the MAP register. They also bypass banking and
remapping, allowing a program to preserve a MAP register value that might be
used by other code.

The Base-Page Quad Indirect Z-Indexed Addressing Mode accesses a 28-bit address
stored as four bytes on the base page. This is an extension of the Base-Page
Indirect Z-Indexed Mode that accesses a 16-bit address stored as two bytes.

In an assembler that supports 45GS02 extensions, such as the ACME assembler, the
notation for Quad Indirect mode is to use square brackets around the first base
page address where the address is stored:

\begin{screencode}
; Select base page $1600.
; (This preserves compatibility with the kernel, which
; reserves base page $0000.)
LDA #$16
TAB

; Store address 810.4AB0 at base page address $40-$43.
; (Remember to use Little Endian order.)
LDA #$B0
STA $40
LDA #$4A
STA $41
LDA #$10
STA $42
LDA #$08  ; (The leftmost digit is zero.)
STA $43

; Load the accumulator with the byte at 810.4AB0 (with a Z-index of zero).
LDZ #$0
LDA [$40],Z
\end{screencode}

This is encoded similarly to regular Base-Page Indirect mode preceded by a NOP
instruction. If your assembler does not support the square bracket notation,
you can write it this way, with parentheses instead of brackets:

\begin{screencode}
LDZ #$0
NOP
LDA ($40),Z
\end{screencode}

See \bookvref{cha:45gs02} for more information about addressing modes.

This is primarily how BASIC's BANK system works. If BANK is set to 0 -- 5, or
a program accesses an address greater than \$FFFF, commands like POKE and PEEK
use Base-Page Quad Indirect Z-Indexed Addressing Mode, writing the full
address to a location reserved in the kernel's base page. This bypasses
address translation without having to change the memory configuration. (The
SYS command does something more complex, which is why it comes with unusual
limitations.)


\newpage
\section{VIC-IV Video Memory}

The VIC-IV video controller reads memory to determine what to display
on the screen, including characters of text, bitmap graphics, sprites, and
colour data. A program changes the display by updating these memory regions.
The memory locations used for these purposes can be adjusted by the program using
I/O registers.

The VIC-IV can only access Chip RAM. Technically, it is capable of accessing
the first 16MB of the address space. Existing models of MEGA65 provide 384KB of
Chip RAM, with potential for future expansion.

This section describes several VIC-IV features that relate to the memory
system. The following descriptions do not apply to VIC-II compatibility mode.
For a complete description of the VIC-IV, see \bookvref{cha:viciv}.

\subsection{Locating Screen Memory}

The VIC-IV reads the content of the screen (characters or pixels) from a
contiguous region of memory. The starting location of this memory is controlled
by registers. The content and size of screen memory depend on the display mode.

The screen memory start address is stored as 28 bits, even though only 20 bits
are required to store a Chip RAM address. The 28-bit address is stored across
four bytes \$D060 -- \$D063. The highest nibble of the address is stored in
the lower nibble of \$D063, which will always be zero for addresses up to 005.FFFF.

For example, to set screen memory to start at 4.26F0 in bank 4 of Chip RAM:

\begin{center}
\begin{tabular}{|c|c|c|c|}
\hline
SCRNPTRLSB & SCRNPTRMSB & SCRNPTRBNK & SCRNPTRMB \\
\$D060 & \$D061 & \$D062 & \$D063 (low 4) \\
\hline
\$F0 & \$26 & \$04 & \$0 \\
\hline
\end{tabular}
\end{center}

\begin{screencode}
LDA #$F0
STA $D060
LDA #$26
STA $D061
LDA #$04
STA $D062

; For completeness, clear bits 0-3 of SCRNPTRMB. The MEGA65 does not have
; VIC-accessible memory beyond 005.FFFF, so this is not necessary if the previous
; value pointed to a valid address.
LDA $D063
AND #$F0   ; clear bits 0-3
STA $D063
\end{screencode}

If you experiment with this in the MEGA65 monitor, note that the BASIC editor
will not automatically use the new screen location after you change it. You can
type blindly to set memory and see that it's using the new location. Press
\specialkey{RUN STOP} + \specialkey{RESTORE} to reset.

If your program writes to screen memory without initializing this register, be
sure to read the location of screen memory from the register. Do not assume
its initial value. The 80x50 display mode that the user can select by pressing
\specialkey{ESC} \specialkey{5} uses a different location for screen memory
than the default 80x25 display mode.

\subsection{Locating Character Data}

When using a character display mode, the VIC interprets screen data as
character screen codes. The pixel data for each character is read from a
character set stored in memory. The start address of the selected character set
is controlled by registers.

The size and format of character data depends
on the character display mode. The default text mode reads character data as
eight bytes per character, one line of monochrome pixels per byte, for a total
of 512 possible screen codes (4KB).

The character set start address is stored as three bytes in \$D068 -- \$D06A.
For example, to select the character set at 3.D000:

\begin{center}
\begin{tabular}{|c|c|c|}
\hline
CHARPTRLSB & CHARPTRMSB & CHARPTRBNK \\
\$D068 & \$D069 & \$D06A \\
\hline
\$00 & \$D0 & \$03 \\
\hline
\end{tabular}
\end{center}

The ROM data includes three monochrome character sets (used by the BASIC {\bf
FONT} command):

\begin{itemize}
\item 2.9000: Character set A, the C64 character set with lower ASCII characters
\item 3.D000: Character set B, a stylized ASCII terminal-like set
\item 2.D000: Character set C, the C64 PETSCII character set (the default)
\end{itemize}

Your program can customize any of the built-in character sets by copying the
4KB region to RAM and setting the CHARPTR registers accordingly.

The VIC-IV character generator has another way to store character glyphs
internally. When CHARPTR is set to the special value \$001000, the VIC-IV uses
4KB of internal character memory that stores the default PETSCII character set. This
allows you to repurpose all three character set ROM regions without losing access
to the PETSCII character set.

\subsection{Locating Sprite Memory}

The VIC-IV supports eight hardware sprites. The image data for these sprites is
referred to by a set of sprite pointers. Unlike the Commodore 64's VIC-II, the
VIC-IV supports relocating the sprite pointers and the image data to anywhere
in Chip RAM. The VIC-IV maintains some VIC-II compatibility by default, so
additional register lines are needed to fully enable relocation.

The location of the sprite pointers is stored in registers at \$D06C -- \$D06E.
Bit 7 of \$D06E tells the VIC-IV that the sprite pointers are
16 bits (two bytes) wide. If not set, the VIC-IV assumes 8-bit (one byte)
values for the pointers.

A sprite pointer value is the address of the sprite's image data divided by 64.
This unusual property originally allowed the VIC-II to locate sprite data in a
16KB range using a single byte value. The VIC-IV maintains this for
compatibility, and also allows this value to be 16 bits wide. One consequence
of this design is that the address of a sprite image must be ``aligned'' to a
64-byte region on a page: \$00, \$40, \$80, or \$C0.

For example, to relocate sprite pointers to 0.3200 and enable 16-bit pointer
values:

\begin{center}
\begin{tabular}{|c|c|c|c|}
\hline
SPRPTRADRLSB & SPRPTRADRMSB & SPRPTRBNK & SPRPTR16 \\
\$D06C & \$D06D & \$D06E (0-6) & \$D06E (7) \\
\hline
\$00 & \$32 & \$00 & +\$80 \\
\hline
\end{tabular}
\end{center}

\begin{screencode}
LDA #$00
STA $D06C
LDA #$32
STA $D06D
LDA #$80
STA $D06E
\end{screencode}

To set the first sprite to use image data at address 4.3100, set the 16-bit
pointer to 4.3100 / 64 = \$10C4 (in Little Endian order):

\begin{screencode}
LDA #$C4
STA $3200
LDA #$10
STA $3201
\end{screencode}

Notice that it is convenient to locate the sprite pointers in bank 0 to make
them easy to update. The image data itself can live in bank 4 comfortably.

\subsection{Colour Memory}

The VIC-IV determines a complete screen image from one set of character or pixel
data in screen memory, and a separate set of colour data in colour memory. For
example, in the 80x25 text display mode, the first 2,000 bytes of screen memory
contain the screen codes for the characters, and the first 2,000 bytes of
colour memory are the colours for those characters.

The VIC-IV has 64KB of dedicated memory for colour data, capable of supporting
all of the possible screen modes. This memory can be accessed directly using
the upper memory addresses FF8.0000 -- FF8.FFFF.

% TODO: confirm that R3 boards have 64KB of colour RAM; some places only cite 32KB

For convenience, the first 2KB of colour memory is
also exposed at addresses 1.F800 -- 1.FFFF. This window of addresses in bank 1
cannot be disabled by memory configuration: it always accesses the first 2KB of
colour memory.

For backwards compatibility, the first 1KB of this window is also banked
into \$D800 -- \$DBFF along with the I/O registers. Bit 0 of the \$D030 VIC-III
banking register (signal name ``CRAM2K'') extends this to the full 2KB, as \$D800
-- \$DFFF.

The VIC-IV start address of colour memory is configurable by a register, as an
offset into the FF8.0000 -- FF8.FFFF colour memory region. If you do not need
all 64KB of colour memory for your display, you can offset the start address
and use the beginning of colour memory for other purposes.

For example, to offset the start of colour memory by 2KB, set the register to
\$0800 (in Little Endian order):

\begin{center}
\begin{tabular}{|c|c|}
\hline
COLPTRLSB & COLPTRMSB \\
\$D064 & \$D065 \\
\hline
\$00 & \$08 \\
\hline
\end{tabular}
\end{center}

Offsetting the start address does {\em not} offset the colour memory window at 1.F800
{\em or} the colour memory banking at \$D800. If you offset the start of colour
memory by 2KB, then both the window and the I/O bank will use the 2KB region
unseen by the VIC-IV. This allows you to repurpose the colour memory window as
general purpose RAM.

This BASIC program sets the colour memory offset to 2KB (\$0800), then
demonstrates how the value at \$FF80000 is also visible at \$1F800 and \$D800.
Finally, it changes the colour of the first character by setting \$FF80800.

\begin{screencode}
10 POKE $D064,0:POKE $D065,8
20 PRINT CHR$(5);CHR$(147);"HI THERE."
30 POKE $FF80000,$BB
40 PRINT HEX$(PEEK($FF80000))
50 PRINT HEX$(PEEK($1F800))
60 PRINT HEX$(PEEK($D800))
70 POKE $FF80800,4
\end{screencode}


\newpage
\section{Large Memory Operations with Direct Memory Access}

In early microcomputers, all memory operations are performed by the CPU. Large
memory operations, such as copying a block of memory from one location to
another, are time consuming and tend to pause user interaction with the program. This
limits programs to small memory operations during interactive tasks such as
games and some applications.

The Commodore 65 provides dedicated hardware for copying and processing large
amounts of memory, a system known as {\em Direct Memory Access} (DMA). The DMA
hardware performs memory tasks, known as {\em jobs}. The MEGA65 extends these
capabilities with higher speeds and features that specialise in
graphics and audio data.

The MEGA65 DMA is unlike the C65 DMA (and DMA in other microcomputers) in an
important way: in the MEGA65, the DMA is built into the CPU, and DMA jobs
pause execution of CPU instructions. This would seem to be against the point
of having dedicated DMA hardware, except for the fact that the MEGA65 executes
DMA jobs very quickly compared to the C65 due to the higher maximum CPU speed.
Copying a large graphic onto the screen happens quickly enough to be a regular
part of a game loop's code.

This also simplifies the programming model: a program can treat a DMA job like
a special kind of CPU instruction, without additional logic to wait for jobs
to complete. There's even a method of invoking the DMA where you can include
the DMA job list directly in your code, without having to reserve additional
memory to describe the job. In general, if the operation you're trying to
perform is supported by DMA and involves more than eight bytes, a DMA job will
be faster than equivalent CPU instructions.

This section introduces DMA briefly, as part of this overall discussion of
memory. For a complete description of the DMA controller and the MEGA65
extended features, including graphics and audio DMA, see \bookvref{cha:dmagic}.

\subsection{DMA Jobs}

To perform a DMA job, you describe the job in memory, then invoke the DMA by
writing the job description's address to registers. The DMA executes the job
as soon as you write the final job address value to the register. The CPU
pauses until the job is complete, and execution continues at the next program
instruction.

Conceptually, a job description contains the following fields:

\begin{enumerate}
\item A command: copy or fill
\item A number of bytes to process, up to 64KB
\item For copy: the source start address
\item For fill: a fill byte
\item The destination start address
\item Configuration on how the source and destination regions should be
traversed (``addressing'')
\end{enumerate}

More than one job can be described in a single invocation of the DMA, in a ``job
list'' structure. The jobs are executed in order.

The MEGA65 extends the C65 DMA job list format to include a list of
job option tokens that apply to every job in a list. Many of the MEGA65's
extended features, like copying image data with transparency, use job
option tokens.

\subsection{Using DMA from BASIC}

In BASIC 65, the {\bf EDMA} command performs a single DMA job, with limited
options. The Commodore 65 {\bf DMA} command is also supported, but EDMA is
recommended for all new programs that don't need to run on a C65.

The EDMA command accepts four comma-delimited arguments: the command (0=copy,
3=fill), the number of bytes, either a fill byte (for fill) or a source start
address (for copy), and the destination start address.

\begin{screencode}
10 SC=WPEEK($D060)+PEEK($D062)*$10000
20 FOR X=32 TO 40
30 EDMA 3,80*25,X,SC
40 GETKEY A$
50 NEXT
\end{screencode}

Other DMA features are not easily accessible from the EDMA command. To perform
more sophisticated DMA jobs, a BASIC program can POKE job data into memory and
invoke the job via registers, as described below.

\subsection{Using DMA from Machine Code}

There are several ways to describe and initiate DMA jobs, including ways to
write programs compatible with the Commodore 65. This section will describe two
methods: invoking a MEGA65 enhanced DMA job list from a data structure
stored at a memory address, and invoking an enhanced job list inline with
machine code. For a complete description of the options, including the list of
possible MEGA65 extended job tokens, see \bookvref{cha:dmagic}.

\subsection{Invoking an Enhanced Job List in Memory}

To invoke an enhanced DMA job list from a record stored in memory, set the
\$D702 register to the highest byte of the address, then set \$D701
to the middle byte, then finally set {\bf \$D705} to the lowest byte.

Notice two important things:

\begin{itemize}
\item The final low byte goes in register \$D705. This indicates that you are using an
enhanced DMA job list and not a C65-compatible DMA job list (which would put
the low byte in \$D700).
\item The registers must be set {\em in this order}, with the low
byte in \$D705 set last. Setting this register immediately pauses the execution of CPU
instructions and begins execution of the DMA job.
\end{itemize}

For example, to invoke an enhanced DMA job list stored at label {\tt dmajob}:

\begin{screencode}
LDA #^dmajob
STA $D702
LDA #>dmajob
STA $D701
LDA #<dmajob
STA $D705  ; DMA executes immediately

; Program continues...
RTS

dmajob:
!byte $0B, $00       ; token list
!byte $03, $00, $08  ; fill ($03) $0800 bytes
!byte $BB, $00, $00  ; with value $BB
!byte $00, $30, $04  ; starting at 4.3000
!byte $00, $00, $00  ; no funny business
\end{screencode}

The {\tt \^{}dmajob} above (a caret followed by a symbol name) is ACME
assembler syntax for the bank byte of the 24-bit address of the {\tt dmajob}
symbol. Adapt this to your assembler's syntax, or hard-code this value if you
know which bank will contain the {\tt dmajob} memory.


\subsection{Invoking an Enhanced Job List Inline with Code}

As an alternative to preparing a job list in memory, you can put the job list
inline with your machine code. This is a convenient way to write a DMA job as
if it were an instruction with pre-determined values. To do so, use the
\stw{STA \$D707} instruction (with any accumulator value), followed by assembler
directives to assemble the enhanced job list data structure. The 45GS02 CPU
knows to give control to the DMA and advance the program counter beyond the end
of the DMA job data.

\begin{screencode}
STA $D707  ; DMA executes immediately
!byte $0B, $00       ; token list
!byte $03, $00, $08  ; fill ($03) $0800 bytes
!byte $BB, $00, $00  ; with value $BB
!byte $00, $30, $04  ; starting at 4.3000
!byte $00, $00, $00  ; no funny business

; Program continues...
RTS
\end{screencode}

\subsection{The Enhanced Job List Format}

This section presents a simplified version of the enhanced job list format.
For a complete description, including fields not yet supported by the 45GS02
DMA and the complete list of job option tokens, see \bookvref{cha:dmagic}.

The {\em job list} is a binary format with values packed into bytes.
It consists of one or more job records, where the ``chain'' bit is set for every
job except the final job in the list.

An enhanced {\em job record} consists of a list of one or more job
option tokens that always ends with the ``end of job option list'' token (\$00),
followed by a 12-byte data structure. Each job option token is a byte value.
Some tokens are followed by a one-byte argument value. Unless you're writing a
program intended for a Commodore 65, your token list will typically use the
\$0B token (no arguments) to select the 12-byte job record format.

The 12-byte structure is as follows:

\begin{center}
\begin{tabular}{|c|p{6cm}|}
\hline
{\bf Bytes} & {\bf Contents} \\
\hline

\$00 & Command \\
& {\bf 1 -- 0}: 00 for copy, 11 for fill \newline
{\bf 2}: 0 if this is the last job in the list, 1 if list continues \newline
{\bf 3}: Yield to interrupts \\
\hline

\$01 -- \$02 & Count (16 bits) \\
\hline

\$03 -- \$05 & Source address or fill byte \\
& Fill byte in \$03 \newline
Source address (20-bit) in \$03 -- \$05 \newline
High bits of \$05: \newline
{\bf 6}: Direction: 0 to count up from start, 1 to count down from start \newline
{\bf 7}: I/O: 1 to access I/O registers at \$D000 -- \$DFFF during DMA \\
\hline

\$06 -- \$08 & Destination address (20-bit) \\
& Uses same format as source address \\
\hline

\$09 & Addressing options \\
& {\bf 1 -- 0}: Addressing mode of source \newline
{\bf 3 -- 2}: Addressing mode of destination \newline
Addressing modes: \newline
00: Linear: advance in direction for all bytes \newline
10: Hold: do not advance, only repeat start address \\
\hline

\$0A -- \$0B & Reserved, always set to \$00 \$00 \\
\hline

\end{tabular}
\end{center}

Source and destination addresses are stored in the job record as 20-bit
addresses, in Little Endian order. To access the full 28-bit address space, use
the job option tokens for setting the highest byte of the start (\$80 \$xx) and
destination (\$81 \$xx) addresses. The token applies to all jobs in the job list.

Example: copy all of BASIC program text memory into Attic RAM.

\begin{screencode}
dmajob:
!byte $80, $00       ; Source highest byte: $00
!byte $81, $80       ; Destination highest byte: $80
!byte $0B            ; Use 12-byte job record format
!byte $00            ; End of token list
!byte $00, $00, $d7  ; copy ($00) $d700 bytes
!byte $00, $20, $00  ; from (00)0.2000
!byte $00, $00, $00  ; to (80)0.0000
!byte $00, $00, $00  ;
\end{screencode}

Example: a chain of three jobs, each performing a fill.

\begin{screencode}
dmajob:
!byte $0B, $00       ; token list
!byte $07, $00, $01  ; fill ($03) $0100 bytes; continue job chain (+$04)
!byte $BB, $00, $00  ; with value $BB
!byte $00, $30, $04  ; starting at 4.3000
!byte $00, $00, $00  ;

!byte $0B, $00       ; token list
!byte $07, $00, $01  ; fill ($03) $0100 bytes; continue job chain (+$04)
!byte $CC, $00, $00  ; with value $CC
!byte $00, $31, $04  ; starting at 4.3100
!byte $00, $00, $00  ;

!byte $0B, $00       ; token list
!byte $03, $00, $01  ; fill ($03) $0100 bytes; end job chain
!byte $DD, $00, $00  ; with value $DD
!byte $00, $32, $04  ; starting at 4.3200
!byte $00, $00, $00  ;
\end{screencode}

DMA jobs ignore all mapping and banking settings. They access I/O registers at
\$D000 -- \$DFFF only when requested by the flag at bit 23 of the source or
destination address.

You can imagine a DMA job maintaining two cursors, one for the source (when
copying) and one for the destination. The DMA advances each cursor according to
the job description, for as many steps as requested by the count, copying or
filling into the destination at each step.

Addressing modes are one way to influence how the cursors move:

\begin{itemize}
\item The {\em linear} addressing mode advances the cursor by one address for each
step.
\item The {\em hold} addressing mode does not advance the cursor. It remains at the
start address for every step.
\item The {\em traversal direction} says whether to increment or decrement for each
step during linear addressing.
\end{itemize}

Example: A copy that holds the source cursor and advances the
destination cursor linearly is similar to performing a fill with the byte at
the source address.

\begin{screencode}
dmajob:
!byte $0B, $00       ; token list
!byte $00, $80, $00  ; copy ($00) $0080 bytes
!byte $00, $19, $00  ; from 0.1900
!byte $40, $31, $04  ; to 4.3140
!byte $02, $00, $00  ; hold the source
\end{screencode}

The MEGA65 extended DMA is capable of many more sophisticated traversal
patterns, controlled by job option tokens. You can use the DMA to draw lines
and patterns into screen and colour memory.

Commodore described but never completed their implementation of the Commodore
65 DMA. The MEGA65 reserves bits of the DMA job record for these incomplete
features for historical preservation purposes, even though an implementation
can never be truly compatible with the Commodore 65 prototype.


\newpage
\section{Advanced Memory Topics}

\subsection{Testing Memory Mapping with the Matrix Mode Debugger}

In the Matrix Mode debugger,\footnote{Remember: To activate the Matrix Mode
debugger, hold down \megasymbolkey and press \specialkey{TAB}. For more
information about the Matrix Mode debugger, see \bookvref{sec:matrix-mode}.}
the \stw{m} command displays data at a given memory location. For example,
to display the contents of RAM starting at address 0.2000:

\begin{screencode}
m2000
\end{screencode}

The debugger {\em always} reads from 32-bit addresses and does not take MAP or
other memory banking features into account, even when given a four hex digit
address. To make it easier to see how the CPU sees the 16-bit addresses, the
MEGA65 puts the CPU's view at the upper addresses 777.0000 -- 777.FFFF. For
example, to display what the CPU sees at the 16-bit address \$2000:

\begin{screencode}
m7772000
\end{screencode}

The 777 bank takes the MAP register into account, as well as the memory
mapping mechanisms described in the following sections.

\subsection{C64-style Memory Banking}

The 16-bit addresses \$0000 and \$0001 refer to registers. These registers
function identically to the CPU registers found at these addresses in a
Commodore 64. They control mapping of the C64 ROM into the 16-bit address
space, just as they do on a C64.

In a MEGA65, these registers serve two purposes: they bank in the I/O registers
at \$D000 -- \$DFFF to overlay Chip RAM, and they provide access to the C64
BASIC and kernel for C64 mode (\stw{GO64}).

To bank out the I/O registers such that the 16-bit addresses \$D000 -- \$DFFF
access the underlying Chip RAM at 0.D000 -- 0.DFFF, clear bits 0 and 1 of
location \$0001. To restore the I/O registers, set bits 0 and 1 of location
\$0001.

This BASIC example demonstrates accessing location \$D020 as an I/O register
(the border colour) then as a RAM address. Try to guess what it will print
before running it.

\begin{screencode}
10 POKE $D020,3
20 POKE 1,240
30 POKE $D020,4
40 X=PEEK($D020)
50 POKE 1,255
60 PRINT X
70 PRINT PEEK($D020)
\end{screencode}

C64-style banking only applies to 8KB address blocks not selected by the MAP
register (``unmapped'' blocks). For example, if C64-style banking has I/O
registers at \$D000 -- \$DFFF, and the \$C000 -- \$DFFF region is mapped with
an offset of 0, the program will access RAM at that location and {\em not} the
I/O registers. The \$C000 -- \$DFFF region must be unmapped for the CPU to
access the I/O registers.

Similarly, it is possible to use the MAP register to access the RAM at
addresses 0.0000 and 0.0001 that are normally hidden by the C64-style banking
registers. If the MAP register selects the \$0000 -- \$1FFF address block for
mapping and applies an offset of 0, 16-bit addresses \$0000 and \$0001 will
access RAM at 0.0000 and 0.0001, and will not access the C64-style banking
registers. The C64-style banking registers can only be accessed with
the \$0000 -- \$1FFF region unmapped.

It is typical for a MEGA65 program to leave I/O registers banked into \$D000,
and use the MAP register when it needs to access the underlying RAM. Such a
program does not need to manipulate the C64-style banking register.

\subsection{The \$D030 VIC-III Banking Register}

The Commodore 65 provides a banking mechanism to make additional ROM data
accessible with 16-bit addresses. This banking is controlled by the I/O
register \$D030.\footnote{Accessing the VIC-III banking register at \$D030
requires that the computer be using the VIC-III/IV ``I/O personality,''
described later. This is the default in MEGA65 mode.}

Bits 3, 4, 5, and 7 map regions of ROM data into the 16-bit address space according to
this table. (Recall that bits are numbered right to left from 0, so bit 7 is the
highest bit.)

\begin{center}
\begin{longtable}{|L{1.5cm}|L{1.5cm}|L{1.5cm}|L{1.5cm}|}
\hline
{\bf \$D030 Bit} & {\bf Signal Name} & {\bf 20-bit Address} & {\bf 16-bit Address} \\
\hline
\endfirsthead
\multicolumn{4}{l@{}}{\ldots continued}\\
\hline
{\bf \$D030 Bit} & {\bf Signal Name} & {\bf 20-bit Address} & {\bf 16-bit Address} \\
\endhead
\multicolumn{4}{l@{}}{continued \ldots}\\
\endfoot
\hline
\endlastfoot
\hline
3 & ROM8 & \$28000 -- \$29FFF & \$8000 -- \$9FFF \\
\hline
4 & ROMA & \$2A000 -- \$2BFFF & \$A000 -- \$BFFF \\
\hline
5 & ROMC & \$2C000 -- \$2CFFF & \$C000 -- \$CFFF \\
\hline
7 & ROME & \$2E000 -- \$2FFFF & \$E000 -- \$FFFF \\
\end{longtable}
\end{center}

For example, if \$D030 is set to \$64 = ``0 1 1 0 0 1 0 0,'' then ROMC is
enabled, and the 16-bit addresses \$C000 -- \$CFFF access 2.C000 -- 2.CFFF.

As with C64-style banking, \$D030 banking only applies to 8KB address blocks not
selected for offsetting by the MAP register. For unmapped regions, if either
C64-style ROM banking or \$D030 ROM banking is enabled for the region,
then those addresses refer to ROM in bank 2. Each of these ROM banking
mechanisms use ROM in bank 2, so there is no conflict if they are both
enabled.

Notice that \$D030 does not bank in any of the MEGA65 kernel ROM from bank 3.
The kernel itself uses the MAP register to access its own code, and only uses
\$D030 banking to access ROM code at 2.C000 with ROMC enabled. As with the
C64-style banking register, a typical MEGA65 program can leave the banking
bits of \$D030 alone and use MAP to access RAM or ROM at these addresses as
needed.

\subsection{Cartridge ROM}

When a Commodore 64 cartridge is inserted in the expansion port, the memory
system overrides regions of the 16-bit address space to access lines to the
port, typically connected to ROM chips in the cartridge. The cartridge
controls two additional lines that allow it to request different memory
configurations using the \$0000 and \$0001 CPU registers, known as {\bf EXROM}
and {\bf GAME}. These configurations are identical to that of a Commodore 64.
See a Commodore 64 reference book for more information.

The Commodore 64 recognizes that a C64 cartridge is installed if one or both
of EXROM and GAME are low (0). A typical C64-style cartridge uses one or two
8KB regions made accessible to the system based on the EXROM and GAME values.
If both lines are high, then no cartridge ROM is mapped to 16-bit addresses.

Cartridge hardware is connected to the entire 16-bit address bus, and can
potentially return values for all 65,536 addresses. On the MEGA65, all 64KB of
the cartridge addresses are available at the dedicated upper addresses
400.0000 -- 400.FFFF.\footnote{The MEGA65 team is developing a protocol and
tools for making MEGA65 cartridges that boot automatically in MEGA65 mode.
This will be documented in a later version of this manual.}

MAP register address translation takes precedent over C64 cartridge memory
configuration. Cartridge ROM mapping overrides \$D030-style memory mapping.

\subsection{I/O Personalities}

The I/O registers are the primary way the kernel, the BASIC interpreter, or a
machine code program engage with the MEGA65 hardware. These registers behave
like memory locations for setting values in hardware such as displaying
graphics or making sound, or reading values from hardware such as user input on
the keyboard or joystick. Programs written in BASIC typically access these
registers indirectly through commands, though they can also do so directly via
memory access.

The MEGA65 starts in a configuration that makes most of its features available
via registers at 16-bit addresses in \$D000 -- \$DFFF. There are four different
configurations for these registers, known as {\em I/O personalities:}

\begin{itemize}
\item {\bf MEGA65}. Enables all features of the MEGA65 VIC-IV and Commodore 65
VIC-III. This is the default.
\item {\bf Commodore 65}. The VIC-III without the VIC-IV enhancements. For C65 compatibility.
\item {\bf Commodore 64}. The VIC-II. Used by C64 mode (\stw{GO64}).
\item {\bf MEGA65 Ethernet}. A special-purpose personality for accessing the
MEGA65 Ethernet port.
\end{itemize}

The registers for all four personalities are available in the 28-bit address
space in each 4KB block of the range FFD.0000 -- FFD.3FFF. When a given
personality is configured, \$D000 -- \$DFFF is mapped to the corresponding
block of upper addresses. For example, the default MEGA65 personality maps
these addresses to FFD.2000 -- FFD.2FFF. You can access all four personalities
simultaneously using 28-bit addresses, though beware of interactions between
them (such as VIC ``hot registers'').

Note that the \$D030 banking register is not available in the C64 I/O
personality. If your program launches in C64 mode or otherwise switches to its
I/O personality, it must change to the C65 or MEGA65 personality before
accessing this register.

A typical program written for the MEGA65 uses the MEGA65 personality and
assumes that it is already configured when it launches. For information on how
to change which personality is being used, such as to defensively set the
desired personality when your program launches, see \bookvref{cha:memory-map}.

\subsection{Converting ROM to RAM}

The MEGA65 stores kernel and BASIC program code in banks 2 and 3. To protect
this code from being overwritten by other programs, it enforces that these
banks are read-only, simulating having Commodore-style ROM chips connected to
those addresses. The MEGA65 Hypervisor loads the ROM data during boot, and
manages the write protection.

A program can disable write protection on banks 2 and 3 to use those regions
as Chip RAM. Most programs don't need all of ROM in memory. For example, a
program running in C65 mode is unlikely to need C64 BASIC ROM code. The Chip
RAM Memory Map describes which ROM regions are used for each purpose.

A carefully written program that doesn't need the built-in routines at all can
repurpose the entire 128KB region. For example, it can load in alternate BASIC
and kernel code. Naturally, this overrides the MEGA65 ROM that the user has
installed, and so can't take advantage of improvements in newer versions of
the ROM.

To toggle the read-only status of banks 2 and 3, invoke the Hypervisor Toggle
ROM Write-protect system trap ({\tt hyppo\_toggle\_rom\_writeprotect}), like so:

\begin{screencode}
LDA #$70
STA $D640
CLV
\end{screencode}

You must use the MAP register, 32-bit instructions, or DMA to write to RAM in
banks 2 and 3. When C64-style banking or \$D030 banking is active for unmapped
16-bit addresses, reading those addresses will access bank 2, but writing to
those addresses will write to RAM in bank 0. (This is similar to what happens
when writing to a ROM-banked address on a Commodore 64.)

The state of this write-protection is not readable in a register. To test
whether the region is write-protected, simply attempt to write a value to the
region. The MEGA65 boots with write protection enabled.

The \$D640 register is only available in the MEGA65 I/O personality.

Notice that this is {\em not} a banking mechanism, like it is on a Commodore
64. If you want access to the original ROM code after overwriting it (such as
to restore it), use DMA to copy the original ROM to Attic RAM first.

\subsection{Using MAP to Access Upper Memory}

As discussed earlier, the MAP register can remap any 8KB block to another
location in the first 1MB of the address space. This is useful for
accessing the 384KB of Chip RAM, including potential future expansions. The MAP
instruction originates with the Commodore 65's 4510 CPU, which was designed to
have a 1MB address space.

The MEGA65's 45GS02 CPU extends the MAP register with the ability to access any
256 byte block in the 28-bit address space. This takes the form of an
additional high byte of offset for each of MAPLO and MAPHI.

You set a MAP offset with a high byte by calling the {\bf MAP} instruction twice
before calling {\bf EOM}. The first call sets the high byte, and the second
call sets the selection flags and remaining 12 bits of the offset, as before.
Interrupts are disabled automatically between the first {\bf MAP} call and the
{\bf EOM} call.

To set the offset high byte, load the high byte into the A register for MAPLO,
or Y register for MAPHI. Then load the special value \$0F into the X register
for MAPLO, or Z register for MAPHI. Call {\bf MAP} to set it. Then load the
selection flags and remaining offset values, call {\bf MAP}, then finally call
{\bf EOM}.

For example, to map \$A000 -- \$BFFF to \$8000000 -- \$8001FFF in Attic RAM,
you use an offset of \$7FF6000. The high byte is \$7F, and the three nibbles
of the regular offset are \$F60. Setting MAPHI to this offset with
the selection flag of \$2 will apply the offset to the desired block.

\begin{screencode}
LDA #$00
LDX #$00
LDY #$7F  ; MAPHI offset high byte of $7F
LDZ #$0F  ; (Special value for setting high byte)
MAP

LDA #$00
LDX #$00
LDY #$60  ; MAPHI offset: $F60
LDZ #$2F  ; MAPHI selection flag: $2 = A000 - BFFF
MAP
EOM
\end{screencode}

Be careful not to change the high byte of the 32KB address region that contains
the code doing the mapping! The high byte settings takes effect upon the first
MAP instruction, and if the code is accidentally mapped out, the program
counter won't be able to reach the remaining instructions, even if the final
mapping is compatible.

In most cases, it is easier for programs to access upper memory through 28-bit
addressing modes or DMA. This extended MAP feature is included for completeness.

\subsection{Reading the MAP Register}

The 45GS02 CPU can set the MAP register using a combination of CPU
instructions, but it does not offer a way to read the register in a similar
way to other CPU registers. Instead, a program can use a Hypervisor trap
for this purpose. The {\tt hyppo\_get\_mapping} Hypervisor trap writes the MAPHI
and MAPLO register values, including megabyte offsets, to six bytes at a
requested location in memory.

The memory location to which the Hypervisor trap writes these values must be
on a page boundary between 0.0000 and 0.7E00. You set the Y register to the
most significant byte of the starting address when calling the trap. For
example, to copy the MAP register to RAM at addresses 0.6200 -- 0.6205:

\begin{screencode}
LDY #$62

LDA #$74
STA $D640
CLV
\end{screencode}

\subsection{Making All Chip RAM Available}

Using the techniques described in this chapter, you can configure the memory
system to make all of Chip RAM available at addresses 0.0000 -- 5.FFFF. A
carefully written program can take over the entire machine, access I/O
registers with upper addresses, and use all of Chip and Attic RAM.

A summary of the procedure:

\begin{enumerate}
\item Disable interrupts.
\item Clear all bits of \$D030.
\item Set the MAP register to all zeroes.
\item Clear bits 0, 1, and 2 of \$0001 to remove I/O registers from \$D000 -- \$DFFF.
\item Move the start of colour RAM forward by at least \$0800 (2KB), so the
Colour RAM Window can be used as general purpose RAM.
\item Disable ROM write-protect on banks 2 and 3.
\item Load custom interrupt handler code into bank 0, configure interrupt
vectors, then re-enable interrupts. (Or just leave interrupts disabled.)
\end{enumerate}

As described earlier, the program must only use display modes that need 30KB or
less colour data in order to use the first 2KB as general purpose memory. A
program that uses all 32KB of colour memory must allow the first 2KB to be
visible in the Colour RAM Window at 1.F800 -- 1.FFFF. (There is no requirement
to bank this memory into \$D800 -- \$DFFF.)

Also as described earlier, the program must use MAP to access RAM at 0.0000
and 0.0001.

\subsection{Why the SYS Command Can Only Use Certain Addresses}

Earlier, this chapter explained that the BASIC {\bf SYS} command has
unintuitive restrictions on which addresses it can access. This can now be
explained in full, using the more advanced concepts in this chapter.

For BASIC functions and commands such as {\bf PEEK} and {\bf POKE}, the KERNAL
uses the 45GS02's 32-bit indirect addressing feature to access large
addresses. Given an address of \$10000 or larger, the command loads the full
address into a base page variable, then invokes 32-bit indirect addressing to
access the memory. Given an address of \$0000 -- \$FFFF with a {\bf BANK}
setting of 0 -- 5, the command uses the {\bf BANK} setting as the upper part
of the address. 32-bit indirect addressing ignores the MAP register and the
memory banking features, and can access any address in the computer.

If the address is \$0000 -- \$FFFF and the {\bf BANK} setting is 128, BASIC
uses the CPU's 16-bit addressing mode, which honors the KERNAL's MAP register
setting, as well as D030 and C64-style memory banking settings. This is how
those commands access ROM and I/O registers with \stw{BANK 128}.

{\bf SYS} tells the CPU to execute code at a location in memory. The CPU's
program counter is a 16-bit register, and it cannot access larger addresses
independently of the MAP register and other memory mapping features the way
32-bit indirect addressing can. As a feature of the KERNAL, it must keep
KERNAL ROM in memory at 16-bit addresses \$E000 -- \$FFFF to handle
interrupts. It must also keep KERNAL variables accessible in addresses \$0000
-- \$1FFF for the interrupt handler code to use. Code called by {\bf SYS} can
make its own decisions about whether to keep the KERNAL functional, but {\bf
SYS} keeps it enabled for the common case where the machine code wants to
return control to a running BASIC program or the \screentext{READY.} prompt.

The KERNAL uses the MAP register to access ROM in bank 3. To keep ROM in
\$E000 -- \$FFFF, it sets the MAPH offset to 3, and marks that memory region
as ``mapped.'' There is only one offset that MAP can use for all 8KB regions in
\$8000 -- \$FFFF, so on entry to a {\bf SYS} call, the other regions in that
range must either be un-mapped and refer to bank 0 (or D030/\$01 banking), or
mapped to bank 3. The most useful choice for {\bf SYS} is to leave the
remaining upper regions un-mapped.

Similarly, for addresses in the range \$0000 -- \$1FFF to continue to refer to
KERNAL variable memory in bank 0, either the region must be un-mapped, or
mapped with an offset of zero. Naturally, it is more useful for {\bf SYS} to
leave that region un-mapped so that it can access other banks for addresses
\$2000 -- \$7FFF, so that's what it does. {\bf SYS} uses the MAP register to
access upper banks for an address whose lower word is \$2000 -- \$7FFF, either
when using {\bf BANK} or when given a large address.

The MAP register is much more flexible than this: it can allow almost any
address to be reached via 16-bit addresses \$2000 -- \$7FFF using a wide range
of possible offsets. However, the machine code called by {\bf SYS} would need
to be written in a way that is aware of what offset is being used. The program
counter would be relative to the offset, and a {\tt JMP} instruction in the
code would have to use an address based on the same offset. It is possible to
write machine code using only relative branch instructions (such as {\tt
BRA}), but historically this has not been a requirement of the {\bf SYS}
command. Instead, {\bf SYS} restricts itself to 64KB offsets to keep the
programming model simple.
