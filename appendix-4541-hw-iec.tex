\chapter{4541 Serial Bus Controller}

\section{Overview}

The 4541 is a Commodore{\texttrademark} serial peripheral bus compatible
bus controller, that greatly reduces the effort required to
communicate with devices on this bus.

\section{Features of the 4541}

\subsection{Supports Enhanced Serial Protocol Varians}

The 4541 supports the
JiffyDOS{\texttrademark} extensions to this protocol, that allow data
transfers approximately 10$\times$ faster than using the original
protocol. It is also expected that a future revision of the 4541 will
support Commodore's fast serial protocol, that is present in the 1571
and 1581 disk drives. 

\subsection{Interrupt Enabled Processor Offload}

The 4541 performs serial communications independent of the
microprocessor. Together with an IRQ functionality, this allows
software to continue on other tasks while serial peripheral
communications occurs, requiring only to be briefly interrupted when
the next event on the serial peripheral bus occurs.

\subsection{Processor Speed Independence}

A major advantage of the 4541, is that it also handles all timing
requirements of communications on this bus, allowing the bus to be
driven by a processor that can run at different speeds, without having
to modify the the bus controller software.

\subsection{Co-Existence through Open-Collector Logic}

Because the 4541 uses open-collector
logic, it can be used in parallel with existing software-based
implementations of the bus protocol, ensuring compatibility with
existing software.  It is installed in this configuration in the
MEGA65, allowing the legacy software-based serial peripheral
communications software that controls the serial peripheral bus to
continue to be used unmodified. 

\section{Theory of Operation}

The 4541 presents a quite simple interface: You issue commands, wait
for a response, and retrieve any data that the command
retrieved.  Some commands also require a data byte, which is provided
by a dedicatetd register.  There is also a device info register, that lets you see
what the 4541 believes about the current status of the most recently
requested device, including whether it is present, and whether it
supports either or both of the JiffyDOS{\texttrademark} or
Commodore{\textrademark} 128 extensions to the standard protocol.

So, for example, to release the Attention line, you can simply write
the appropriate command byte value (65 = \$41) to the command register
at \$D698, and then check for the completion status in the status
register at \$D697, as shown in the following example written in
BASIC65:

\begin{screencode}
  10 POKE $D698,$41
  20 IF ( PEEK($D697) AND $20 ) = $00 GOTO 20
  30 PRINT "DONE"
\end{screencode}

The 4541 implements a set of commands that map very closely to the
KERNAL calls that are used to control the IEC bus on the C64 and
related computers.  In most cases, there is a single corresponding
command for the 4541, although in a few cases, you may need to issue
two commands, as summarised in the following table:

\index{IEC Controller Commands}
\begin{center}
    \setlength{\def\arraystretch{1.5}\tabcolsep}{6pt}
    \begin{longtable}{|c|L{8cm}|}
        \hline
        \textbf{KERNAL Call} & \textbf{Meaning and Equivalent 4541 Command(s)}\\
        \hline
        \endhead
        \$FF93 LSTNSA & Send LISTEN secondary address.  \\
         & 4541 Equivalent: Data = Binary OR of \$60 and the desired
        secondary address.  Command \$30. Then Command \$41 to release
        the Attention line. \\
        \hline
        \$FF96 TALKSA & Sent TALK secondary address.  \\
         & 4541 Equivalent: Data = Binary OR of \$60 and the desired
        secondary address.  Command \$30. Then Command \$41 to release
        the Attention line. \\
        \hline
        \$FFA5 IECIN & Receive a byte from the serial peripheral bus.  \\
         & 4541 Equivalent : Command \$32. Received byte is available
        in the data register on completion.  \\
        \hline
        \$FFA8 IECOUT & Send a byte to the serial peripheral bus.  \\
         & 4541 Equivalent : Data = the byte to send. Command \$31 (or \$30 if the byte is to
        be sent under Attention). \\
        \hline
        \$FFA8 UNTALK  & Send UNTALK command to serial peripheral bus.  \\
        & 4541 Equivalent : Data = \$5F. Command \$30. \\    
        \hline
        \$FFAB UNLISTN  & Send UNLISTEN command to serial peripheral bus.  \\
        & 4541 Equivalent : Data = \$3F. Command \$30. \\    
        \hline
        \$FFB1 LISTEN  & Send LISTEN command to the serial peripheral bus.  \\
        & 4541 Equivalent : Data = \$20 plus the device number. Command \$30. \\    
        \hline
        \$FFB4 TALK  & Send TALK command to the serial peripheral bus.  \\
        & 4541 Equivalent : Data = \$40 plus the device number. Command \$30. \\    
        \hline
        \$FFB7 READST & Read the status of the serial peripheral bus.  \\
        & 4541 Equivalent : Read the status bits from \$D698. For
        convenience, the
        upper bits of this status byte have the same layout as used in
        the KERNAL. \\    
        \hline
    \end{longtable}
\end{center}

The 4541 is very obedient: If you ask it to do something new, it
will start doing that immediately, even if it was in the middle of
doing something else -- even if it was half-way through sending a byte
of data to the serial peripheral bus!  

You should therefore always wait until the IRQREADY bit in \$D697 is
set before issuing each command, or reading the status bits, to make
sure that the controller has finished whatever it was last asked to
do.

\section{Examples}

\subsection{Reading the DOS channel status}

The following BASIC65 program can be used to talk to a device
connected to the serial peripheral bus. Note that because it uses the
4541, and not the C65/MEGA65 CBDOS, it ignores the presence of the
internal 3.5'' disk drive, because that drive is not connected to the
serial peripheral bus.

\begin{screencode}
10 REM ABORT ANY EXISTING 4541 COMMAND IN PROGRESS
20 POKE $D698,$00
30 REM RESET THE SERIAL PERIPHERAL BUS BY PULLING
40 REM THE RESET PIN LOW FOR 1 SECOND
50 POKE $D698,$72 : SLEEP 1 : POKE $D698,$52
60 REM GIVE CONNECTED DEVICES TIME TO GET READY AFTER RESET
70 SLEEP 2: REM THE 1541 TAKES QUITE A WHILE TO RESET!
80 REM SELECT THE DEVICE NUMBER TO TALK TO
90 D = 8
100 PRINT"DEVICE"; D; "TALK"
110 POKE $D699,$40 + D : POKE $D698,$30 : GOSUB 1000
120 PRINT "SECONDARY ADDRESS 15"
130 POKE $D699,$6F : POKE $D698,$30 : GOSUB 1000
140 PRINT"TURN AROUND TO LISTEN"
150 POKE $D698,$35 : GOSUB 1000
160 PRINT "READ DOS STATUS"
170 POKE $D698,$32 : GOSUB 1000
180 PRINT CHR$( PEEK($D699) );
190 CC = CC + 1 : IF CC < 200 GOTO 180
999 END
1000 REM WAIT FOR 4541 TO FINISH COMMAND
1010 S=PEEK($D697) : IF (S AND 32) = 32 GOTO 1020
1020 GOTO 1000
1030 IF S AND 128 THEN PRINT "DEVICE NOT PRESENT" : END
1040 RETURN
\end{screencode}

If you have a 1581 with a JiffyDOS{\texttrademark} ROM connected, this
program will display something like the following when run:

\begin{screencode}
READY.
RUN
DEVICE 8 TALK
SECONDARY ADDRSS 15
TURN AROUND TO LISTEN
READ DOS STATUS
73,(C) 1989 JIFFYDOS 6.0 1581,00,00
00, OK,00,00
00, OK,00,00
00, OK,00,00
...
00, OK,0
READY.
\end{screencode}

\section{4541 Command Reference}

\section{Register Table}

The 4541 has the following registers:

\input{regtable_AUTOIEC.MEGA65}

\section{Optional Integrated Data-Logger}

The 4541 is available in a variant that contains an embedded serial
peripheral bus data-logger. This is designed to aid with debugging
protocol errors on this bus. It commences capturing data whenever
command \$30 (send byte under attention) is issued to the 4541, and
will capture data for approximately 4 milli-seconds.

\begin{itemize}
  \item {\bf DATALOG0} The low-byte of the optional integrated serial
    peripheral bus data logger.  Writing \$00 to this register causes
    the read pointer to the data log to be reset to the beginning of
    the capture. Writing any other value to this register, causes the
    read point of the data logger to be advanced by one time step.
    The time steps are approximately 1 micro-second, with additional
    time steps added whenever one of the signals on the serial
    peripheral bus changes state.
    \item {\bf DATALOG1} The high-byte of the optional integrated serial
      peripheral bus data logger.
\end{itemize}

Each time-step records 16 bits of information about the serial
peripheral bus:
\begin{itemize}
\item {\bf DATALOG0 Bit 0} DATA line input value
\item {\bf DATALOG0 Bit 1} CLK line input value
\item {\bf DATALOG0 Bit 2} SRQ line input value
\item {\bf DATALOG0 Bit 3} DATA line output state
\item {\bf DATALOG0 Bit 4} CLK line output state
\item {\bf DATALOG0 Bit 5} SRQ line output state
\item {\bf DATALOG0 Bit 6} ATN line output state
\item {\bf DATALOG0 Bit 7} RESET line output state
\item {\bf DATALOG1} The lower 8-bits of the current state of the 4541's internal state machine.
\end{itemize}

The input values are the voltages that the 4541 reads on the
respective pins.  Separately, the data-logger records whether the 4541
is pulling each of those lines low.  This allows determination as to
whether a connected peripheral or the 4541 is pulling a given signal
low. This provides much more information than simply probing the lines
of the serial peripheral bus, where you cannot readily determine who
has pulled a given line low.

The following truth table explains how to interpret these signals:

\begin{center}
    \setlength{\def\arraystretch{1.5}\tabcolsep}{6pt}
    \begin{longtable}{|c|c|L{8cm}|}
        \hline
        \textbf{Input Value} & \textbf{Output Value} &
        \textbf{Meaning}  \\
        \hline
        \endhead
        1 & 1 & Signal is floating at 5V. No device is pulling it
        low. \\
        \hline
        0 & 1 & Signal is at 0V. The 4541 is not pulling it low,
        either a connected peripheral or the CIA is pulling it low. \\
        \hline
        0 & 0 & Signal is 0V. The 4541 is pulling it low. Other
        devices might also be pulling it low, but it's not possible to
        discriminate between these two situations.
        low. \\
        \hline
        1 & 0 & Signal is floating at 5V, but the 4541 is pulling it
        to 0V. Your computer is probably broken, or someone has
        connected 5V to the signal, without a current-limiting
        resistor, in which case, your computer is about to be broken. \\
        \hline
    \end{longtable}
\end{center}
